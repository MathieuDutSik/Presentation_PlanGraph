\documentclass[12pt]{article}
\usepackage{amsfonts, amsmath, latexsym, epsfig}
\usepackage{color}
\usepackage{epsf}
\usepackage{url}
\pagestyle{empty}
\title{Elementary polycycles}
%\usepackage{vmargin}
%\setpapersize{custom}{21cm}{29.7cm}
%\setmarginsrb{1.7cm}{1cm}{1.7cm}{3.5cm}{0pt}{0pt}{0pt}{0pt}
%marge gauche, marge haut, marge droite, marge bas.

\setlength{\textwidth}{6.3in}
\setlength{\textheight}{8.7in}
\setlength{\topmargin}{0pt}
\setlength{\headsep}{0pt}
\setlength{\headheight}{0pt}
\setlength{\oddsidemargin}{0pt}
\setlength{\evensidemargin}{0pt}

\makeatletter
\newfont{\footsc}{cmcsc10 at 8truept}
\newfont{\footbf}{cmbx10 at 8truept}
\newfont{\footrm}{cmr10 at 10truept}
\date{}
\author{Michel DEZA\\
CNRS/ENS, Paris, and Institute of Statistical Mathematics, Tokyo,\\
\ Mathieu DUTOUR SIKIRI\'C\\
ENS, Paris, and Institut Rudjer Bo\u skovi\'c, Zagreb\\
\ Mikhail SHTOGRIN\\
Steklov Mathematical Institute, Moscow.
}


\begin{document}
\newcommand{\RR}{\ensuremath{\mathbb{R}}}
\newcommand{\NN}{\ensuremath{\mathbb{N}}}
\newcommand{\QQ}{\ensuremath{\mathbb{Q}}}
\newcommand{\CC}{\ensuremath{\mathbb{C}}}
\newcommand{\ZZ}{\ensuremath{\mathbb{Z}}}
\newcommand{\TT}{\ensuremath{\mathbb{T}}}
\newtheorem{proposition}{Proposition}
\newtheorem{theorem}{Theorem}
\newtheorem{corollary}{Corollary}
\newtheorem{lemma}{Lemma}
\newtheorem{conjecture}{Conjecture}
\newtheorem{claim}{Claim}
\newtheorem{remark}{Remark}
\newtheorem{definition}{Definition}
\newcommand{\qed}{\hfill $\Box$ }
\newcommand{\proof}{\noindent{\bf Proof.}\ \ }

\maketitle
\thispagestyle{empty}


\begin{abstract}
Given $q\in \NN$ and $R\subset \NN$, a $(R,q)$-{\em polycycle} is
a non-empty $2$-connected plane, locally finite (i.e. any circle
contain only finite number of its vertices) graph $G$
with faces partitioned in two non-empty sets $F_1$ and $F_2$, so that:

(i) all elements of $F_1$ (called {\em proper faces}) are
combinatorial $i$-gons with $i\in R$;

(ii) all elements of $F_2$ (called {\em holes}, the exterior face(s)
are amongst them) are pair-wisely disjoint, i.e. have no common vertices;

(iii) all vertices have degree within $\{2,\dots,q\}$ and all 
{\em interior} (i.e. not on the boundary of a hole)
vertices are $q$-valent.

Such polycycle is called {\em elliptic}, {\em parabolic} or
{\em hyperbolic} if $\frac{1}{q} + \frac{1}{r} - \frac{1}{2}$
(where $r={max_{i \in R}i}$)
is positive, zero or negative, respectively.\\
\par
A {\em bridge} of a $(R,q)$-polycycle is an edge, which is not on a boundary
and goes from a hole to a hole (possibly, the same).
An $(R,q)$-polycycle is called {\em elementary} if it has no bridges.
An {\em open edge} of a $(R,q)$-polycycle is an edge on a boundary,
such that each of its end-vertices have degree less than $q$.
Every $(R,q)$-polycycle is formed by the agglomeration of elementary
$(R,q)$-polycycles along their open edges.\\
\par
We classify all elliptic elementary $(R,q)$-polycycles and present
various applications of it.

\end{abstract}
\end{document}

