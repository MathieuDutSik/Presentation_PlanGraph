\documentclass[%
pdf,
%nocolorBG,
colorBG,
slideColor,
%slideBW,
%draft,
%frames
%azure
%contemporain
%nuancegris
%troispoints
%lignesbleues
%darkblue
%alienglow
%autumn
]{prosper}
\usepackage{pifont, amsmath, multicol}
%\usepackage{wrapfig,subfigure}
\usepackage{color}
%\usepackage{floatflt}
\usepackage{epsfig}
\usepackage{pifont}
%\usepackage[francais]{babel}
\usepackage[T1]{fontenc}
\usepackage[latin1]{inputenc}
%\catcode`\�=\active
%\catcode`\�=\active
%\def�{\og\ignorespaces}%
%\def�{{\fg}}%

\newcommand{\spacer}{\rule[-3mm]{0mm}{8mm}}
\newtheorem{theorem}{Theorem}
\newtheorem{definition}[theorem]{Definition}
\newtheorem{conjecture}[theorem]{Conjecture}
\newtheorem{corollary}[theorem]{Corollary}
\newtheorem{problem}[theorem]{Problem}
\newtheorem{lemma}{Lemma}
\newtheorem{claim}[theorem]{Claim}
\newtheorem{remark}[theorem]{Remark}
\newtheorem{proposition}[theorem]{Proposition}


\newcommand{\RR}{\ensuremath{\mathbb{R}}}
\newcommand{\NN}{\ensuremath{\mathbb{N}}}
\newcommand{\QQ}{\ensuremath{\mathbb{Q}}}
\newcommand{\CC}{\ensuremath{\mathbb{C}}}
\newcommand{\ZZ}{\ensuremath{\mathbb{Z}}}
\newcommand{\TT}{\ensuremath{\mathbb{T}}}

\def\QuotS#1#2{\leavevmode\kern-.0em\raise.2ex\hbox{$#1$}\kern-.1em/\kern-.1em\lower.25ex\hbox{$#2$}}


\title{\Large \textcolor{blue}{Goldberg-Coxeter construction}\\[3mm]
\textcolor{blue}{for $3$- or $4$-valent plane graphs}}
\author{
\textcolor{red}{\large Mathieu Dutour}\\[2mm]
\textcolor{red}{\Large ENS/CNRS, Paris and Hebrew University, Jerusalem}\\[2mm]
\textcolor{red}{\large Michel Deza}\\[2mm]
\textcolor{red}{\Large ENS/CNRS, Paris and ISM, Tokyo}
}
\slideCaption{}

\date{}



\begin{document}



\begin{slide}{Icosahedral, Octahedral}
\begin{itemize}
\item Icosahedral symmetry
\begin{itemize}
\item $I$(order $60$): the groups of rotations of the Icosahedron
\item $I_h$(order $120$): the groups of isometries of the Icosahedron
\end{itemize}
\item Cubic symmetry
\begin{itemize}
\item $O$(order $24$): the groups of rotations of the Cube
\item $O_h$(order $48$): the groups of isometries of the Cube
\end{itemize}
\item Tetrahedral symmetry
\begin{itemize}
\item $T$(order $12$): the groups of rotations of the Tetrahedron
\item $T_d$(order $24$): the groups of isometries of the Tetrahedron
\item $T_h$(order $24$): $T\cup -T$
\end{itemize}

\end{itemize}


\end{slide}



\begin{slide}{Dihedral symmetry}
\begin{itemize}
\item $D_m$ is the group formed of all rotations preserving the Prism of order $m$:
\item $D_{mh}$ is the group formed of all isometries preserving the Prism of order $m$:
\item $D_{md}$ is the group formed of all isometries preserving the AntiPrism of order $m$:
\item $C_m$ is the rotation group of order $m$.


\end{itemize}
\end{slide}


\end{document}
