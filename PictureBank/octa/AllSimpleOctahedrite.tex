\documentclass[12pt,epsfig,epsf]{article}
\usepackage{url}
\usepackage{epic, eepic, subfigure, floatflt}
\usepackage{amsfonts, epsfig, latexsym, amsmath}
\topmargin=-0.5truein
\oddsidemargin=0.25truein
\evensidemargin=0.25truein
\textwidth=6truein
\textheight=9truein

\begin{document}
\newcommand{\R}{\ensuremath{\mathbb{R}}}
\newcommand{\N}{\ensuremath{\mathbb{N}}}
\newcommand{\Q}{\ensuremath{\mathbb{Q}}}
\newcommand{\C}{\ensuremath{\mathbb{C}}}
\newcommand{\Z}{\ensuremath{\mathbb{Z}}}
\newcommand{\T}{\ensuremath{\mathbb{T}}}
\newtheorem{theorem}{Theorem}[section]
\newtheorem{definition}[theorem]{Definition}
\newtheorem{conjecture}[theorem]{Conjecture}
\newtheorem{corollary}[theorem]{Corollary}
\newtheorem{lemma}[theorem]{Lemma}
\newtheorem{claim}[theorem]{Claim}
\newtheorem{remark}[theorem]{Remark}
\newtheorem{proposition}[theorem]{Proposition}
\newcommand{\qed}{\hfill $\Box$ }
\newcommand{\proof}{\noindent{\bf Proof.}\ \ }






{\small
\begin{figure}
\setlength{\unitlength}{1cm}
\begin{minipage}[t]{3.5cm}
\begin{picture}(3.5,3.5)
\leavevmode
\epsfxsize=3.5cm
\epsffile{octa/4reg_21_1.ps}
\end{picture}\par
\begin{center}
{{\bf Nr.6} \quad $O_h$ \\ $(4^3)$ \\}
\end{center}
\end{minipage}
\setlength{\unitlength}{1cm}
\begin{minipage}[t]{3.5cm}
\begin{picture}(3.5,3.5)
\leavevmode
\epsfxsize=3.5cm
\epsffile{octa/4reg_21_11.ps}
\end{picture}\par
\begin{center}
{{\bf Nr.12-1} \quad $O_h$ \\ $(6^4)$ \\}
\end{center}
\end{minipage}
\setlength{\unitlength}{1cm}
\begin{minipage}[t]{3.5cm}
\begin{picture}(3.5,3.5)
\leavevmode
\epsfxsize=3.5cm
\epsffile{octa/8-hedrite12-1sec.eps}
\end{picture}\par
\begin{center}
{{\bf Nr.12-5} \quad $D_{3h}$ \\$(6^4)$ \\}
\end{center}
\end{minipage}
\setlength{\unitlength}{1cm}
\begin{minipage}[t]{3.5cm}
\begin{picture}(3.5,3.5)
\leavevmode
\epsfxsize=3.5cm
\epsffile{octa/4reg_21_20.ps}
\end{picture}\par
\begin{center}
{{\bf Nr.14-1} \quad $D_{4h}$ \\ $(6^2,8^2)$ \\}
\end{center}
\end{minipage}
\setlength{\unitlength}{1cm}
\begin{minipage}[t]{3.5cm}
\centering
\epsfxsize=2.5cm
\epsffile{octa/PureOctahedrite20.eps}\par
{{\bf Nr.20-1} \quad $D_{2d}$ \\ $(8^5)$ \\ }
\end{minipage}
% \setlength{\unitlength}{1cm}
\begin{minipage}[t]{3.5cm}
\centering
\epsfxsize=2.5cm
\epsffile{octa/PureOctahedrite22.eps}\par
{{\bf Nr.22-1} \quad $D_{2h}$ \\ $(8^3,10^2)$ \\ }
\end{minipage}
\setlength{\unitlength}{1cm}
\begin{minipage}[t]{3.5cm}
\centering
\epsfxsize=2.3cm
\epsffile{octa/oc30-1.eps}\par
{{\bf Nr.30-1} \quad $O$ \\ $(10^6)$ \\}
\end{minipage}
\setlength{\unitlength}{1cm}
\begin{minipage}[t]{3.5cm}
\centering
\epsfxsize=2.3cm
\epsffile{octa/PureOctahedrite32-1.eps}\par
{{\bf Nr.32-1} \quad $D_{4h}$ \\ $(10^4,12^2)$ \\}
\end{minipage}
\caption{All pure tight octahedrites}
\label{ThePureTightOctahedriteWith56CC}
\end{figure}
}
%jb97}













\end{document}
