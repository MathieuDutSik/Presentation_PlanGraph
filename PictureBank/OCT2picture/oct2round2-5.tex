\documentclass[12pt]{article}

%\usepackage{epic,eepic}
\usepackage{amsfonts,amsmath, epsfig,latexsym, subfigure, url}
% emlines.sty  M�rz 1990, Georg Horn / Eberhard Mattes.
%
% Makros zum Zeichnen von Linien mit beliebiger Steigung.
% Nur bei Verwendung der DVI-Treiber von Eberhard Mattes.
%
% Der Makro \emline#1#2#3#4#5#6 setzt an die Koordinaten (#1,#2) den
% Punkt #3 und an die Koordinaten (#4,#5) den Punkt #6. Diese beiden 
% Punkte werden dann verbunden.
%
% TeXcad erzeugt Aufrufe dieses Makros f�r mit eingeschalteter
% EMLines-Option gezeichnete Linien.
% (z.B. \emline{0.0}{0.0}{1}{17.0}{4.0}{2}).
%
% Der Makro \newpic#1 definiert den Makro \emline so, dass den
% Punktnummern ##3 und ##6 die Ziffer #1 vorangestellt wird.
% Dies ist notwendig, wenn mehr als ein Bild auf einer Seite
% des Dokuments eingefuegt wird.
%
\def\newpic#1{%
   \def\emline##1##2##3##4##5##6{%
      \put(##1,##2){\special{em:point #1##3}}%
      \put(##4,##5){\special{em:point #1##6}}%
      \special{em:line #1##3,#1##6}}}
%
% Standarddefinition von \emline herstellen
%
\newpic{}
%
% Beispiel: 
% \input bild1.pic
% \newpic{1} \input bild2.pic
%


\setlength{\textwidth}{15cm}
\setlength{\textheight}{23cm}
\setlength{\topmargin}{-0.5cm}
\setlength{\oddsidemargin}{0.5cm}
\setlength{\evensidemargin}{0.5cm}



\newtheorem{proposition}{Proposition}
\newtheorem{theor}{Theorem}
\newtheorem{conjecture}{Conjecture}
\newtheorem{corollary}{Corollary}
\newtheorem{lemma}{Lemma}
\newtheorem{claim}{Claim}
\newtheorem{remark}{Remark}
\newcommand{\proof}{\noindent{\bf Proof.}\ \ }

\begin{document}

\title{$4$-valent plane graph with $2$-, $3$- and $4$-gonal faces}


\author{Michel DEZA \\
  CNRS/ENS, Paris and Institute of Statistical Mathematics, 
Tokyo,\\
\ Mathieu DUTOUR \\
 ENS, Paris and The Hebrew University, Jerusalem\\
\ and  Mikhail SHTOGRIN \thanks{third author acknowledges financial support 
of the Russian Foundation of Fundamental Research (grant 02-01-00803)
and the Russian Foundation for Scientific Schools (grant 00-15-96011)}\\
Steklov Mathematical Institute, Moscow, Russia} 
\date{\today}

\maketitle



\begin{abstract}
Call {\em $i$-hedrite $oc_n$} any $4$-valent $n$-vertex plane graph, whose 
faces are $2$-, $3$- and $4$-gons only and $p_2=8-i$. The edges of an $i$-hedrite, as of 
any Eulerian plane graph, are partitioned
by its {\em central circuits}, i.e. those, which are obtained by starting with an
edge and continuing at each vertex by the edge opposite the entering one. 
So, any $i$-hedrite is a projection of an alternating link, whose components
correspond to its central circuits.

Call an $i$-hedrite {\em irreducible}, if it has no 
{\em railroad}, i.e. a 
circuit of $4$-gonal faces, in which every $4$-gon is adjacent to two of its 
neighbors on opposite edges.

We present the list of all $i$-hedrites with at most $14$ vertices. Examples of results: 

(i) Any irreducible $i$-hedrite has at most $i-2$ central circuits.

(ii) All irreducible $i$-hedrites without self-intersecting central circuits are listed.

(iii) The symmetry group of $i$-hedrites are listed.

\end{abstract}

{\em Mathematics Subject Classification}. Primary 52B05, 52B10;
Secondary 05C30, 05C10.

{\em Key words}. Polyhedra, Eulerian graphs, alternating links, point groups.

\section{Introduction}

See \cite{Gr} for terms used here for polyhedra (it means below only convex
$3$-polytopes).
It is well-known that the p-vector of any $4$-valent plane graph satisfies to
$2p_2+p_3=8+ \sum_{i\geq 5} (i-4)p_i$.
Some examples of applications of plane $4$-valent graphs are {\em projections
of links}, {\em rectilinear embedding} in VLSI and {\em Gauss crossing 
problem} for plane graphs.

\vspace{2mm}





Call an {\em $i$-hedrite} any plane $2$-connected
$4$-valent graph, such that the number
$p_j$ of its $j$-gonal faces is zero for any $j$, different from 
$3,4$ and $2$, and such that $p_2=8-i$. So, 
an $n$-vertex $i$-hedrite has $(p_2, p_3, p_4)=(8-i, 2i-8, n+2-i)$.
Clearly, $(i;p_2,p_3)=(8;0,8)$, $(7;1,6)$, $(6;2,4)$,
$(5;3,2)$ and $(4;4,0)$ are all possibilities. 

An $8$-hedrite is called {\em octahedrite} in \cite{DSt}; in fact, this paper is a follow-up of \cite{DSt}.

In a way, this paper continue the program of Kirkman (\cite{Kirk} p.~282) of classification of projections of alternating knot.


















\section{Central circuits partitions}

In this Section we consider a connected plane graph $G$ with all vertices of 
even degree, i.e. an Eulerian graph. Clearly, such graph has no cut-edges and
we can, without loss of generality, suppose the absence of 
cut-vertices, i.e. that $G$ is $2$-connected.
Call a circuit in $G$ {\it central} if it is obtained by starting with an
edge and continuing at each vertex by the edge opposite the entering one; such 
circuit is called also {\em traverse} 
(\cite{GK}), {\em straight ahead} (\cite{Ha}),  
{\em straight-ahead} (\cite{PTZ}), {\em straight Eulerian}
(Chapter 17 of \cite{God}), {\em cut-through}
(\cite{Je}),
{\em intersecting} etc. Clearly, the edge-set of 
$G$ is partitioned by all its central circuits.

%Such CC-partition can be considered (see, for example, \cite{Ha}) for any 
%drawing on the plane of any Eulerian (in general, not planar) graph, so that 
%edges are mapped into simple curves with at most one crossing point. The path
%called {\it straight ahead} in \cite{Ha} if, going through a vertex of the
%drawing, it leaves to its right and to its left the same number of edges.
%For example, the drawing of $K_2^4=C_4 \times C_4$ 
%($4$-hypercube=$4 \times 4$-torus) is CC-partitioned by two 
%$16$-circuits, which are, moreover, Hamiltonian circuits. 


Denote by 
$CC(G):=(...,a_i^{\alpha_i},...;...,b_j^{\beta_j},...)$ its {\it CC-vector}, 
where $...,a_i,...$ and  $...,b_j,...$ are increasing sequences of lengths of 
all its central circuits, simple ones and self-intersecting, respectively, 
and $\alpha_i, \beta_j$ are their respective multiplicities.
Clearly, $\sum_{i} a_i{\alpha_i}+ \sum_{j} b_j{\beta_j}=2n$, where 
$n$ is the number of vertices of $G$; denote by $r$ the number of 
central-circuits.


For a central circuit $C$, denote by $Int(C):=(c_0;...,c_k^{\gamma_k},...)$,
the {\em intersection vector of} $C$, where $c_0$ is
the number of self-intersections of the circuit $C$ and $(...,c_k,...)$ is
decreasing sequence of sizes of its intersection with other $r-1$ 
circuits, while the numbers $\gamma_k$ are respective multiplicities.


Two central circuits intersect in an even number number of vertices. 
The length of a central circuit is twice the number of its points of 
self-intersection plus the sum of its intersections with other circuits, 
so the length of a central circuit is even.



We will say that an $i$-hedrite is {\it pure} if any of its central circuits 
simple, i.e. has no self-intersections.
Easy to check that any pure $i$-hedrite has an even number $n$ of 
vertices. In fact, any vertex in this case belong to the intersection 
of exactly two central circuits.




\subsection{Intersection of central circuits}

The following Theorem is a local version (for ``parts'' of the sphere) of
the Euler formula
$2p_2+p_3=8+ \sum_{i\geq 5} (i-4)p_i$ for $p$-vector of any $4$-valent 
plane $3$-connected graph $P$.

Let $A$ be a patch of $P$ bounded by $t$ arcs (paths of edges) belonging to 
central circuits (different or coinciding). So, $A$ can be seen as a 
$t$-gon; we admit also $0$-gonal $A$, i.e.
just the interior of a simple central circuit. Suppose that the patch $A$ is
{\em regular}, i.e.
the continuation of any of bounding arc (on the central circuit to which it
belongs) lies outside of the patch; in other words, the patch
contains, around each its vertex, only one (not three) amongst four angles, 
obtained when two central circuits intersect in this vertex.  See below two 
examples of patch.

\begin{center}
\epsfxsize=60mm
\epsfbox{patch.eps}
\end{center}

Let $p'(A):=p'_3,...$ be the $p$-vector enumerating the faces of the patch $A$. 

\begin{theor} \label{Local-Euler-Formula}(\cite{DSt})
Let $A$ be a patch, described above, of a $4$-valent plane $3$-connected 
graph. Then we have $4-t=\sum_{i\geq 1} (4-i)p'_i$
\end{theor}



\begin{proposition}
(i) Any central circuit of a $4$-hedrite has no self-intersection vertices;

(ii) At least one central-circuit of a $7$-hedrite self-intersects.
\end{proposition}
\proof In fact, if a central circuit of a $4$-hedrite self-intersects, then we have a $1$-gonal regular patch. The equality of above Theorem becomes $2p'_2+p'_3=3$, an impossibility since $p_3=0$.

Take a central circuit containing an edge of the unique $2$-gon, the sequence (possibly empty) of adjacent $4$-gons will necesseraly finish by a triangle or this $2$-gon. both cases yield a self-intersection:

\begin{center}
\epsfxsize=60mm
\epsfbox{7hedrite-SelfIntersect.eps}
\end{center}





The Euler formula for the p-vector of a $4$-valent polyhedron:
$$8=\sum_{t\geq 1} (4-t)p_t$$
is a discrete analog of the Gauss-Bonnet formula $2\pi(1-g)=\int_{S} K(x)dx$ for the Gaussian curvature $K$ of a surface $S$ of genus $g$. So, the $t$-gons can be seen as respectively positively curved, flat, or negatively curved, if $t<4$, $t=4$, or $t>4$.


Given an $i$-hedrite $G$; it has $p_2=8-i$, $p_3=2i-8$. Let us 
call {\em graph of curvatures} of $G$, the graph (possibly, with loops 
and multiple edges) having as vertex-set all $2$-gons and $3$-gons of $G$. 
Two vertices (say, faces, $2$- or $3$-gonal faces $b$ and $c$ of $G$) of 
this $i$-vertex graph are connected by edge if there is a pseudo-road 
connecting them. A {\em pseudo-road} is sequence of $4$-gons 
(say, $a_1$, \dots, $a_l$), such that putting $a_0:=b$ and $a_{l+1}:=c$, 
we have that any $a_k$ with $1\leq k\leq l$ is adjacent to $a_{k-1}$ 
and $a_{k+1}$ on opposite edges. Clearly, in the graph of curvatures, 
the vertices corresponding to $2$- and $3$-gons have degree $2$ and $3$, 
respectively.







\begin{proposition}\label{intersec}
Let $C_1$, $C_2$ be any two central circuits of an $i$-hedrite. Then 
they are disjoint if and only if they are simple and there exist a 
ring of $4$-gons separating them.

\end{proposition}
\proof In fact, if both $C_1,C_2$ are simple circuits, Theorem is evident:
the curvature of the interior of a patch is $4$ and so, two
circuits are separated by $4$-gons only. Suppose that $C_1$ is 
self-intersecting. Then it has at least three regular patches and each 
of them has curvature at most $3$.
The circuit $C_2$, being disjoint with $C_1$, lies entirely inside one 
of those patches, say, $A$. So, all its triangles and digons, except, 
possibly, those from its exterior patch, lie in $A$. But the curvature 
of those elements is at least four, since the exterior
patch contains at most four triangles. It contradicts to the fact that $A$
has curvature at most three. So, $C_2$ intersects with $C_1$ and 
Theorem \ref{intersec} is proved.








\section{Adding and removing central circuits}


%It is the main operation, considered in this paper. First, we will introduce 
%associated notions; they are valid, in fact, for any $4$-valent plane graph. 
%
%
%Fix an $i$-hedrite $G$. 
%
%For a fixed central circuit $C$ consider its {\em shores}, i.e. two circuits
%of faces, lying on the left and on the right of $C$. Clearly, each shore is a 
%zone; call such zones {\em shore-zones}.
%
%
%Given a zone $Z=(F_1,...,F_m)$ of faces $F_i$, let us denote by $e_1,...,e_m$
%the edges of adjacency of pairs $(F_1,F_2),(F_2,F_3),...,(F_{m-1},F_m),(F_m,F_1)$, respectively. Call a {\em cutting of zone Z} the octahedrite obtained from
%the given one by adding $m$ new vertices $v_1,...,v_m$, one in the mid-point of
%each corresponding edge $e_i$ ($1 \le i \le m$) and $m$ new edges
%$(v_1,v_2),(v_2,v_3),...,(v_{m-1},v_m),(v_m,v_1)$. This new octahedrite 
%will have one new central circuit $V=(v_1,...,v_m)$.


The {\em deleting of a central circuit $C$} in an $i$-hedrite $G$ consists 
of removing all edges and vertices contained in $C$. It produces a 
$4$-valent polyhedron $P'$ having only $k$-gonal faces 
with $k \leq 4$ (cases $k=1,0$ are possible).


The {\em cutting} of an $i$-hedrite $G$ consists of adding another central 
circuit to it. The faces of the new $i'$-hedrite $G'$ with $8\geq i'\geq i$ 
comes from the cutting of faces of $G$. This operation is only partially
defined since arbitrary cutting can produce $k$-gons with $k>4$. The 
cutting of a $4$-gon in several $4$-gons (two if the face
is traversed only once) is possible only if the $4-gon$ is traversed 
on opposite edges; this correspond to the notion of {\it shore-zone} 
in \cite{DSt}.


A cutting changes CC-partition of an $i$-hedrite only in the following 
way: new central circuit $C$ is added and all others central circuits 
remain unchanged except that the length of each of them increases by one 
for any intersection with $C$. 


Call {\em rail-road} a circuit of $4$-gons (possibly self intersecting), 
in which every $4$-gon is adjacent to two of its neighbors on opposite
edges. A rail-road is bounded by two ``parallel'' central circuits.
The deleting of  one of those central circuits (in other word, collapse 
rail-road into one central circuit) is called {\em reducing}.
The cutting produces a rail-road if and only if it is {\em inflation 
along a central circuit $C$}, i.e. remplacing of it by thin enough 
rail-road. {\em $t$-inflation along a central circuit $C$} is remplacing
this central circuit by $t-1$ parallel (thin enough) rail-roads.
{\em $t$-inflation of an $i$-hedrite} is new $i$-hedrite
obtained from original one by simultaneous $t$-inflation along all
of its central circuits. 



Let $G'$ be the result of cutting of an $i$-hedrite $G$. Then $G'$ is an 
$i'$-hedrite with $i\leq i'\leq 8$; but $i'=i$ if the cutting is an 
inflation along a central circuit.


An $i$-hedrite is called {\em irreducible} if it contains no 
rail-road. It is called {\em maximal irreducible} if it cannot be
obtained!! from another $i'$-hedrite!! by a cutting.

$t$-inflation of $G$ is $G$ if $t=1$, and it is just inflation of $G$ 
if $t=2$.
% One can call $Med(G)$ {\em $\frac{3}{2}$-inflation} of $G$, 
% since $Med(Med(G))$ is $2$-inflation of $G$.

\begin{remark}

Let $C$ be a central circuit of $G$ with $CC(G)=(...,a_i^{\alpha_i},...;...,b_j^{\beta_j},...)$,  and let 
$Int(C)=(c_0;c_1^{\gamma_1},...,c_r^{\gamma_r})$. Let $C'$ be one of 
$t$ parallel copies, which are put instead of the circuit $C$ in $G^t$.
Then $CC(G^t)=(...,ta_i^{t\alpha_i},...;...,tb_j^{t\beta_j},...)$,
$Int(C')=(c_0;c_1^{t\gamma_1},...,c_r^{t\gamma_r}, (2c_0)^{t-1})$.
\end{remark}


\section{Some special $i$-hedrites}
For any integer $n$ denote:
!!
by $I_{6,n}$  the $(2n+2)$-vertex
$6$-hedrite, such that each $2$-gon is adjacent to two $3$-gons; 
!!
by $I_{5,n}$ the $(2n+3)$-vertex $5$-hedrite, such that 
two $2$-gons share a vertex and remaining $2$-gon is adjacent
to two triangles;
!!
by $I_{4,n}$ the $(2n+4)$-vertex $4$-hedrite, such that 
four $2$-gons are organised into two pairs sharing a vertex. 
!!perhaps, we can say here that they are exactly the case (ii2),(ii3) of
Th.7 below (classification of 4-hedrites) and exactly the case of this
symmetry D2h, D2d; they correspond also to the shift i=1 in terms of (iii)
of Th.7 below!!
See on Figure \ref{fig:FamilyIin} the first occurrence of those graphs and on Table \ref{FundamentalInfo} the Groups and CC-vector of those graphs

\begin{figure}
\centering
\epsfxsize=100mm
\epsfbox{FamilyIin.eps}
\label{fig:FamilyIin}
\end{figure}

\begin{table}
\begin{center}
\begin{tabular}{||c|c|c|c||}
\hline
$i$-hedrite!!               &$nbr. vertices$!!       &Grp
&CC-vector\\\hline
$I_{6,n}$, $n$ odd  &$2(n+1)$    &$D_{2d}$   &$4n+4$\\
$I_{6,n}$, $n$ even &$2(n+1)$    &$D_{2h}$   &$(2n+2)^2$\\
$I_{5,n}$           &$2(n+1)+1$  &$C_{2v}$   &$4n+6$\\
$I_{4,n}$, $n$ odd  &$2(n+1)+2$  &$D_{2d}$   &$(2n+4)^2$!!\\
$I_{4,n}$, $n$ even &$2(n+1)+2$  &$D_{2h}$   &$(2n+4)^2$\\\hline
\end{tabular}
\end{center}
\caption{$2$-connected $i$-hedites}
\label{FundamentalInfo}
\end{table}



\begin{lemma}
Any $i$-hedrite is $2$-connected.
\end{lemma}
\proof Let $G$ be an $i$-hedrite and assume that there is one vertex $v$ such that $G-\{v\}$ is disconnected in two components $C_1$, $C_2$. Then two edges from $v$ will connect to a vertex $w$ of $C_1$ and two edges from $v$ will connect to a vertex $w'$ of $C_2$. See below the corresponding drawing.!!

\begin{center}
\epsfxsize=60mm
\epsfbox{2-connected.eps}
\end{center}

But the vertex $w$ will disconnect the graph and so, iterating the construction, we obtain an infinite sequence $v_1$, \dots, $v_n$ that disconnects!! $G$. This contradicts to initial assumption and proves that $G$ is $2$-connected.



\begin{theor}\label{3-connectedness}
Any $i$-hedrite that is not $3$-connected is one of the following possibilities:
\begin{enumerate}
\item a $4$-hedrite obtained by $t$ inflation of only!! one central circuit of the $4$-hedrite $2-1$!!.
!!perhaps, better just define them as 4-hedrites $J(4,n) and also say
that they are projections of {\em composite} alternating links
2_1#2#2_1...#2_1$!!
\item an $i$-hedrite $I_{i,n}$ for $n \ge 1$ and  $i=4,5,6$!!.
\end{enumerate}

\end{theor}

\proof Let $G$ be an $i$-hedrite and assume that it is not $3$-connected. 
Then there are two vertices, say, $v$ and $v'$, such that 
$G-\{v, v'\}$ is disconnected in two components, say, $C_1$ and $C_2$.
Amongst!! $4$ edges from $v$, $\{e_1,\dots, e_{t}\}$ goes to $C_1$ and
amongst!! $4$ edges from $v'$, $\{e'_1,\dots, e'_{t'}\}$ goes to $C_1$.
Two!! numbers $t$ and $t'$ can takes values $1$, $2$ or $3$; we
will consider all possible cases.

Assume that $t=1$ and $t'=1$, then the edges $e$ and $e'$ must be distinct, since otherwise $C_1$ is emptygraph. Moreover, $e$ and $e'$ have no common vertex since, otherwise, $G$ would not be $2$-connected. So, $v$ and!! $v'$ are connected by $e$ and!! $e'$ 
to a vertex $w$ !! $w'$,!! respectively. Since face size is at most $4$, the vertices $v$ and $v'$ 
(!!respectively,!! $w$ and $w'$)!! are linked by two edges (see Figure \ref{fig:Case33}).

\begin{figure}
\centering
\epsfxsize=40mm
\epsfbox{Case3-3.eps}
\caption{The case $t=1$, $t'=1$}
\label{fig:Case33}
\end{figure}

Two!! points $w$ and $w'$ can either be connected by two edges and
we are done, or disconnect the graph. In the latter case, we can iterate
the construction. Since the graph is finite, the construction eventually
finish and we get a $t$-inflation of a $4$-hedrite $2_1$. If $t=1$ and
$t'=3$, then by a similar reasoning, one gets again a $t$-inflation of
$4$-hedrite $2_1$.

Assume that $t=2$ and $t'=2$. One has $\{e_1, e_2\}\cap \{e'_1, e'_2\}=\emptyset$, since otherwise one can attribute an edge to $C_2$ and get the case $t=1$, $t'=1$. So, one has, say, $e_1\cap e'_1=\{w_1\}$ and $e_1\cap e'_1=\{w_2\}$ and the following two possibilities (see Figure \ref{fig:AllThreeCases}): either $w_1=w_2$ (this corresponds to $\{e_1, e_2\}$ and $\{e'_1, e'_2\}$ being!! the edges of two digons) and we are done,!! or $w_1\not= w_2$.
\begin{figure}
\centering
\epsfxsize=40mm
\epsfbox{Case2-2.eps}
\caption{The Case $t=2$, $t'=2$}
\label{fig:AllThreeCases}
\end{figure}

Assume now that $w_1\not= w_2$; two!! points $w_1$ and $w_2$ can 
either be connected by two edges and we are done, or disconnect the
graph. In the latter case, we can iterate the construction. Since the 
graph is finite, the construction eventually finish. If we do the same
construction on the other side, then we get a similar structure and the 
graph is of the form $I_{4,n_0}$, $I_{5, n_0}$!! 
or $I_{6,n_0}$!! with $n_0\geq 1$.


Assume now that $t=2$ and $t'=1$. The edges $e_1$, $e_2$ and $e'_1$ are!! all distinct,!! since, otherwise, the vertex!! $v$ 
disconnects!! the graph.
So, $v'$ is connected by $e'_1$ to a vertex $w'$. Now either $v'$ or $w'$
is connected to $v$ since, otherwise, we would have a $5$-gonal face.
If $w'$ is connected to $v$ then the pair $(w', v)$ disconnects!! the graph.
This construction is infinite (see Figure \ref{InfiniteStructure.eps}), so
we get a contradiction.!!

\begin{figure}
\centering
\epsfxsize=40mm
\epsfbox{InfiniteStructure.eps}
\caption{The case $t=2$, $t'=1$}
\label{fig:Infinite structure}
\end{figure}


\begin{theor}
(i) If $G$ is an $i$-hedrite with two adjacent $2$-gons, then this is a $t$-inflation along only!! one central circuit of $4$-hedrite $2-1$!!.

(ii) If $G$ is an $i$-hedrite with two $2$-gon sharing a vertex, then it is an $I_{4,n}$ or an $I_{5,n}$.

\end{theor}

\proof The proof for (i) is very similar to the case $t=1$, $t'=1$ of Theorem \ref{3-connectedness}, while the proof of (ii) is similar to the proof used for $t=2$, $t'=2$.





\begin{theor}

(i) $4$-hedrites exist if and only if $n\geq 2$, even.

(ii) $5$-hedrites exist if and only if $n\geq 3$, $n\not= 4$.

(iii) $6$-hedrites exist if and only if $n\geq 4$.

(iv) $7$-hedrites exist if and only if $n\geq 7$.

(v) $8$-hedrites exist if and only if $n\geq 6$, $n\not= 7$.

\end{theor}
\proof The case (v) is proven in \cite{Gr}, page 282.

Our computation presented all $i$-hedrites with at most $14$ vertices. Easy to see NEED TO COMPLETE
Suppose that an $i$-hedrite has a $t$-gonal face $F$ adjacent to $t$ triangles. Then, one can add to every edge $e_i$ of $F$ a vertex $v_i$ and form another $t$-gon $v_1\dots v_t$. The obtained graph is an $i$-hedrite with $n+t$ vertices.




Call an Eulerian graph $G$ {\it balanced}, if all its central circuits of
same length have the same intersection vector.
Any $8$-hedrite with $n \le 21$ is balanced, but there is unbalanced $22$-vertex $8$-hedrite. (DELETE OR FIND UNBALANCED $i$-HEDRITE)







For a plane graph $G$, denote by $G^*$ its planar dual and {\em $Med(G)$} 
denotes its {\em medial} graph, i.e. the vertices of $Med(G)$ are the edges of 
$G$, two of them being adjacent if the corresponding edges share a vertex and 
belong to the same face of the embedding of $G$ in the plane. 
So, $Med(G)$=$Med(G^*)$.

Clearly, $Med(G)$ is a $4$-valent plane graph and, for any $i$-hedrite $G$, $Med(G)$, $Med(G)$ is an $i$-hedrite with twice the number of vertices of $G$ and all $2$-, $3$-gonal faces being isolated. The operation of taking the medial is a particular case of a general construction called Goldberg-Coxeter construction for the parameter $z=1+i$.



\section{Irreducible $i$-hedrites}



\begin{theor}\label{irre}
Any irreducible $i$-hedrite has at most $i-2$ central circuits and equality is attained for each $i$, $4\leq i\leq 8$.
\end{theor}
\proof For $i=8$, the theorem is proved in \cite{DSt}. We will show using suitable cutting that this result implies the Theorem for $i<8$.

Let us start with the simplest case of a $7$-hedrite. Consider a simple circuit $S$ in its curvature graph, which contains the vertex corresponding to unique $2$-gon. Remind that the vertices in the curvature graph correspond to $2$- or $3$-gons, while edges correspond to pseudo-roads. So, the simple circuit $S$ corresponds to the circuit of faces of $G$, containing our $2$-gon, some $4$-gons and, possibly, some of six $3$-gons; see the picture below.


\begin{center}
\epsfxsize=40mm
\epsfbox{CC_FowlerGraph.eps}
\end{center}






Suppose that the $7$-hedrite $G$ is irreducible and has $k$ central circuits. By adding the central circuit $C$, which is shown by dotted lines on picture above, we produce an $8$-hedrite (since, the $2$-gon is cut by $C$ in two triangles), which is still irreducible and has $k+1$ central circuits. So, $k+1\leq 6$, by Theorem 3 of \cite{DSt}.


For remaining cases of $i$-hedrites with $i=4,5,6$, the proof is similar. In each case, we consider all possible distribution of $2$-gons by simple circuits in the graph of curvatures and for each such circuit, to add suitable number of new central circuits. 


All possibilities are presented on picture below: two for $i=6$, three for $i=5$ and two for $i=4$. In the last case, there are no triangles and so, simple circuits in the curvature graph contain only even number of $2$-gons by local Euler formula (\ref{Local-Euler-Formula}). Note that the case of $4$-hedrite is obvious by Theorem 5 of \cite{DSt}.


\begin{center}
\epsfxsize=60mm
\epsfbox{PossibilitiesTwoToFour.eps}
\end{center}



\begin{lemma}
Let $G$ be an irreducible $8$-hedrite and $C$ a central circuit, assume that $C$ is adjacent to three triangles on one side. Then one can add another central circuit to $G$ while staying irreducible.
\end{lemma}
\proof From every one of the three triangles, say $T_1$, $T_2$, $T_3$, one can define a pseudo-road from the side of the central triangle that belongs to the central circuit. See on picture below
\begin{center}
\epsfxsize=40mm
\epsfbox{ThrreTrianglesCentral.eps}
\end{center}
Each such pseudo-road define an edge, say $e_1$, $e_2$, $e_3$ in the graph 
of curvatures. We have two possibilities: either a path exist from $T_i$ 
to $T_j$ that use the edges $e_i$, $e_j$ or such a path does not exist. In 
first case one can add a central circuit which cut all triangles $T_1$, 
$T_2$, $T_3$ in the form depicted below:
\begin{center}
\epsfxsize=40mm
\epsfbox{CuttingAllTriangles.eps}
\end{center}

Assume now that there is no path between $T_i$ and $T_j$. Then, since the graph of curvatures is is $3$-valent, the graph NEED TO COMPLETE













See below example of an irreducible $7$-hedrite with $5$ (the maximum number) central circuits

\begin{center}
\epsfxsize=60mm
\epsfbox{7hedrite32Vertex5CC.eps}
\end{center}

For all other $i$, there is an example of pure irreducible $i$-hedrite with $i-2$ central circuits (see Theorem \ref{TheOneWithSimpleCentralCircuit})















\begin{conjecture}
An irreducible $i$-hedrite is maximal irreducible if and only if it has $i-2$ central circuits.
\end{conjecture}
\proof NEED TO FIND.



\section{Symmetry group of $i$-hedrites}
We consider below the maximal symmetry groups of plane graphs; these groups are identified with the corresponding space groups.



%SHTOGRIN METHOD, PLEASE KEEP
%--------------------------------
%Consider a putative $5$-hedrite $G$ with symmetry $C_{3h}$. Then its three $2$-gons form orbit of size three. The only possibility for each of those $2$-gons is to intersect the plane of symmetry. But there are two ways of such intersection: either throught vertices of $2$-gons, or throught its edges. So, there is a belt of $4$-gons on which three digons are placed on equal distance. In the first case, the $4$-gons and $2$-gons of the belt are incident to its neighbor by opposite vertices; in the second case, they are adjacent by opposite edges.
%
%INSERT DRAWING, PLEASE
%
%One can see that the only way to complete those belts while preserving the $C_3$-symmetry, produces an inflation of $5$-hedrite Nr.~3-1 (in first case), or, in the second case, of the $5$-hedrite Nr.~6-2 (the medial of Nr.~3-1).
%
%Also (NEED TO COMPLETE), the symmetry $C_3$ implies the symmetry $D_3$, so the symmetry $C_{3v}$ implies the symmetry $D_{3h}$.
%---------------------------




\begin{theor}
\begin{itemize}
\item[(i)] The only symmetry groups of $4$-hedrites are point subgroups of $D_{4h}$, which contain $D_{2}$ as point subgroup, i.e. $D_{4h}$, $D_4$, $D_{2h}$, $D_{2d}$, $D_2$.

\item[(ii)] The only symmetry groups of $5$-hedrites are $C_1$, $C_2$, $C_s$, $C_{2v}$, $D_3$ and $D_{3h}$.

\item[(iii)] The only symmetry groups of $6$-hedrites are $D_{2d}$, $D_{2h}$ and all their point subgroups, i.e. $D_{2}$, $C_{2h}$, $C_{2v}$, $C_i$, $C_{2}$, $C_{s}$, $C_{1}$.

\item[(iv)] The only symmetry groups of $7$-hedrites are point subgroups of $C_{2v}$, i.e. $C_{2v}$, $C_{2}$, $C_{s}$, $C_{1}$.

\item[(v)] The only symmetry groups of $8$-hedrite are $C_{1}$, $C_s$, $C_2$, $C_{2v}$, $C_i$, $C_{2h}$, $S_4$, $D_2$, $D_{2d}$, $D_{2h}$, $D_3$, $D_{3d}$, $D_{3h}$, $D_4$, $D_{4d}$, $D_{4h}$, $O$, $O_h$.

\end{itemize}


\end{theor}
\proof For $4$-hedrite, see \cite{DSt}. Any transformation stabilizing a digon can interchange its two edges and two vertices. So, the stabilizing point subgroup of a $2$-gon can be $C_{2v}$, $C_s$, $C_2$ or $C_1$.

A $7$-hedrite has only one $2$-gon which has to be preserved by the symmetry group; so, all possibilities are $C_{2v}$, $C_s$, $C_2$, $C_1$.

Every symmetry of an $i$-hedrite induces a symmetry on its $2$-gons and triangles. Since the stabilizer of a digon, triangle has maximal size $4$, $6$, this imply that the order of the symmetry group of an $i$-hedrite is bounded above by $4|Sym(8-i)|=4(8-i)!$ and $6|Sym(2i-8)|=6(2i-8)!$.

So, in particular, the order of symmetry group of an $6$-hedrite is at most $8$. If $f$ is an element of order three then it fix each of two digons. Since, the stabilizer of a digon does not contain an element of order three, we have that no such $f$ exists. If $f$ is a rotation of order $4$ then $f^2$ is a rotation of order $2$ stabilizing each $2$-gon; so, the axis of $f$ goes throught the two $2$-gons. This is a contradiction. By the search in the Tables of the groups one can see that the only possibilities are $C_1$, $C_s$, $C_2$, $C_i$, $C_{2v}$, $C_{2h}$, $D_2$, $D_{2h}$, $D_{2d}$. In fact, we found a $6$-hedrite for any such symmetries in Figure XXXX and Subsection YYYYY

For $5$-hedrite, since there is two triangles, the maximal order of the group is $12$. The oddness of the number of $2$-gons excludes!! central symmetry, axis of order $4$, and groups $D_2$, $D_{2h}$, $D_{2d}$.

If $G$ is a $5$-hedrite with a $3$-fold axis then this axis goes throught the two triangle, say $T_1$, $T_2$. If one consider a belt of $4$-gons around $T_1$, then after a number $p$ of steps, one will encounter a $2$-gon and so, by symmetry three $2$-gons. So, we will have the following possibilities:

\begin{center}
\epsfxsize=60mm
\epsfbox{PossibleNeighbor.eps}
\end{center}

There is only one way to extend this graph to a $5$-hedrite and the obtained extension has symmetry at least $D_3$. This eliminates the cases $C_3$, $C_{3v}$, $C_{3h}$.

An $8$-hedrite can a priori have $k$-fold axis of rotation with $k=2, 3, 4$. If $k=3$, then the axis of the rotation goes through two triangles, say $T_1$, $T_8$. If one consider around triangle $T_1$ a belt of $4$-gons, then after a number $p$ of steps, one will encounter a triangle and so, by symmetry three triangles, say $\{T_2, T_3, T_4\}$. Adding, if necessary, belts of $4$-gons, one will encounter the last three triangles, say $\{T_5, T_6, T_7\}$. There is an unique way to complete the graph while preserving the $3$-fold symmetry. The symmetry of the obtained graph can be $D_3$, $D_{3d}$, $D_{3h}$. 

If $k=4$, then the axis of the rotation goes through a vertex or a $4$-gon. Assume for simplicity, that this axis goes through a vertex, then repeating above reasoning, one obtains two orbits of triangles, say $\{T_1, T_2, T_3, T_4\}$ and $\{T_5, T_6, T_7, T_8\}$ under $4$-fold symmetry and the symmetry group is $D_4$, $D_{4d}$ or $D_{4h}$.

So, one obtains the above list of $18$ possible point groups. All this group appear in subsection XXXXXX or in Figure DRAWING TO BE DONE.


\begin{center}
\epsfxsize=60mm
\epsfbox{6-hedriteCi.eps}
\end{center}

\begin{center}
\epsfxsize=60mm
\epsfbox{8-hedriteCi.eps}
\end{center}



See below an $8$-hedrite with symmetry $C_{2h}$ (THERE IS A SMALLER ONE)

REORGANIZE; $i$-hedrites with $26$ and $34$ vertices


\begin{center}
\epsfxsize=60mm
\epsfbox{OcahedriteC2h.eps}
\end{center}

\begin{center}
\epsfxsize=60mm
\epsfbox{OctahedriteS4.eps}
\end{center}



\begin{center}
\epsfxsize=60mm
\epsfbox{OcahedriteC2hOther.eps}
\end{center}

The smallest $i$-hedrite and the smallest $8$-hedrite with symmetry $D_3$ are given in Figure \ref{fig:WithSymmetryD3} (both are knots).

\begin{figure}
\centering
\mbox{\subfigure[$15$-vertices $5$-hedrite]{\epsfig{figure=5-hedrite15-3sec.eps,width=.20\textwidth}}\quad
\subfigure[$18$-vertices $8$-hedrite]{\epsfig{figure=ZZprojection7:3.ps,width=.20\textwidth}}}\caption{Two smallest $i$-hedrites with symmetry $D_3$}
\label{fig:WithSymmetryD3}
\end{figure}








\section{Classification of pure irreducible $i$-hedrites}

NEED TO SHRINK THIS ONE
\begin{theor}

Let $R$ be any $4$-hedrite $R_{4,2m}$.

(i) $R$ is irreducible if and only if it has exactly two central
circuits; there exist pair of integers
$t_1,t_2 \ge 0$ and
unique irreducible $4$-hedrite $R_0$ with $2m_0$ vertices, such that $R$ can be obtained from
$R_0$ by simultaneous $t_1$- and $t_2$-inflation, respectively, along two
central circuits of $R_0$ (so, $m=m_0(t_1+1)(t_2+1)$).

(ii) The group of symmetry of $R$ is one of five groups (see Figure 6.2)
$D_{4h}$, $D_{2d}$, $D_{2h}$, $D_4$, $D_2$; it is $D_2$ unless it is:     

(ii1) $D_{4h}$, if $t_1=t_2$ and $m_0=1,2$;

(ii2) $D_{2d}$, if $t_1 \neq t_2$, $m_0=2$ or if $t_1=t_2$, $m_0$ is odd
and $R_0$ has two isolated pairs of vertex-intersecting $2$-gons;

(ii3) $D_{2h}$, if $t_1 \neq t_2$, $m_0=1$ or if $t_1=t_2$, $m_0$ is even
and $R_0$ has two isolated pairs of vertex-intersecting $2$-gons;

(ii4) $D_4$, if $m=a^2+b^2$, for some integers $a>b>0$, and $R$ is organized as illustrated on Figure 6.3; in terms of $R_0$, 
$t_1=t_2=gcd(a,b)-1$, $R_0$ is of symmetry $D_4$ and 
$m_0=(\frac{a}{gcd(a,b)})^2+(\frac{b}{gcd(a,b)})^2$. Above case (ii1) can also
be seen as the sub-case $b=0,a$ of the case (ii4).

(iii) $R$ is, up to a $t$-inflation along a central 
circuit, defined (see example for $2m=14$ vertices on Figure 5) by the shift by $i, i=0,1,2,..., [\frac{m}{2}]$ vertices between the pair
of boundary $2$-gons of the horizontal circuit and the remaining pair of
$2$-gons. Moreover:

(iii1) $R$ with shift $i=1$ is irreducible and has, for $m \ge 3$,
two isolated pairs of vertex-intersecting $2$-gons (see examples of this serie
on Figure 6.1);

(iii2) All $R$ with shift $i, 1<i< [\frac{m}{2}]$ have four isolated $2$-gons;
amongst them are possible isomorphisms (for example, between 3rd and 4th on
Figure 5) and not irreducible $R$ (for example,
for any shift $i, i \ge 2$, and $m \equiv0\pmod{2i}$ there exists  
$R$
having $i+1$ central circuits (all, but the horizontal one, are concentric).
  
(iii3) The symmetry of irreducible $R$ is $D_2$ if and only if interchange of 
central circuits changes the value of shift.

\end{theor}






%\begin{center}
%\mbox{}\hspace{-3.5cm}
%\input NUMB1-OK.PIC
%\end{center}
%
%\begin{center}
%	Figure 7: Projections of all alternating links, 
%to which reduce the octahedrites of Figure 8 and which are not
%octahedrites 
%\end{center}







\begin{theor}\label{TheOneWithSimpleCentralCircuit}

Any pure irreducible $i$-hedrite is, either any $4$-hedrite with two central circuits, or a $5$-hedrite 6-2, or one of $6$-hedrites 
$8-6$!!, $14-{20}$!!, or one of the following eight $8$-hedrites: 
$6-{1}$!!, $12-{4}$!!, $12-{5}$!!, $14-{7}$!!, and (see Figure ????) 20-1, 22-1, 30-1, 32-1.

or one of the four $8$-hedrite with $20$, $22$, $30$, $32$ vertices given on the Figure XXXXXXXX.!!SVP, DELETE THOSE 2 LINES

\end{theor}


\proof Let $G$ be a pure irreducible $i$-hedrite having $r$ central circuits. 
If one deletes a central circuit, then, in general, $1$-gon can appear. It 
does not happen for $G$, since it will imply a self-intersection of a central 
circuit. Clearly, the result of deletion of a central circuit from $G$ 
produces an pure irreducible $i$-hedrite with $r-1$ central circuits.

First, if $r=2$ then the Theorem XXX from \cite{DSt} gives that such $G$ are 
exactly $4$-hedrites with two central circuits; all of them are classified 
in Proposition XXX above.

We prove the Theorem by systematical analysis of all possible ways to add 
to $G$ (for $r=2,3,4,5$) a central circuit, in order to get a pure 
irreducible $i$-hedrite with $r+1$ central circuits. 


(i) Let $r=2$. Then $G$ can be only one of two smallest $4$-hedrites. In 
fact, if $G$ is another $4$-hedrite then because of classification Theorem 
XXX it has a form as in the picture below.

(INSERT DRAWING)

New central circuit should cut both $2$-gons on opposite edges, since otherwise there is a rail-road. But Figure below shows, on example for $n=6$, that a self-intersection appears if the two central circuits interesect in more than four vertices of intersection.


\begin{center}
\epsfxsize=40mm
\epsfbox{SelfIntersectionForSix.eps}
\end{center}



So, the only possible $4$-hedrites with two central circuits!!, which can be cut in order to produce irreducible pure $i$-hedrite are 
$4$-hedrites !!$2-1$ and $4-1$, corresponding to links!! $2^2_1$ and $4^2_1$. 
All cases are indicated below. 

\begin{center}
\epsfxsize=120mm
\epsfbox{PurityThreeCentral.eps}
\end{center}
!!in this Figure put ``6-hedrite 8-5'' instead of 6-hedrite 8-1!!

So, all irreducible pure $i$-hedrites with three central circuits are $5$-hedrite 6-2, $6$-hedrite 8-5, $8$-hedrite 6-1!!!
(i.e. the!! projections of links $6^3_1$, $8^3_6$, $6^3_2$)!!
!! I ALSO CHANGED ORDER OF TWO LAST LINKS!!.
 Now we apply the same procedure to those three $i$-hedrites; see Figure below:

\begin{center}
\epsfxsize=120mm
\epsfbox{PurityFourCentral.eps}
\end{center}
!!in this Figure write under first pictures of each of 3 rows,
respectively:
5-hedrite 6-2
8-hedrite 6-1
6-hedrite 6-5!!

So, all irreducible pure $i$-hedrites with four central circuits are $8$-hedrites 12-4, 12-5, 14-7 and $6$-hedrite 14-20.!!

By the same method, one can see that there are exactly two pure irreducible $i$-hedrites with five central circuits and two with six central circuits; all four are octahedrites given above.!!

\begin{figure}
{\small
\setlength{\unitlength}{1cm}
\begin{minipage}[t]{3.5cm}
\begin{picture}(3.5,3.5)
\leavevmode
\epsfxsize=3.5cm
\epsffile{4reg_21_114.ps}
\end{picture}\par
\begin{center}
{{\bf Nr.20-1} \quad $(8^5)$ \\ $D_{2d}$ \\}
\end{center}
\end{minipage}
% \setlength{\unitlength}{1cm}
\begin{minipage}[t]{3.5cm}
\begin{picture}(3.5,3.5)
\leavevmode
\epsfxsize=2.5cm
\epsffile{PureOctahedrite22.eps}
\end{picture}
\par
\begin{center}
{{\bf Nr.22-1} \quad $(8^3,10^2)$ \\ $D_{2h}$ \\}
\end{center}
\end{minipage}
\setlength{\unitlength}{1cm}
\begin{minipage}[t]{3.5cm}
\hfil\begin{picture}(2.3,2.3)
\leavevmode
\epsfxsize=2.3cm
\epsffile{oc30-1.ps}
\end{picture}\hfil\par
\begin{center}
{{\bf Nr.30-1} \quad $(10^6)$ \\ $O$ \\}
\end{center}
\end{minipage}
\setlength{\unitlength}{1cm}
\begin{minipage}[t]{3.5cm}
\begin{picture}(3.5,3.5)
\leavevmode
% \epsfxsize=3.5cm
% \epsffile{fig-32.pic}
\mbox{} \hspace{-1cm}
\input fig-32.pic
\end{picture}\par
\begin{center}
{{\bf Nr.32-1} \quad $(10^4,12^2)$ \\ $D_{4h}$ \\}
\end{center}
\end{minipage}
}
\caption{Pure irreducible $i$-hedrites!! with $5$ or!! $6$ central circuits}
\end{figure}












\section{Small $i$-hedrites}

Here and below all links are given 
in Rolfsen's notation (see the table in \cite{Rolf} and also,  
for example, \cite{Kaw}) for links with at most 9 
crossings and knots with 10 crossings, or, otherwise, in
Dowker-Thistlewhaite's numbering (see \cite{T}), if any.
We write $\sim$ if the projection in our Table is different 
from the one given in corresponding cases of one of the above 

We give below all $i$-hedrites with at most $12$ vertices indicating under 
picture of each its symmetry, CC-vector and corresponding alternating link.
All $i$-hedrites with $13$ and $14$ vertices are listed in Table \ref{tab:i-hedrite13_14}.!!


On the Figure below, in order to express better the (maximal)
symmetry of an $i$-hedrite, we put:

(i) a double arrow, in order to represent an edge passing at infinity,

(ii) a quadruple arrow, in order to represent a vertex at infinity.





\subsection{$8$-hedrites}
{\small
\setlength{\unitlength}{1cm}
\begin{minipage}[t]{3.5cm}
\begin{picture}(3.5,3.5)
\leavevmode
\epsfxsize=2.5cm
\epsffile{8-hedrite6-1.eps}
\end{picture}\par
\begin{center}
{{\bf Nr.6-1} \quad $O_h$\\ $6^3_2$ \quad $(4^3)$\\ }
\end{center}
\end{minipage}
\setlength{\unitlength}{1cm}
\begin{minipage}[t]{3.5cm}
\begin{picture}(3.5,3.5)
\leavevmode
\epsfxsize=2.5cm
\epsffile{8-hedrite8-1.eps}
\end{picture}\par
\begin{center}
{{\bf Nr.8-1} \quad $D_{4d}$\\ $8_{18}$ \quad $(16)$\\ }
\end{center}
\end{minipage}
\setlength{\unitlength}{1cm}
\begin{minipage}[t]{3.5cm}
\begin{picture}(3.5,3.5)
\leavevmode
\epsfxsize=2.5cm
\epsffile{8-hedrite9_1.eps}
\end{picture}\par
\begin{center}
{{\bf Nr.9-1} \quad $D_{3h}$\\ $9_{40}$ \quad $(18)$\\ }
\end{center}
\end{minipage}
\setlength{\unitlength}{1cm}
\begin{minipage}[t]{3.5cm}
\begin{picture}(3.5,3.5)
\leavevmode
\epsfxsize=2.5cm
\epsffile{8-hedrite10_1sec.eps}
\end{picture}\par
\begin{center}
{{\bf Nr.10-1} \quad $D_{2}$\\ $10^2_{56}$ \quad $(6;14)$\\ }
\end{center}
\end{minipage}
\setlength{\unitlength}{1cm}
\begin{minipage}[t]{3.5cm}
\begin{picture}(3.5,3.5)
\leavevmode
\epsfxsize=2.5cm
\epsffile{8-hedrite10_2sec.eps}
\end{picture}\par
\begin{center}
{{\bf Nr.10-2} \quad $D_{4h}$\\ $10^4_{169}$ \quad $(4^2,6^2)$ red.\\ }
\end{center}
\end{minipage}
\setlength{\unitlength}{1cm}
\begin{minipage}[t]{3.5cm}
\begin{picture}(3.5,3.5)
\leavevmode
\epsfxsize=2.5cm
\epsffile{8-hedrite11_1.eps}
\end{picture}\par
\begin{center}
{{\bf Nr.11-1} \quad $C_{2v}$\\ $11^3_{520}$ \quad $(6^2;10)$\\ }
\end{center}
\end{minipage}
\setlength{\unitlength}{1cm}
\begin{minipage}[t]{3.5cm}
\begin{picture}(3.5,3.5)
\leavevmode
\epsfxsize=2.2cm
\epsffile{8-hedrite12-5cage.ps}
\end{picture}\par
\begin{center}
{{\bf Nr.12-1} \quad $D_{3d}$\\ $12_{1019}$ \quad $(24)$\\ }
\end{center}
\end{minipage}
\setlength{\unitlength}{1cm}
\begin{minipage}[t]{3.5cm}
\begin{picture}(3.5,3.5)
\leavevmode
\epsfxsize=2.5cm
\epsffile{8-hedrite12_2.eps}
\end{picture}\par
\begin{center}
{{\bf Nr.12-2} \quad $D_2$\\ $12_{868}$ \quad $(24)$\\ }
\end{center}
\end{minipage}
\setlength{\unitlength}{1cm}
\begin{minipage}[t]{3.5cm}
\begin{picture}(3.5,3.5)
\leavevmode
\epsfxsize=2.5cm
\epsffile{8-hedrite12_3sec.eps}
\end{picture}\par
\begin{center}
{{\bf Nr.12-3} \quad $C_2$\\ $?????$ \quad $(6;18)$\\ }
!!correct this picture: move to verical edges by 45 degrees!!
\end{center}
\end{minipage}
\setlength{\unitlength}{1cm}
\begin{minipage}[t]{3.5cm}
\begin{picture}(3.5,3.5)
\leavevmode
\epsfxsize=2.5cm
\epsffile{8-hedrite12_4.eps}
\end{picture}\par
\begin{center}
{{\bf Nr.12-4} \quad $O_h$\\ $?????$ \quad $(6^4)$\\ }
\end{center}
\end{minipage}
\setlength{\unitlength}{1cm}
\begin{minipage}[t]{3.5cm}
\begin{picture}(3.5,3.5)
\leavevmode
\epsfxsize=2.2cm
\epsffile{8-hedrite12-1cage.ps}
\end{picture}\par
\begin{center}
{{\bf Nr.12-5} \quad $D_{3h}$\\ $????$ \quad $(6^4)$\\ }
\end{center}
\end{minipage}
}






\subsection{$7$-hedrites}
{\small
\setlength{\unitlength}{1cm}
\begin{minipage}[t]{3.5cm}
\begin{picture}(3.5,3.5)
\leavevmode
\epsfxsize=2.5cm
\epsffile{7-hedrite7_1.eps}
\end{picture}\par
\begin{center}
{{\bf Nr.7-1} \quad $C_{2v}$\\ $7^2_{6}$ \quad $(4;10)$\\ }
\end{center}
\end{minipage}
\setlength{\unitlength}{1cm}
\begin{minipage}[t]{3.5cm}
\begin{picture}(3.5,3.5)
\leavevmode
\epsfxsize=2.5cm
\epsffile{7-hedrite8_1.eps}
\end{picture}\par
\begin{center}
{{\bf Nr.8-1} \quad $C_{s}$\\ $8^2_{13}$ \quad $(4;12)$\\ }
\end{center}
\end{minipage}
\setlength{\unitlength}{1cm}
\begin{minipage}[t]{3.5cm}
\begin{picture}(3.5,3.5)
\leavevmode
\epsfxsize=2.5cm
\epsffile{7-hedrite9_1.eps}
\end{picture}\par
\begin{center}
{{\bf Nr.9-1} \quad $C_{s}$\\ $9_{34}$ \quad $(18)$\\ }
\end{center}
\end{minipage}
\setlength{\unitlength}{1cm}
\begin{minipage}[t]{3.5cm}
\begin{picture}(3.5,3.5)
\leavevmode
\epsfxsize=2.5cm
\epsffile{7-hedrite10_1sec.eps}
\end{picture}\par
\begin{center}
{{\bf Nr.10-1} \quad $C_{s}$\\ $10_{121}$ \quad $(20)$\\ }
\end{center}
\end{minipage}
\setlength{\unitlength}{1cm}
\begin{minipage}[t]{3.5cm}
\begin{picture}(3.5,3.5)
\leavevmode
\epsfxsize=2.5cm
\epsffile{7-hedrite10_2sec.eps}
\end{picture}\par
\begin{center}
{{\bf Nr.10-2} \quad $C_{2v}$\\ $10^2_{111}$ \quad $(10^2)$\\ }
\end{center}
\end{minipage}
\setlength{\unitlength}{1cm}
\begin{minipage}[t]{3.5cm}
\begin{picture}(3.5,3.5)
\leavevmode
\epsfxsize=2.5cm
\epsffile{7-hedrite10_3.eps}
\end{picture}\par
\begin{center}
{{\bf Nr.10-3} \quad $C_{s}$\\ $\sim 10^2_{69}$ \quad $(8,!!12)$\\ }
\end{center}
\end{minipage}
\setlength{\unitlength}{1cm}
\begin{minipage}[t]{3.5cm}
\begin{picture}(3.5,3.5)
\leavevmode
\epsfxsize=2.5cm
\epsffile{7-hedrite11_1sec.eps}
\end{picture}\par
\begin{center}
{{\bf Nr.11-1} \quad $C_2$\\ $11_{288}$ \quad $(22)$\\ }
\end{center}
\end{minipage}
\setlength{\unitlength}{1cm}
\begin{minipage}[t]{3.5cm}
\begin{picture}(3.5,3.5)
\leavevmode
\epsfxsize=2.5cm
\epsffile{7-hedrite11_2.eps}
\end{picture}\par
\begin{center}
{{\bf Nr.11-2} \quad $C_{1}$\\ $11_{301}$ \quad $(22)$\\ }
\end{center}
\end{minipage}
\setlength{\unitlength}{1cm}
\begin{minipage}[t]{3.5cm}
\begin{picture}(3.5,3.5)
\leavevmode
\epsfxsize=2.5cm
\epsffile{7-hedrite11_3sec.eps}
\end{picture}\par
\begin{center}
{{\bf Nr.11-3} \quad $C_{s}$\\ $11^2_{150}$ \quad $(8,14)$\\ }
\end{center}
\end{minipage}
\setlength{\unitlength}{1cm}
\begin{minipage}[t]{3.5cm}
\begin{picture}(3.5,3.5)
\leavevmode
\epsfxsize=2.5cm
\epsffile{7-hedrite11_4sec.eps}
\end{picture}\par
\begin{center}
{{\bf Nr.11-4} \quad $C_{2v}$\\ $11^3_{487}$ \quad $(4^2;14)$ red.\\ }
\end{center}
\end{minipage}
\setlength{\unitlength}{1cm}
\begin{minipage}[t]{3.5cm}
\begin{picture}(3.5,3.5)
\leavevmode
\epsfxsize=2.5cm
\epsffile{7-hedrite12_1.eps}
\end{picture}\par
\begin{center}
{{\bf Nr.12-1} \quad $C_{1}$\\ $\sim 12_{361}$ \quad $(24)$\\ }
\end{center}
\end{minipage}
\setlength{\unitlength}{1cm}
\begin{minipage}[t]{3.5cm}
\begin{picture}(3.5,3.5)
\leavevmode
\epsfxsize=2.5cm
\epsffile{7-hedrite12_2.eps}
\end{picture}\par
\begin{center}
{{\bf Nr.12-2} \quad $C_{1}$\\ $?????$ \quad $(6;18)$\\ }
\end{center}
\end{minipage}
\setlength{\unitlength}{1cm}
\begin{minipage}[t]{3.5cm}
\begin{picture}(3.5,3.5)
\leavevmode
\epsfxsize=2.5cm
\epsffile{7-hedrite12_3sec.eps}
\end{picture}\par
\begin{center}
{{\bf Nr.12-3} \quad $C_{s}$\\ $????$ \quad $(10,14)$\\ }
\end{center}
\end{minipage}
\setlength{\unitlength}{1cm}
\begin{minipage}[t]{3.5cm}
\begin{picture}(3.5,3.5)
\leavevmode
\epsfxsize=2.5cm
\epsffile{7-hedrite12_4.eps}
\end{picture}\par
\begin{center}
{{\bf Nr.12-4} \quad $C_{2v}$\\ $????$ \quad $(6^2;12)$\\ }
\end{center}
\end{minipage}
\setlength{\unitlength}{1cm}
\begin{minipage}[t]{3.5cm}
\begin{picture}(3.5,3.5)
\leavevmode
\epsfxsize=2.5cm
\epsffile{7-hedrite12_5.eps}
\end{picture}\par
\begin{center}
{{\bf Nr.12-5} \quad $C_{s}$\\ $???$ \quad $(4^2;16)$ red.\\ }
\end{center}
\end{minipage}
}







\subsection{$6$-hedrites}
{\small
\setlength{\unitlength}{1cm}
\begin{minipage}[t]{3.5cm}
\begin{picture}(3.5,3.5)
\leavevmode
\epsfxsize=2.5cm
\epsffile{6-hedrite4_1.eps}
\end{picture}\par
\begin{center}
{{\bf Nr.4-1} \quad $D_{2d}$\\ $4_{1}$ \quad $(8)$\\ }
\end{center}
\end{minipage}
\setlength{\unitlength}{1cm}
\begin{minipage}[t]{3.5cm}
\begin{picture}(3.5,3.5)
\leavevmode
\epsfxsize=2.5cm
\epsffile{6-hedrite5_1.eps}
\end{picture}\par
\begin{center}
{{\bf Nr.5-1} \quad $C_{2v}$\\ $5^2_{1}$ \quad $(4;6)$\\ }
\end{center}
\end{minipage}
\setlength{\unitlength}{1cm}
\begin{minipage}[t]{3.5cm}
\begin{picture}(3.5,3.5)
\leavevmode
\epsfxsize=2.5cm
\epsffile{6-hedrite6_1sec.eps}
\end{picture}\par
\begin{center}
{{\bf Nr.6-1} \quad $C_{2}$\\ $6_{3}$ \quad $(12)$\\ }
\end{center}
\end{minipage}
\setlength{\unitlength}{1cm}
\begin{minipage}[t]{3.5cm}
\begin{picture}(3.5,3.5)
\leavevmode
\epsfxsize=2.5cm
\epsffile{6-hedrite6_2.eps}
\end{picture}\par
\begin{center}
{{\bf Nr.6-2} \quad $D_{2h}$\\ $\sim 6^2_{3}$ \quad $(6^2)$\\ }
\end{center}
\end{minipage}
\setlength{\unitlength}{1cm}
\begin{minipage}[t]{3.5cm}
\begin{picture}(3.5,3.5)
\leavevmode
\epsfxsize=2.5cm
\epsffile{6-hedrite7_1.eps}
\end{picture}\par
\begin{center}
{{\bf Nr.7-1} \quad $C_{2}$\\ $\sim 7_{7}$ \quad $(14)$\\ }
\end{center}
\end{minipage}
\setlength{\unitlength}{1cm}
\begin{minipage}[t]{3.5cm}
\begin{picture}(3.5,3.5)
\leavevmode
\epsfxsize=2.5cm
\epsffile{6-hedrite8_1sec.eps}
\end{picture}\par
\begin{center}
{{\bf Nr.8-1} \quad $D_{2h}$\\ $\sim 8_{12}$ \quad $(16)$\\ }
\end{center}
\end{minipage}
\setlength{\unitlength}{1cm}
\begin{minipage}[t]{3.5cm}
\begin{picture}(3.5,3.5)
\leavevmode
\epsfxsize=2.5cm
\epsffile{6-hedrite8_2sec.eps}
\end{picture}\par
\begin{center}
{{\bf Nr.8-2} \quad $C_{2}$\\ $8_{17}$ \quad $(16)$\\ }
\end{center}
\end{minipage}
\setlength{\unitlength}{1cm}
\begin{minipage}[t]{3.5cm}
\begin{picture}(3.5,3.5)
\leavevmode
\epsfxsize=2.5cm
\epsffile{6-hedrite8_3.eps}
\end{picture}\par
\begin{center}
{{\bf Nr.8-3} \quad $D_{2d}$\\ $8^2_{14}$ \quad $(4;12)$\\ }
\end{center}
\end{minipage}
\setlength{\unitlength}{1cm}
\begin{minipage}[t]{3.5cm}
\begin{picture}(3.5,3.5)
\leavevmode
\epsfxsize=2.5cm
\epsffile{6-hedrite8_4.eps}
\end{picture}\par
\begin{center}
{{\bf Nr.8-4} \quad $C_{2}$\\ $\sim 8^2_{8}$ \quad $(6;10)$\\ }
\end{center}
\end{minipage}
\setlength{\unitlength}{1cm}
\begin{minipage}[t]{3.5cm}
\begin{picture}(3.5,3.5)
\leavevmode
\epsfxsize=2.5cm
\epsffile{6-hedrite8_5.eps}
\end{picture}\par
\begin{center}
{{\bf Nr.8-5} \quad $D_{2h}$\\ $8^3_{6}$ \quad $(4,6^2)$\\ }
\end{center}
\end{minipage}
\setlength{\unitlength}{1cm}
\begin{minipage}[t]{3.5cm}
\begin{picture}(3.5,3.5)
\leavevmode
\epsfxsize=2.5cm
\epsffile{6-hedrite9_1.eps}
\end{picture}\par
\begin{center}
{{\bf Nr.9-1} \quad $C_{2}$\\ $\sim 9_{31}$ \quad $(18)$\\ }
\end{center}
\end{minipage}
\setlength{\unitlength}{1cm}
\begin{minipage}[t]{3.5cm}
\begin{picture}(3.5,3.5)
\leavevmode
\epsfxsize=2.5cm
\epsffile{6-hedrite9_2.eps}
\end{picture}\par
\begin{center}
{{\bf Nr.9-2} \quad $C_{1}$\\ $9_{33}$ \quad $(18)$\\ }
\end{center}
\end{minipage}
\setlength{\unitlength}{1cm}
\begin{minipage}[t]{3.5cm}
\begin{picture}(3.5,3.5)
\leavevmode
\epsfxsize=2.5cm
\epsffile{6-hedrite9_3sec.eps}
\end{picture}\par
\begin{center}
{{\bf Nr.9-3} \quad $C_{s}$\\ $9^2_{38}$ \quad $(4;14)$\\ }
\end{center}
\end{minipage}
\setlength{\unitlength}{1cm}
\begin{minipage}[t]{3.5cm}
\begin{picture}(3.5,3.5)
\leavevmode
\epsfxsize=2.5cm
\epsffile{6-hedrite9_4sec.eps}
\end{picture}\par
\begin{center}
{{\bf Nr.9-4} \quad $C_{2v}$\\ $9^3_{12}$ \quad $(4^2;10)$ red.\\ }
\end{center}
\end{minipage}
\setlength{\unitlength}{1cm}
\begin{minipage}[t]{3.5cm}
\begin{picture}(3.5,3.5)
\leavevmode
\epsfxsize=2.5cm
\epsffile{6-hedrite9_5.eps}
\end{picture}\par
\begin{center}
{{\bf Nr.9-5} \quad $C_{s}$\\ $9^3_{11}$ \quad $(4,6;8)$\\ }
\end{center}
\end{minipage}
%\setlength{\unitlength}{1cm}
%\begin{minipage}[t]{3.5cm}
%\begin{picture}(3.5,3.5)
%\leavevmode
%\epsfxsize=2.5cm
%\epsffile{6-hedrite10_1.eps}
%\end{picture}\par
%\begin{center}
%{{\bf Nr.10-1} \quad $C_{2}$\\ $10_{115}$ \quad $(20)$\\ }
%\end{center}
%\end{minipage}
\setlength{\unitlength}{1cm}
\begin{minipage}[t]{3.5cm}
\begin{picture}(3.5,3.5)
\leavevmode
\epsfxsize=2.5cm
\epsffile{6-hedrite10_3.eps}
\end{picture}\par
\begin{center}
{{\bf Nr.10-1} \quad $C_{2v}$\\ $10_{120}$ \quad $(20)$\\ }
\end{center}
\end{minipage}
\setlength{\unitlength}{1cm}
\begin{minipage}[t]{3.5cm}
\begin{picture}(3.5,3.5)
\leavevmode
\epsfxsize=2.5cm
\epsffile{6-hedrite10_4sec.eps}
\end{picture}\par
\begin{center}
{{\bf Nr.10-2} \quad $C_{2}$\\ $\sim 10_{88}$ \quad $(20)$\\ }
\end{center}
\end{minipage}
\setlength{\unitlength}{1cm}
\begin{minipage}[t]{3.5cm}
\begin{picture}(3.5,3.5)
\leavevmode
\epsfxsize=2.5cm
\epsffile{6-hedrite10_5sec.eps}
\end{picture}\par
\begin{center}
{{\bf Nr.10-3} \quad $C_{2}$\\ $\sim 10_{45}$ \quad $(20)$\\ }
\end{center}
\end{minipage}
\setlength{\unitlength}{1cm}
\begin{minipage}[t]{3.5cm}
\begin{picture}(3.5,3.5)
\leavevmode
\epsfxsize=2.5cm
\epsffile{6-hedrite10_6sec.eps}
\end{picture}\par
\begin{center}
{{\bf Nr.10-4} \quad $C_{2}$\\ $10_{115}$ \quad $(20)$\\ }
\end{center}
\end{minipage}
\setlength{\unitlength}{1cm}
\begin{minipage}[t]{3.5cm}
\begin{picture}(3.5,3.5)
\leavevmode
\epsfxsize=2.5cm
\epsffile{6-hedrite10_7.eps}
\end{picture}\par
\begin{center}
{{\bf Nr.10-5} \quad $D_{2h}$\\ $\sim 10^2_{87}$ \quad $(10^2)$\\ }
\end{center}
\end{minipage}
\setlength{\unitlength}{1cm}
\begin{minipage}[t]{3.5cm}
\begin{picture}(3.5,3.5)
\leavevmode
\epsfxsize=2.5cm
\epsffile{6-hedrite10_9sec.eps}
\end{picture}\par
\begin{center}
{{\bf Nr.10-6} \quad $C_{2}$\\ $10^2_{86}$ \quad $(6;14)$\\ }
\end{center}
\end{minipage}
\setlength{\unitlength}{1cm}
\begin{minipage}[t]{3.5cm}
\begin{picture}(3.5,3.5)
\leavevmode
\epsfxsize=2.5cm
\epsffile{6-hedrite10_8sec.eps}
\end{picture}\par
\begin{center}
{{\bf Nr.10-7} \quad $C_{2}$\\ $10^2_{43}$ \quad $(4;16)$\\ }
\end{center}
\end{minipage}
\setlength{\unitlength}{1cm}
\begin{minipage}[t]{3.5cm}
\begin{picture}(3.5,3.5)
\leavevmode
\epsfxsize=2.5cm
\epsffile{6-hedrite10_10sec.eps}
\end{picture}\par
\begin{center}
{{\bf Nr.10-8} \quad $C_{2h}$\\ $\sim 10^3_{136}$ \quad $(4;8^2)$\\ }
\end{center}
\end{minipage}
\setlength{\unitlength}{1cm}
\begin{minipage}[t]{3.5cm}
\begin{picture}(3.5,3.5)
\leavevmode
\epsfxsize=2.5cm
\epsffile{6-hedrite10_11sec.eps}
\end{picture}\par
\begin{center}
{{\bf Nr.10-9} \quad $C_{2v}$\\ $10^3_{136}$ \quad $(4,6;10)$\\ }
\end{center}
\end{minipage}
\setlength{\unitlength}{1cm}
\begin{minipage}[t]{3.5cm}
\begin{picture}(3.5,3.5)
\leavevmode
\epsfxsize=2.5cm
\epsffile{6-hedrite11_1.eps}
\end{picture}\par
\begin{center}
{{\bf Nr.11-1} \quad $C_{2v}$\\ $11_{332}$ \quad $(22)$\\ }
\end{center}
\end{minipage}
\setlength{\unitlength}{1cm}
\begin{minipage}[t]{3.5cm}
\begin{picture}(3.5,3.5)
\leavevmode
\epsfxsize=2.5cm
\epsffile{6-hedrite11_2.eps}
\end{picture}\par
\begin{center}
{{\bf Nr.11-2} \quad $C_{2v}$\\ $11_{297}$ \quad $(22)$\\ }
\end{center}
\end{minipage}
\setlength{\unitlength}{1cm}
\begin{minipage}[t]{3.5cm}
\begin{picture}(3.5,3.5)
\leavevmode
\epsfxsize=2.5cm
\epsffile{6-hedrite11_3.eps}
\end{picture}\par
\begin{center}
{{\bf Nr.11-3} \quad $C_{1}$\\ $\sim 11_{125}$ \quad $(22)$\\ }
\end{center}
\end{minipage}
\setlength{\unitlength}{1cm}
\begin{minipage}[t]{3.5cm}
\begin{picture}(3.5,3.5)
\leavevmode
\epsfxsize=2.5cm
\epsffile{6-hedrite11_6.eps}
\end{picture}\par
\begin{center}
{{\bf Nr.11-4} \quad $C_{2}$\\ $11^2_{317}$ \quad $(8;14)$\\ }
\end{center}
\end{minipage}
\setlength{\unitlength}{1cm}
\begin{minipage}[t]{3.5cm}
\begin{picture}(3.5,3.5)
\leavevmode
\epsfxsize=2.5cm
\epsffile{6-hedrite11_7.eps}
\end{picture}\par
\begin{center}
{{\bf Nr.11-5} \quad $C_{s}$\\ $11^2_{351}$ \quad $(10,12)$\\ }
\end{center}
\end{minipage}
\setlength{\unitlength}{1cm}
\begin{minipage}[t]{3.5cm}
\begin{picture}(3.5,3.5)
\leavevmode
\epsfxsize=2.5cm
\epsffile{6-hedrite11_4.eps}
\end{picture}\par
\begin{center}
{{\bf Nr.11-6} \quad $C_{s}$\\ $11^2_{????}$ \quad $(8,!!14)$\\ }
\end{center}
\end{minipage}
\setlength{\unitlength}{1cm}
\begin{minipage}[t]{3.5cm}
\begin{picture}(3.5,3.5)
\leavevmode
\epsfxsize=2.5cm
\epsffile{6-hedrite11_5.eps}
\end{picture}\par
\begin{center}
{{\bf Nr.11-7} \quad $C_{2}$\\ $11^2_{?????}$ \quad $(8;14)$\\ }
!!exchange it with Nr.11-5!!
\end{center}
\end{minipage}
\setlength{\unitlength}{1cm}
\begin{minipage}[t]{3.5cm}
\begin{picture}(3.5,3.5)
\leavevmode
\epsfxsize=2.5cm
\epsffile{6-hedrite12_2sec.eps}
\end{picture}\par
\begin{center}
{{\bf Nr.12-1} \quad $D_{2d}$\\ $\sim 12_{477}$ \quad $(24)$\\ }
\end{center}
\end{minipage}
\setlength{\unitlength}{1cm}
\begin{minipage}[t]{3.5cm}
\begin{picture}(3.5,3.5)
\leavevmode
\epsfxsize=2.5cm
\epsffile{6-hedrite12_4sec.eps}
\end{picture}\par
\begin{center}
{{\bf Nr.12-2} \quad $D_{2}$\\ $12_{1152}$ \quad $(24)$\\ }
\end{center}
\end{minipage}
\setlength{\unitlength}{1cm}
\begin{minipage}[t]{3.5cm}
\begin{picture}(3.5,3.5)
\leavevmode
\epsfxsize=2.5cm
\epsffile{6-hedrite12_1sec.eps}
\end{picture}\par
\begin{center}
{{\bf Nr.12-3} \quad $C_{2}$\\ $\sim 12_{499}$ \quad $(24)$\\ }
\end{center}
\end{minipage}
\setlength{\unitlength}{1cm}
\begin{minipage}[t]{3.5cm}
\begin{picture}(3.5,3.5)
\leavevmode
\epsfxsize=2.5cm
\epsffile{6-hedrite12_3sec.eps}
\end{picture}\par
\begin{center}
{{\bf Nr.12-4} \quad $C_{2}$\\ $\sim 12_{458}$ \quad $(24)$\\ }
\end{center}
\end{minipage}
\setlength{\unitlength}{1cm}
\begin{minipage}[t]{3.5cm}
\begin{picture}(3.5,3.5)
\leavevmode
\epsfxsize=2.5cm
\epsffile{6-hedrite12_5sec.eps}
\end{picture}\par
\begin{center}
{{\bf Nr.12-5} \quad $C_{2}$\\ $12_{1102}$ \quad $(24)$\\ }
\end{center}
\end{minipage}
\setlength{\unitlength}{1cm}
\begin{minipage}[t]{3.5cm}
\begin{picture}(3.5,3.5)
\leavevmode
\epsfxsize=2.5cm
\epsffile{6-hedrite12_7sec.eps}
\end{picture}\par
\begin{center}
{{\bf Nr.12-6} \quad $C_{2}$\\ $12_{1167}$ \quad $(24)$\\ }
\end{center}
\end{minipage}
\setlength{\unitlength}{1cm}
\begin{minipage}[t]{3.5cm}
\begin{picture}(3.5,3.5)
\leavevmode
\epsfxsize=2.5cm
\epsffile{6-hedrite12_6.eps}
\end{picture}\par
\begin{center}
{{\bf Nr.12-7} \quad $C_{1}$\\ $\sim 12_{626}$ \quad $(24)$\\ }
\end{center}
\end{minipage}
\setlength{\unitlength}{1cm}
\begin{minipage}[t]{3.5cm}
\begin{picture}(3.5,3.5)
\leavevmode
\epsfxsize=2.5cm
\epsffile{6-hedrite12_8.eps}
\end{picture}\par
\begin{center}
{{\bf Nr.12-8} \quad $C_{1}$\\ $????$ \quad $(6;18)$\\ }
\end{center}
\end{minipage}
\setlength{\unitlength}{1cm}
\begin{minipage}[t]{3.5cm}
\begin{picture}(3.5,3.5)
\leavevmode
\epsfxsize=2.5cm
\epsffile{6-hedrite12_9sec.eps}
\end{picture}\par
\begin{center}
{{\bf Nr.12-9} \quad $C_{2}$\\ $????$ \quad $(10,14)$\\ }
\end{center}
\end{minipage}
\setlength{\unitlength}{1cm}
\begin{minipage}[t]{3.5cm}
\begin{picture}(3.5,3.5)
\leavevmode
\epsfxsize=2.5cm
\epsffile{6-hedrite12_10sec.eps}
\end{picture}\par
\begin{center}
{{\bf Nr.12-10} \quad $C_{s}$\\ $?????$ \quad $(8,16)$\\ }
\end{center}
\end{minipage}
\setlength{\unitlength}{1cm}
\begin{minipage}[t]{3.5cm}
\begin{picture}(3.5,3.5)
\leavevmode
\epsfxsize=2.5cm
\epsffile{6-hedrite12_11sec.eps}
\end{picture}\par
\begin{center}
{{\bf Nr.12-11} \quad $D_{2d}$\\ $?????$ \quad $(4^2;16)$ red.\\ }
\end{center}
\end{minipage}
\setlength{\unitlength}{1cm}
\begin{minipage}[t]{3.5cm}
\begin{picture}(3.5,3.5)
\leavevmode
\epsfxsize=2.5cm
\epsffile{6-hedrite12_12.eps}
\end{picture}\par
\begin{center}
{{\bf Nr.12-12} \quad $C_{2v}$\\ $????$ \quad $(8;8^2)$\\ }
\end{center}
\end{minipage}
\setlength{\unitlength}{1cm}
\begin{minipage}[t]{3.5cm}
\begin{picture}(3.5,3.5)
\leavevmode
\epsfxsize=2.5cm
\epsffile{6-hedrite12_13.eps}
\end{picture}\par
\begin{center}
{{\bf Nr.12-13} \quad $C_{s}$\\ $?????$ \quad $(6;8,10)$\\ }
\end{center}
\end{minipage}
\setlength{\unitlength}{1cm}
\begin{minipage}[t]{3.5cm}
\begin{picture}(3.5,3.5)
\leavevmode
\epsfxsize=2.5cm
\epsffile{6-hedrite12_14.eps}
\end{picture}\par
\begin{center}
{{\bf Nr.12-14} \quad $D_{2h}$\\ $????$ \quad $(4^2,8^2)$ red.\\ }
\end{center}
\end{minipage}
}












\subsection{$5$-hedrites}
{\small
\setlength{\unitlength}{1cm}
\begin{minipage}[t]{3.5cm}
\begin{picture}(3.5,3.5)
\leavevmode
\epsfxsize=2.5cm
\epsffile{5-hedrite3_1.eps}
\end{picture}\par
\begin{center}
{{\bf Nr.3-1} \quad $D_{3h}$\\ $3_{1}$ \quad $(6)$\\ }
\end{center}
\end{minipage}
\setlength{\unitlength}{1cm}
\begin{minipage}[t]{3.5cm}
\begin{picture}(3.5,3.5)
\leavevmode
\epsfxsize=2.5cm
\epsffile{5-hedrite5_1.eps}
\end{picture}\par
\begin{center}
{{\bf Nr.5-1} \quad $C_{2v}$\\ $5_{2}$ \quad $(10)$\\ }
\end{center}
\end{minipage}
\setlength{\unitlength}{1cm}
\begin{minipage}[t]{3.5cm}
\begin{picture}(3.5,3.5)
\leavevmode
\epsfxsize=2.5cm
\epsffile{5-hedrite6_1.eps}
\end{picture}\par
\begin{center}
{{\bf Nr.6-1} \quad $C_{2v}$\\ $6^2_{3}$ \quad $(4;8)$\\ }
\end{center}
\end{minipage}
\setlength{\unitlength}{1cm}
\begin{minipage}[t]{3.5cm}
\begin{picture}(3.5,3.5)
\leavevmode
\epsfxsize=2.5cm
\epsffile{5-hedrite6_2.eps}
\end{picture}\par
\begin{center}
{{\bf Nr.6-2} \quad $D_{3h}$\\ $6^3_{1}$ \quad $(4^3)$\\ }
\end{center}
\end{minipage}
\setlength{\unitlength}{1cm}
\begin{minipage}[t]{3.5cm}
\begin{picture}(3.5,3.5)
\leavevmode
\epsfxsize=2.5cm
\epsffile{5-hedrite7_1sec.eps}
\end{picture}\par
\begin{center}
{{\bf Nr.7-1} \quad $C_{2v}$\\ $\sim 7_{5}$ \quad $(14)$\\ }
\end{center}
\end{minipage}
\setlength{\unitlength}{1cm}
\begin{minipage}[t]{3.5cm}
\begin{picture}(3.5,3.5)
\leavevmode
\epsfxsize=2.5cm
\epsffile{5-hedrite7_2.eps}
\end{picture}\par
\begin{center}
{{\bf Nr.7-2} \quad $C_{s}$\\ $7^2_{5}$ \quad $(4;10)$\\ }
\end{center}
\end{minipage}
\setlength{\unitlength}{1cm}
\begin{minipage}[t]{3.5cm}
\begin{picture}(3.5,3.5)
\leavevmode
\epsfxsize=2.5cm
\epsffile{5-hedrite7_3sec.eps}
\end{picture}\par
\begin{center}
{{\bf Nr.7-3} \quad $C_{2v}$\\ $7^3_1$ \quad $(4^2;6)$\\ }
\end{center}
\end{minipage}
\setlength{\unitlength}{1cm}
\begin{minipage}[t]{3.5cm}
\begin{picture}(3.5,3.5)
\leavevmode
\epsfxsize=2.5cm
\epsffile{5-hedrite8_1sec.eps}
\end{picture}\par
\begin{center}
{{\bf Nr.8-1} \quad $C_{2}$\\ $\sim 8_{15}$ \quad $(16)$\\ }
\end{center}
\end{minipage}
\setlength{\unitlength}{1cm}
\begin{minipage}[t]{3.5cm}
\begin{picture}(3.5,3.5)
\leavevmode
\epsfxsize=2.5cm
\epsffile{5-hedrite9_1sec.eps}
\end{picture}\par
\begin{center}
{{\bf Nr.9-1} \quad $C_{2}$\\ $9_{38}$ \quad $(18)$\\ }
\end{center}
\end{minipage}
\setlength{\unitlength}{1cm}
\begin{minipage}[t]{3.5cm}
\begin{picture}(3.5,3.5)
\leavevmode
\epsfxsize=2.5cm
\epsffile{5-hedrite9_2sec.eps}
\end{picture}\par
\begin{center}
{{\bf Nr.9-2} \quad $C_{2v}$\\ $\sim 9_{18}$ \quad $(18)$\\ }
\end{center}
\end{minipage}
\setlength{\unitlength}{1cm}
\begin{minipage}[t]{3.5cm}
\begin{picture}(3.5,3.5)
\leavevmode
\epsfxsize=2.5cm
\epsffile{5-hedrite10_3.eps}
\end{picture}\par
\begin{center}
{{\bf Nr.10-1} \quad $C_{2v}$\\ $10^3_{155}$ \quad $(4^2;12)$ red.\\ }
\end{center}
\end{minipage}
\setlength{\unitlength}{1cm}
\begin{minipage}[t]{3.5cm}
\begin{picture}(3.5,3.5)
\leavevmode
\epsfxsize=2.5cm
\epsffile{5-hedrite10_1.eps}
\end{picture}\par
\begin{center}
{{\bf Nr.10-2} \quad $C_{1}$\\ $10^2_{85}$ \quad $(6;!!14)$\\ }
\end{center}
\end{minipage}
\setlength{\unitlength}{1cm}
\begin{minipage}[t]{3.5cm}
\begin{picture}(3.5,3.5)
\leavevmode
\epsfxsize=2.5cm
\epsffile{5-hedrite10_2sec.eps}
\end{picture}\par
\begin{center}
{{\bf Nr.10-3} \quad $C_{2v}$\\ $10^4_{173}$ \quad $(4^2,6^2)$ red.\\ }
\end{center}
\end{minipage}
\setlength{\unitlength}{1cm}
\begin{minipage}[t]{3.5cm}
\begin{picture}(3.5,3.5)
\leavevmode
\epsfxsize=2.5cm
\epsffile{5-hedrite11_2sec.eps}
\end{picture}\par
\begin{center}
{{\bf Nr.11-1} \quad $C_{2v}$\\ $\sim 11_{236}$ \quad $(22)$\\ }
\end{center}
\end{minipage}
\setlength{\unitlength}{1cm}
\begin{minipage}[t]{3.5cm}
\begin{picture}(3.5,3.5)
\leavevmode
\epsfxsize=2.5cm
\epsffile{5-hedrite11_1sec.eps}
\end{picture}\par
\begin{center}
{{\bf Nr.11-2} \quad $C_{2}$\\ $\sim 11_{124}$ \quad $(22)$\\ }
\end{center}
\end{minipage}
\setlength{\unitlength}{1cm}
\begin{minipage}[t]{3.5cm}
\begin{picture}(3.5,3.5)
\leavevmode
\epsfxsize=2.5cm
\epsffile{5-hedrite11_3.eps}
\end{picture}\par
\begin{center}
{{\bf Nr.11-3} \quad $C_{1}$\\ $11^2_{226}$ \quad $(6;16)$\\ }
\end{center}
\end{minipage}
\setlength{\unitlength}{1cm}
\begin{minipage}[t]{3.5cm}
\begin{picture}(3.5,3.5)
\leavevmode
\epsfxsize=2.5cm
\epsffile{5-hedrite11_4.eps}
\end{picture}\par
\begin{center}
{{\bf Nr.11-4} \quad $C_{s}$\\ $11^3_{500}$ \quad $(4^2;14)$ red.\\ }
\end{center}
\end{minipage}
\setlength{\unitlength}{1cm}
\begin{minipage}[t]{3.5cm}
\begin{picture}(3.5,3.5)
\leavevmode
\epsfxsize=2.5cm
\epsffile{5-hedrite11_5.eps}
\end{picture}\par
\begin{center}
{{\bf Nr.11-5} \quad $C_{s}$\\ $11^4_{547}$ \quad $(4^2,6;8)$ red.\\ }
\end{center}
\end{minipage}
\setlength{\unitlength}{1cm}
\begin{minipage}[t]{3.5cm}
\begin{picture}(3.5,3.5)
\leavevmode
\epsfxsize=2.5cm
\epsffile{5-hedrite12_1.eps}
\end{picture}\par
\begin{center}
{{\bf Nr.12-1} \quad $C_{1}$\\  $\sim 12_{431}$ \quad $(24)$\\ }
\end{center}
\end{minipage}
\setlength{\unitlength}{1cm}
\begin{minipage}[t]{3.5cm}
\begin{picture}(3.5,3.5)
\leavevmode
\epsfxsize=2.5cm
\epsffile{5-hedrite12_2.eps}
\end{picture}\par
\begin{center}
{{\bf Nr.12-2} \quad $C_{2v}$\\ $????$ \quad $(12^2!!)$\\ }
\end{center}
\end{minipage}
\setlength{\unitlength}{1cm}
\begin{minipage}[t]{3.5cm}
\begin{picture}(3.5,3.5)
\leavevmode
\epsfxsize=2.5cm
\epsffile{5-hedrite12D3hsec.eps}
\end{picture}\par
\begin{center}
{{\bf Nr.12-3} \quad $D_{3h}$ red.\\ $????$ \quad $(12^2!!$\\ }
!!move ``red.'' down!!
\end{center}
\end{minipage}
}







\subsection{$4$-hedrites}
{\small
\setlength{\unitlength}{1cm}
\begin{minipage}[t]{3.5cm}
\begin{picture}(3.5,3.5)
\leavevmode
\epsfxsize=2.5cm
\epsffile{4-hedrite2_1.eps}
\end{picture}\par
\begin{center}
{{\bf Nr.2-1} \quad $D_{4h}$\\ $2^2_1$ \quad $(2^2)$\\ }
\end{center}
\end{minipage}
\setlength{\unitlength}{1cm}
\begin{minipage}[t]{3.5cm}
\begin{picture}(3.5,3.5)
\leavevmode
\epsfxsize=2.5cm
\epsffile{4-hedrite4_1.eps}
\end{picture}\par
\begin{center}
{{\bf Nr.4-1} \quad $D_{4h}$\\ $4^2_1$ \quad $(4^2)$\\ }
\end{center}
\end{minipage}
\setlength{\unitlength}{1cm}
\begin{minipage}[t]{3.5cm}
\begin{picture}(3.5,3.5)
\leavevmode
\epsfxsize=2.5cm
\epsffile{4-hedrite6_1sec.eps}
\end{picture}\par
\begin{center}
{{\bf Nr.6-1} \quad $D_{2d}$\\ $6^2_2$ \quad $(6^2)$\\ }
\end{center}
\end{minipage}
\setlength{\unitlength}{1cm}
\begin{minipage}[t]{3.5cm}
\begin{picture}(3.5,3.5)
\leavevmode
\epsfxsize=2.5cm
\epsffile{4-hedrite8_1.eps}
\end{picture}\par
\begin{center}
{{\bf Nr.8-1} \quad $D_{2h}$\\ $\sim 8^2_4$ \quad $(8^2)$\\ }
\end{center}
\end{minipage}
\setlength{\unitlength}{1cm}
\begin{minipage}[t]{3.5cm}
\begin{picture}(3.5,3.5)
\leavevmode
\epsfxsize=2.5cm
\epsffile{4-hedrite8_2.eps}
\end{picture}\par
\begin{center}
{{\bf Nr.8-2} \quad $D_{2d}$\\ $8^3_4$ \quad $(4^2,8)$ red.\\ }
\end{center}
\end{minipage}
\setlength{\unitlength}{1cm}
\begin{minipage}[t]{3.5cm}
\begin{picture}(3.5,3.5)
\leavevmode
\epsfxsize=2.5cm
\epsffile{4-hedrite8_3.eps}
\end{picture}\par
\begin{center}
{{\bf Nr.8-3} \quad $D_{4h}$\\ $8^4_{1}$ \quad $(4^4)$ red.\\ }
\end{center}
\end{minipage}
\setlength{\unitlength}{1cm}
\begin{minipage}[t]{3.5cm}
\begin{picture}(3.5,3.5)
\leavevmode
\epsfxsize=2.5cm
\epsffile{4-hedrite10_1sec.eps}
\end{picture}\par
\begin{center}
{{\bf Nr.10-1} \quad $D_4$\\ $10^2_{121}$ \quad $(10^2!!)$\\ }
\end{center}
\end{minipage}
\setlength{\unitlength}{1cm}
\begin{minipage}[t]{3.5cm}
\begin{picture}(3.5,3.5)
\leavevmode
\epsfxsize=2.5cm
\epsffile{4-hedrite10_2sec.eps}
\end{picture}\par
\begin{center}
{{\bf Nr.10-2} \quad $D_{2d}$\\ $\sim 10^2_{120}$ \quad $(10^2!!)$\\ }
\end{center}
\end{minipage}
\setlength{\unitlength}{1cm}
\begin{minipage}[t]{3.5cm}
\begin{picture}(3.5,3.5)
\leavevmode
\epsfxsize=2.5cm
\epsffile{4-hedrite12_1.eps}
\end{picture}\par
\begin{center}
{{\bf Nr.12-1} \quad $D_{2h}$\\ $?????$ \quad $(12^2)$\\ }
\end{center}
\end{minipage}
\setlength{\unitlength}{1cm}
\begin{minipage}[t]{3.5cm}
\begin{picture}(3.5,3.5)
\leavevmode
\epsfxsize=2.5cm
\epsffile{4-hedrite12_2.eps}
\end{picture}\par
\begin{center}
{{\bf Nr.12-2} \quad $D_2$ red.\\ $????$ \quad $(6^2,12)$\\ }
\end{center}
\end{minipage}
\setlength{\unitlength}{1cm}
\begin{minipage}[t]{3.5cm}
\begin{picture}(3.5,3.5)
\leavevmode
\epsfxsize=2.5cm
\epsffile{4-hedrite12_3.eps}
\end{picture}\par
\begin{center}
{{\bf Nr.12-3} \quad $D_{2d}$\\ $????$ \quad $(4^3,12)$ red.\\ }
\end{center}
\end{minipage}
\setlength{\unitlength}{1cm}
\begin{minipage}[t]{3.5cm}
\begin{picture}(3.5,3.5)
\leavevmode
\epsfxsize=2.5cm
\epsffile{4-hedrite12_4.eps}
\end{picture}\par
\begin{center}
{{\bf Nr.12-4} \quad $D_{2h}$\\ $????$ \quad $(4^3,6^2)$ red.\\ }
\end{center}
\end{minipage}
}




\begin{table}
\begin{center}
{\small
\begin{minipage}{6cm}
\begin{tabular}{||l|l|l|l||}
\hline
Nr.	&Grp 	&CC-vector	&link\\\hline
\multicolumn{4}{||c||}{$5$-hedrites}\\\hline
13-1	&$C_{2v}$	&$26$		&????\\
13-2	&$C_1$	&$26$		&????\\
13-3	&$C_2$	&$6^2; 14$	&????\\
13-4	&$C_s$	&$6; 8, 12$	&????\\\hline
14-1	&$C_s$	&$28$		&????\\
14-2	&$C_1$	&$6; 22$		&????\\
14-3	&$C_1$	&$8; 20$		&????\\
14-4	&$C_{2v}$	&$8^2; 12$	&????\\
14-5	&$C_{2v}$	&$4^3; 16$	&????\\
14-6	&$C_{2v}$	&$6^2; 8^2$	&????\\
14-7	&$C_{2v}$	&$4^3, 8^2$	&????\\\hline
\hline
\multicolumn{4}{||c||}{$7$-hedrites}\\\hline
13-1	&$C_s$	&$26$		&????\\
13-2	&$C_1$	&$26$		&????\\
13-3	&$C_1$	&$6; 20$		&????\\
13-4	&$C_1$	&$10, 16$		&????\\
13-5	&$C_s$	&$10, 16$		&????\\
13-6	&$C_{2v}$	&$6^2; 14$	&????\\
13-7	&$C_{s}$	&$6^2; 14$	&????\\\hline
14-1	&$C_1$	&$28$		&????\\
14-2	&$C_1$	&$28$		&????\\
14-3	&$C_1$	&$28$		&????\\
14-4	&$C_1$	&$28$		&????\\
14-5	&$C_1$	&$6; 22$		&????\\
14-6	&$C_s$	&$10, 18$		&????\\
14-7	&$C_2$	&$14^2$		&????\\
14-8	&$C_s$	&$6^2, 16$	&????\\
14-9	&$C_{2v}$	&$6^2, 16$	&????\\\hline
\hline
\multicolumn{4}{||c||}{$8$-hedrites}\\\hline
13-1	&$C_2$	&$26$		&????\\
13-2	&$C_{2v}$	&$6^2; 14$	&????\\\hline
14-1	&$C_2$	&$28$		&????\\
14-2	&$C_s$	&$6; 22$		&????\\
14-3	&$D_2$	&$6; 22$		&????\\
14-4	&$D_{2d}$	&$14^2$		&????\\
14-5	&$C_2$	&$6^2; 16$	&????\\
14-6	&$D_2$	&$8; 10^2$	&????\\
14-7	&$D_{4h}$	&$6^2, 8^2$	&????\\
14-8	&$D_{4h}$	&$4^3, 8^2$	&????\\\hline
\end{tabular}
\end{minipage}
\begin{minipage}[t]{6cm}
\begin{tabular}{||l|l|l|l||}
\hline
Nr.	&Grp 	&CC-vector	&link\\\hline
\multicolumn{4}{||c||}{$4$-hedrites}\\\hline
14-1	&$D_{2d}$	&$14^2$		&????\\\hline
14-2	&$D_2$	&$14^2$		&????\\	
\hline
\multicolumn{4}{||c||}{$6$-hedrites}\\\hline
13-1	&$C_2$	&$26$		&????\\
13-2 	&$C_2$	&$26$		&????\\
13-3 	&$C_2$	&$26$		&????\\
13-4 	&$C_s$	&$26$		&????\\
13-5 	&$C_1$	&$26$		&????\\
13-6	&$C_1$	&$26$		&????\\
13-7	&$C_1$	&$26$		&????\\
13-8	&$C_1$	&$8; 18$		&????\\
13-9	&$C_s$	&$12, 14$		&????\\
13-10	&$C_s$	&$4^2; 18$	&????\\
13-11	&$C_{2v}$	&$8^2, 10$	&????\\
13-12	&$C_{2v}$	&$8^2; 10$	&????\\
13-13	&$C_{2v}$	&$4^3; 14$	&????\\
13-14	&$C_s$	&$4^2,8;10$	&????\\\hline
14-1 	&$C_{2h}$	&$28$		&????\\
14-2	&$C_{2}$	&$28$		&????\\
14-3	&$C_2$	&$28$		&????\\
14-4 	&$C_2$	&$28$		&????\\
14-5 	&$C_2$	&$28$		&????\\
14-6 	&$C_1$	&$28$		&????\\
14-7 	&$C_1$	&$28$		&????\\
14-8	&$C_{2}$	&$6; 22$		&????\\
14-9 	&$C_2$	&$6; 22$		&????\\
14-10	&$C_2$	&$10; 18$		&????\\
14-11	&$C_2$	&$10; 18$		&????\\
14-12	&$C_s$	&$10, 18$		&????\\
14-13	&$D_{2h}$	&$14^2$		&????\\
14-14	&$C_{2v}$	&$14^2$		&????\\
14-15	&$C_2$	&$12, 16$		&????\\
14-16	&$C_{s}$	&$12, 16$		&????\\
14-17	&$C_2$	&$4^2; 20$	&????\\
14-18	&$C_2$	&$8; 10^2$	&????\\
14-19	&$C_2$	&$6^2; 16$	&????\\
14-20	&$D_{2h}$	&$6^2, 8^2$	&????\\
14-21	&$C_{2v}$	&$6^3, 10$	&????\\
14-22	&$C_{2h}$	&$4^2; 10^2$	&????\\
14-23	&$C_{2v}$	&$4^2, 8; 12$	&????\\\hline
\end{tabular}
\end{minipage}
}
\end{center}
\caption{All $i$-hedrites with $13$ and $14$ vertices}
\label{tab:i-hedrite13_14}
\end{table}











\begin{thebibliography}{99}


%\bibitem[Alex50]{A}
%A.D.Alexandrov, {\em Vypuklie Mnogogranniki},
%GITL, Moscow, 1950. Translated in German as {\em Convexe Polyheder},
%\item Akademie-Verlag, Berlin, 1958.

\bibitem[DDF02]{DDF}
M.Deza, M.Dutour and P.W.Fowler,
{\em Zigzags, Railroads and Knots in Fullerenes},
submitted.


\bibitem[DeGr99]{DG2}
M.Deza and V.P.Grishukhin,
{\em $l_1$-embeddable polyhedra},
in: Algebras and Combinatorics, Int. Congress CAC '97 Hong Kong,
ed. by K.P. Shum et al., Springer-Verlag (1999), pp. 189--210.


\bibitem[DHL02]{DHL}
M.Deza, T.Huang and K-W.Lih,
{\em Central Circuit Coverings of Octahedrites and Medial Polyhedra},
Journal of Math. Research and Exposition {\bf 22-1} (2002) 49--66.


\bibitem[DeSt02]{DSt}
M.Deza and M.Shtogrin,
{\em Octahedrites}, 
Symmetry, Special Issue ``Polyhedra and Science and Art'', 2002.


\bibitem[GaKe94]{GK}
M.L.Gargano and J.W.Kennedy,
{\em Gaussian graphs and digraphs}, Congressus Numerantium {\bf 101}
(1994) 161--170.


\bibitem[GoRo01]{God}
C.Godsil and G.Royle, {\em Algebraic Graph Theory}, Graduate Texts in 
Mathematics {\bf 207}, Springer-Verlag, Berlin - New York, 2001.


\bibitem[Gr\"{u}n67]{Gr}
B.Gr\"{u}nbaum, {\em Convex polytopes}, Interscience, New York, 1967.


\bibitem[Gr\"{u}n72]{Gr2}
B.Gr\"{u}nbaum, {\em Arrangements and Spreads}, Regional Conference Series in
Mathematics {\bf 10}, American Mathematical Society, 1972.


\bibitem[Gr\"{u}nMo63]{GrMo}
B.Gr\"{u}nbaum and T.S.Motzkin, {\em The number of hexagons and the simplicity
of geodesics on certain polyhedra}, Canadian Journal of Mathematics {\bf 15} (1963) 744--751.


\bibitem[Harb97]{Ha}
H.Harborth, {\em Eulerian straight ahead cycles in drawings of complete
bipartite graphs}, Bericht 97/23, Institute f\"{u}r Mathematik, Tech. 
Universit\"{a}t
Braunschweg, 1997.


\bibitem[Heid98]{He}
O.Heidemeier, {\em Die Erzeugung von 4-regul\"{a}ren, planaren,
simplen, zusammenh\"{a}ngenden Graphen mit vorgegebenen Fl\"{a}chentypen},
Diplomarbeit, Universit\"{a}t Bielefeld, Fakult\"{a}t f\"{u}r Wirtschaft und
Mathematik, 1998. 

\bibitem[Jeo95]{Je}
D.Jeong, {\em Realizations with a cut-through Eulerian circuit},
Discrete Mathematics {\bf 137} (1995) 265--275.


%\bibitem[Kau87]{Kau}
%L.H.Kauffman, {\em On knots}, Princeton University Press, Princeton, New
%Jersey, 1987.

\bibitem[Kaw96]{Kaw}
A.Kawauchi, {\em A survey of knot theory}, Birkh\"{a}user, 1996.

\bibitem[Kir85]{Kirk}
T. Kirkman, {\em The enumeration, description, and construction of knots with fewer than $10$ crossings}, Trans. Roy. Soc. Edin. {\bf 32} (1885), 281--309.


\bibitem[Kot69]{Ko}
A.Kotzig, {\em Eulerian lines in finite 4-valent graphs and their 
transformations}, in: Theory of Graphs, Proceedings of a colloquium, 
Tihany 1966, ed.!! by P.Erdos and G.Katona, Academic Press, 
New York (1969), pp. 219--230.


\bibitem[PTZ96]{PTZ}
T.Pisanski, T.Tucker and A.Zitnik, {\em Eulerian Embedding of Graphs},
University of Ljubljana, IMMF Preprint Series {\bf 34}
(1996) 531.

\bibitem[Rol76]{Rolf}
D.Rolfsen, {\em Knots and Links}, Mathematics Lecture Series 7, Publish or
Perish, Berkeley, 1976;
second corrected printing!!: Publish or Perish, Houston, 1990.

\bibitem[Sha75]{Sh}
H.Shank, {\em The Theory of Left-Right Paths}, in: Combinatorial 
Mathematics III,
Proceedings of 3rd Australian Conference, St Lucia 1974, Lecture Notes in
Mathematics {\bf 452}, Springer-Verlag , Berlin - New York (1975),  pp. 42--54.

\bibitem[Thi]{T}
M.Thistlewaite, \url{http://www.math.utk.edu/~morwen}.

\bibitem[Dut]{Dut}
M.Dutour, {\em .....}

%\bibitem[Weis99]{Weis}
%E.W.Weisstein, {\em CRC Concise Encyclopedia of Mathematics},
%Chapman and Hall/CRC, Boca Raton, 1999. 


 
\end{thebibliography}


\end{document}
