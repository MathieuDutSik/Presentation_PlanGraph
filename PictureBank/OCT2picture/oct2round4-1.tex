\documentclass[12pt]{article}

%\usepackage{epic,eepic}
\usepackage{amsfonts,amsmath, epsfig,latexsym, subfigure, url}
% emlines.sty  M�rz 1990, Georg Horn / Eberhard Mattes.
%
% Makros zum Zeichnen von Linien mit beliebiger Steigung.
% Nur bei Verwendung der DVI-Treiber von Eberhard Mattes.
%
% Der Makro \emline#1#2#3#4#5#6 setzt an die Koordinaten (#1,#2) den
% Punkt #3 und an die Koordinaten (#4,#5) den Punkt #6. Diese beiden 
% Punkte werden dann verbunden.
%
% TeXcad erzeugt Aufrufe dieses Makros f�r mit eingeschalteter
% EMLines-Option gezeichnete Linien.
% (z.B. \emline{0.0}{0.0}{1}{17.0}{4.0}{2}).
%
% Der Makro \newpic#1 definiert den Makro \emline so, dass den
% Punktnummern ##3 und ##6 die Ziffer #1 vorangestellt wird.
% Dies ist notwendig, wenn mehr als ein Bild auf einer Seite
% des Dokuments eingefuegt wird.
%
\def\newpic#1{%
   \def\emline##1##2##3##4##5##6{%
      \put(##1,##2){\special{em:point #1##3}}%
      \put(##4,##5){\special{em:point #1##6}}%
      \special{em:line #1##3,#1##6}}}
%
% Standarddefinition von \emline herstellen
%
\newpic{}
%
% Beispiel: 
% \input bild1.pic
% \newpic{1} \input bild2.pic
%


\setlength{\textwidth}{15cm}
\setlength{\textheight}{23cm}
\setlength{\topmargin}{-0.5cm}
\setlength{\oddsidemargin}{0.5cm}
\setlength{\evensidemargin}{0.5cm}



\newtheorem{proposition}{Proposition}
\newtheorem{theorem}{Theorem}
\newtheorem{conjecture}{Conjecture}
\newtheorem{corollary}{Corollary}
\newtheorem{lemma}{Lemma}
\newtheorem{claim}{Claim}
\newtheorem{remark}{Remark}
\newcommand{\proof}{\noindent{\bf Proof.}\ \ }

\begin{document}

\title{$4$-valent plane graphs with $2$-, $3$- and $4$-gonal faces}


\author{Michel DEZA \\
  CNRS/ENS, Paris and Institute of Statistical Mathematics, 
Tokyo,\\
\ Mathieu DUTOUR \\
 ENS, Paris and Hebrew University, Jerusalem
\footnote{!!research of the second author was!! financed by EC's IHRP Programme, within the Research Training Network ``Algebraic Combinatorics in Europe,'' grant HPRN-CT-2001-00272.}\\
\ and  Mikhail SHTOGRIN \thanks{third author acknowledges financial support 
of the Russian Foundation of Fundamental Research (grant 02-01-00803)
and the Russian Foundation for Scientific Schools (grant 00-15-96011)}\\
Steklov Mathematical Institute, Moscow, Russia} 
\date{\today}

\maketitle



\begin{abstract}
Call {\em $i$-hedrite $oc_n$} any $4$-valent $n$-vertex plane graph, whose 
faces are $2$-, $3$- and $4$-gons only and $p_2=8-i$. The edges of an $i$-hedrite, as of 
any Eulerian plane graph, are partitioned
by its {\em central circuits}, i.e. those, which are obtained by starting with an
edge and continuing at each vertex by the edge opposite the entering one. 
So, any $i$-hedrite is a projection of an alternating link, whose components
correspond to its central circuits.

Call an $i$-hedrite {\em irreducible}, if it has no 
{\em rail-road}, i.e. a 
circuit of $4$-gonal faces, in which every $4$-gon is adjacent to two of its 
neighbors on opposite edges.

We present the list of all $i$-hedrites with at most $14$ vertices. Examples of results: 

(i) Any irreducible $i$-hedrite has at most $i-2$ central circuits.

(ii) All irreducible $i$-hedrites without self-intersecting central circuits are listed.

(iii) All symmetry group of $i$-hedrites are listed.

\end{abstract}

{\em Mathematics Subject Classification}. Primary 52B05, 52B10;
Secondary 05C30, 05C10.

{\em Key words}. Plane graphs, Eulerian graphs, alternating links, point groups.

\section{Introduction and main notions}

See \cite{Gr} for terms used here for plane graphs.
It is well-known that the p-vector of any $4$-valent plane graph satisfies to
$2p_2+p_3=8+ \sum_{i\geq 5} (i-4)p_i$.
Some examples of applications of plane $4$-valent graphs are {\em projections
of links}, {\em rectilinear embedding} in VLSI and {\em Gauss crossing 
problem} for plane graphs, see for example \cite{Liu}

\vspace{2mm}


Call an {\em $i$-hedrite} any plane $2$-connected
$4$-valent graph, such that the number
$p_j$ of its $j$-gonal faces is zero for any $j$, different from 
$3,4$ and $2$, and such that $p_2=8-i$. So, 
an $n$-vertex $i$-hedrite has $(p_2, p_3, p_4)=(8-i, 2i-8, n+2-i)$.
Clearly, $(i;p_2,p_3)=(8;0,8)$, $(7;1,6)$, $(6;2,4)$,
$(5;3,2)$ and $(4;4,0)$ are all possibilities. 

An $8$-hedrite is called {\em octahedrite} in \cite{DSt}; in fact, this paper is a follow-up of \cite{DSt}.
In a way, this paper continue the program of Kirkman (\cite{Kirk} p.~282) of classification of projections of alternating knots.


















\section{Central circuits partition}
%Clearly, such graph has no cut-edges and
%we can, without loss of generality, suppose the absence of 
%cut-vertices, i.e. that $G$ is $2$-connected.

In this Section, we consider a connected plane graph $G$ with all vertices of 
even degree, i.e. an Eulerian graph. 
Call a circuit in $G$ {\it central} if it is obtained by starting with an
edge and continuing at each vertex by the edge opposite the entering one; such 
circuit is called also {\em traverse} 
(\cite{GK}), {\em straight ahead} (\cite{Ha}),  \cite{PTZ}), 
{\em straight Eulerian} (Chapter 17 of \cite{God}), 
{\em cut-through} (\cite{Je}),
{\em intersecting} etc. Clearly, the edge-set of 
$G$ is partitioned by all its central circuits.

%Such CC-partition can be considered (see, for example, \cite{Ha}) for any 
%drawing on the plane of any Eulerian (in general, not planar) graph, so that 
%edges are mapped into simple curves with at most one crossing point. The path
%called {\it straight ahead} in \cite{Ha} if, going through a vertex of the
%drawing, it leaves to its right and to its left the same number of edges.
%For example, the drawing of $K_2^4=C_4 \times C_4$ 
%($4$-hypercube=$4 \times 4$-torus) is CC-partitioned by two 
%$16$-circuits, which are, moreover, Hamiltonian circuits. 


Denote by 
$CC(G):=(...,a_i^{\alpha_i},...;...,b_j^{\beta_j},...)$ its {\it CC-vector}, 
where $...,a_i,...$ and  $...,b_j,...$ are increasing sequences of lengths of 
all its central circuits, simple ones and self-intersecting, respectively, 
and $\alpha_i, \beta_j$ are their respective multiplicities.
Clearly, $\sum_{i} a_i{\alpha_i}+ \sum_{j} b_j{\beta_j}=2n$, where 
$n$ is the number of vertices of $G$; denote by $r$ the number of 
central-circuits.


For a central circuit $C$, denote by $Int(C):=(c_0;...,c_k^{\gamma_k},...)$,
the {\em intersection vector of} $C$, where $c_0$ is
the number of self-intersections of the circuit $C$ and $(...,c_k,...)$ is
decreasing sequence of sizes of its intersection with other $r-1$ 
circuits, while the numbers $\gamma_k$ are respective multiplicities.


Two central circuits intersect in an even number number of vertices. 
The length of a central circuit is twice the number of its points of 
self-intersection plus the sum of its intersections with other circuits, 
so the length of a central circuit is even.



We will say that an $i$-hedrite is {\it pure} if any of its central circuits 
simple, i.e. has no self-intersections.
Easy to check that any pure $i$-hedrite has an even number $n$ of 
vertices. In fact, any vertex in this case belong to the intersection 
of exactly two central circuits.


Call an Eulerian graph $G$ {\it balanced}, if all its central circuits of
same length have the same intersection vector.
Any $8$-hedrite with $n \le 21$ is balanced, but there is unbalanced $22$-vertex $8$-hedrite. We do not find unbalanced $5$-hedrite or 
$7$-hedrite for $n\leq 15$. The first unbalanced $6$-hedrite 
has $12$ vertices !!indicate his number!!. Any $4$-hedrite is 
balanced (Theorem \ref{Theorem-for-4-hedrite}).









For a plane graph $G$, denote by $G^*$ its planar dual and {\em $Med(G)$} 
denotes its {\em medial} graph, i.e. the vertices of $Med(G)$ are the edges of 
$G$, two of them being adjacent if the corresponding edges share a vertex and 
belong to the same face of the embedding of $G$ in the plane. 
So, $Med(G)$=$Med(G^*)$.

Clearly, $Med(G)$ is a $4$-valent plane graph and, for any $i$-hedrite $G$, $Med(G)$ is an $i$-hedrite with twice the number of vertices of $G$ and all $2$-, $3$-gonal faces being isolated. The operation of taking the medial is a particular case of !!the {\em Goldberg-Coxeter construction}!! for the parameter $z=1+i$ (\cite{Gold37}, \cite{Cox71}, \cite{DD}).





\subsection{Intersection of central circuits}

The following Theorem is a local version (for ``parts'' of the sphere) of
the Euler formula
$2p_2+p_3=8+ \sum_{i\geq 5} (i-4)p_i$ for $p$-vector of any $4$-valent 
plane $3$-connected graph $P$.

Let $A$ be a patch of $P$ bounded by $t$ arcs (paths of edges) belonging to 
central circuits (different or coinciding). So, $A$ can be seen as a 
$t$-gon; we admit also $0$-gonal $A$, i.e.
just the interior of a simple central circuit. Suppose that the patch $A$ is
{\em regular}, i.e.
the continuation of any of bounding arc (on the central circuit to which it
belongs) lies outside of the patch; in other words, the patch
contains, around each its vertex, only one (not three) amongst four angles, 
obtained when two central circuits intersect in this vertex.  See below two 
examples of patch.

\begin{center}
\epsfxsize=60mm
\epsfbox{patch.eps}
\end{center}

Let $p'(A):=p'_1,...$ be the $p$-vector enumerating the faces of the patch $A$. The curvature of a patch is denoted by $c(A)=4-t$.

\begin{theorem} \label{Local-Euler-Formula}(\cite{DSt})
Let $A$ be a regular patch, described above, of a $4$-valent plane
$3$-connected graph. Then we have $4-t=\sum_{i\geq 1} (4-i)p'_i$
\end{theorem}



\begin{proposition}
(i) Any central circuit of a $4$-hedrite has no self-intersection vertices;

(ii) At least one central-circuit of a $7$-hedrite self-intersects.
\end{proposition}
\proof In fact, if a central circuit of a $4$-hedrite self-intersects, then we have a $1$-gonal regular patch. The equality of above Theorem becomes $2p'_2+p'_3=3$, an impossibility since $p'_3\leq p_3=0$.

Take a central circuit containing an edge of the unique $2$-gon, the sequence (possibly empty) of adjacent $4$-gons will necesseraly finish by a $3$-gon or this $2$-gon; both cases yield a self-intersection.

%\begin{center}
%\epsfxsize=60mm
%\epsfbox{7hedrite-SelfIntersect.eps}
%\end{center}





The Euler formula for the p-vector of a $4$-valent plane graph:
$$8=\sum_{t\geq 1} (4-t)p_t$$
is a discrete analog of the Gauss-Bonnet formula $2\pi(1-g)=\int_{S} K(x)dx$ for the Gaussian curvature $K$ of a surface $S$ of genus $g$. So, the $t$-gons can be seen as respectively positively curved, flat, or negatively curved, if $t<4$, $t=4$, or $t>4$.


%Given an $i$-hedrite $G$; it has $p_2=8-i$, $p_3=2i-8$. 
Let us call {\em graph of curvatures} of an $i$-hedrite $G$, the
graph (possibly, with loops and multiple edges) having as vertex-set
all $2$-gons and $3$-gons of $G$. 
Two vertices (say, faces, $2$- or $3$-gonal faces $b$ and $c$ of $G$) of 
this $i$-vertex graph are connected by edge if there is a pseudo-road 
connecting them. A {\em pseudo-road} is sequence of $4$-gons 
,say, $a_1$, \dots, $a_l$, such that putting $a_0=b$ and $a_{l+1}=c$, 
we have that any $a_k$ with $1\leq k\leq l$ is adjacent to $a_{k-1}$ 
and $a_{k+1}$ on opposite edges. Clearly, in the graph of curvatures, 
the vertices corresponding to $2$- and $3$-gons have degree $2$ and $3$, 
respectively.







\begin{proposition}\label{intersec}
Let $C_1$, $C_2$ be any two central circuits of an $i$-hedrite. Then 
they are disjoint if and only if they are simple and there exist a 
ring of $4$-gons separating them.

\end{proposition}
\proof In fact, if both $C_1$ and $C_2$ are simple circuits, Theorem 
is evident: the curvature of the interior of a patch is $4$ and so, two
circuits are separated by $4$-gons only. Suppose that $C_1$ is 
self-intersecting. Then it has at least three regular patches and each 
of them has curvature at most $3$.
The circuit $C_2$, being disjoint with $C_1$, lies entirely inside one 
of those patches, say, $A$. So, all its $3$-gons and $2$-gons, except, 
possibly, those from its exterior patch, lie in $A$. So,$c(A)\geq 5$,
since the exterior patch of $C_2$ has curvature at most $3$.

It contradicts to the fact that $A$ has curvature at most $3$. 
%So, $C_2$ intersects with $C_1$ and Theorem \ref{intersec} is proved.








\section{Adding and removal central circuits}


%It is the main operation, considered in this paper. First, we will introduce 
%associated notions; they are valid, in fact, for any $4$-valent plane graph. 
%
%
%Fix an $i$-hedrite $G$. 
%
%For a fixed central circuit $C$ consider its {\em shores}, i.e. two circuits
%of faces, lying on the left and on the right of $C$. Clearly, each shore is a 
%zone; call such zones {\em shore-zones}.
%
%
%Given a zone $Z=(F_1,...,F_m)$ of faces $F_i$, let us denote by $e_1,...,e_m$
%the edges of adjacency of pairs $(F_1,F_2),(F_2,F_3),...,(F_{m-1},F_m),(F_m,F_1)$, respectively. Call a {\em cutting of zone Z} the octahedrite obtained from
%the given one by adding $m$ new vertices $v_1,...,v_m$, one in the mid-point of
%each corresponding edge $e_i$ ($1 \le i \le m$) and $m$ new edges
%$(v_1,v_2),(v_2,v_3),...,(v_{m-1},v_m),(v_m,v_1)$. This new octahedrite 
%will have one new central circuit $V=(v_1,...,v_m)$.


The {\em deleting of a central circuit $C$} in an $i$-hedrite $G$ consists 
of removal!! all edges and vertices contained in $C$. It produces a 
$4$-valent plane graph $P'$ having only $k$-gonal faces 
with $k \leq 4$.!! But since cases $k=0,1$ are possible, we do not
always obtain an $i$-hedrite.


The {\em cutting} of an $i$-hedrite $G$ consists of adding another central 
circuit to it. The faces of the new $i'$-hedrite $G'$ with $8\geq i'\geq i$ 
comes from the cutting of faces of $G$. This operation is only partially
defined since arbitrary cutting can produce $k$-gons with $k>4$. The 
cutting of a $4$-gon in several $4$-gons (two if the face
is traversed only once) is possible only if the $4$-gon is traversed 
on opposite edges. This corresponds to the notion of {\it shore-zone} 
in \cite{DSt}.!!!
A cutting changes CC-partition of an $i$-hedrite only in the following 
way: new central circuit $C$ is added and all others central circuits 
remain unchanged except that the length of each of them increases by one 
for any intersection with $C$. 


Call {\em rail-road} a circuit of $4$-gons (possibly self intersecting), 
in which every $4$-gon is adjacent to two of its neighbors on opposite
edges. A rail-road is bounded by two ``parallel'' central circuits.
The deleting of  one of those central circuits (in other word, collapse 
rail-road into one central circuit) is called {\em reducing}.
The cutting produces a rail-road if and only if it is {\em inflation 
along a central circuit $C$}, i.e. remplacing of it by thin enough 
rail-road. A!! {\em $t$-inflation along a central circuit $C$} is remplacing
this central circuit by $t-1$ parallel (thin enough) rail-roads.
A!! {\em $t$-inflation of an $i$-hedrite} is new $i$-hedrite
obtained from original one by simultaneous $t$-inflation along all
of its central circuits.!!! 
$t$-inflation of $G$ is $G$ if $t=1$, and it is just inflation of $G$ 
if $t=2$.

%Let $G'$ be the result of cutting of an $i$-hedrite $G$. Then $G'$ is an 
%$i'$-hedrite with $i\leq i'\leq 8$; but $i'=i$ if the cutting is an 
%inflation along a central circuit.

An $i$-hedrite is called {\em irreducible} if it contains no 
rail-road. It is called {\em maximal irreducible} if it cannot be
obtained from another $i'$-hedrite by a cutting.

% One can call $Med(G)$ {\em $\frac{3}{2}$-inflation} of $G$, 
% since $Med(Med(G))$ is $2$-inflation of $G$.

\begin{remark}

Let $C$ be a central circuit of $G$ with $CC(G)=(...,a_i^{\alpha_i},...;...,b_j^{\beta_j},...)$,  and let 
$Int(C)=(c_0;c_1^{\gamma_1},...,c_r^{\gamma_r})$. Let $C'$ be one of 
$t$ parallel copies, which are put instead of the circuit $C$ in 
$G^t$!!, where $G^t$ denotes the $t$-inflation of $G$ along $C$!!.
Then $CC(G^t)=(...,ta_i^{t\alpha_i},...;...,tb_j^{t\beta_j},...)$,
$Int(C')=(c_0;c_1^{t\gamma_1},...,c_r^{t\gamma_r}, (2c_0)^{t-1})$.
\end{remark}


\section{Some special $i$-hedrites}
For any integer $m \ge 2!!$ denote:

by $I_{6,2m}$ the $2m$-vertex $6$-hedrite, such that each $2$-gon is adjacent to two $3$-gons;

by $I_{5,2m+1}$ the $(2m+1)$-vertex $5$-hedrite, such that 
two $2$-gons share a vertex and remaining $2$-gon is adjacent
to two $3$-gons;

by $I_{4,2m+2}$ the $(2m+2)$-vertex $4$-hedrite, such that 
four $2$-gons are organised into two pairs sharing a vertex (they correspond to shift $i=1$!! in Theorem \ref{Theorem-for-4-hedrite} below).

by $J_{4,2m}$ the $m$-inflation of only one central circuit of
$4$-hedrite {\bf 2-1}; they are projections of {\em composite}
alternating links $2^2_1\#2^2_1\#2^2_1\dots\#2^2_1$ ($m$ times),
which we denote by $m\times 2^2_1$.

See on Figure \ref{fig:FamilyIin} the first occurrences
(for $2!!\leq m\leq 5$) of those graphs and on Table \ref{FundamentalInfo}
their symmetry groups and CC-vectors.

\begin{figure}
\centering
\epsfxsize=100mm
\epsfbox{FamilyIin.eps}
\label{fig:FamilyIin}
\end{figure}
!!J_4,2m is 3-connected for m=1; start them from m=2.!!
!!Also, perhaps, put m=1,...,m=5 only once (up or down)?!!


\begin{table}
\begin{center}
\begin{tabular}{||c|c|c||}
\hline\hline
$i$-hedrite            &Group      &CC-vector\\\hline\hline
$I_{6,2m}$, $m$ even   &$D_{2d}$   &$4m$\\\hline
$I_{6,2m}$, $m$ odd    &$D_{2h}$   &$(2m)^2$\\\hline
$I_{5,2m+1}$           &$C_{2v}$   &$4m+2$\\\hline
$I_{4,2m+2}$, $m$ even &$D_{2d}$   &$(2m+2)^2$\\\hline
$I_{4,2m+2}$, $m$ od   &$D_{2h}$   &$(2m+2)^2$\\\hline
$J!!_{4,2m}$              &$D_{2h}$   &$2^m, 2m$\\\hline\hline
\end{tabular}
\end{center}
\caption{$2$-connected $i$-hedites}
\label{FundamentalInfo}
\end{table}



\begin{lemma}
Any $i$-hedrite is $2$-connected.
\end{lemma}
\proof Let $G$ be an $i$-hedrite and assume that there is one vertex $v$ such that $G-\{v\}$ is disconnected in two components $C_1$, $C_2$. Then two edges from $v$ will connect to a vertex $w$ of $C_1$ and two edges from $v$ will connect to a vertex $w'$ of $C_2$, because, otherwise, the exterior face is $m$-gonal with $m>4$. See below the corresponding drawing.

\begin{center}
\epsfxsize=60mm
\epsfbox{2-connected.eps}
\end{center}

But the vertex $w$ will disconnect the graph and so, iterating the construction, we obtain an infinite sequence $v_1$, \dots, $v_n$ of vertices that disconnects $G$. 
This is impossible since $G$ is finite.
%This contradicts to initial assumption and proves that $G$ is $2$-connected.


Any $i$-hedrite (moreover, any Eulerian graph) has at least one 
Eulerian circuit of edges; so, there is no cut-edge.
But a cut-vertex appears already for some Eulerian ($3$-vertex)
and $4$-valent ($4$-vertex) plane graphs.


\begin{theorem}\label{3-connectedness}
Any $i$-hedrite, which is not $3$-connected, is one of $I_{6,2m}$, $I_{5, 2m+1}$, $I_{4, 2m+2}$, $J_{4, 2m}$ for some $m\geq 2$.


\end{theorem}

\proof Let $G$ be an $i$-hedrite and assume that it is not $3$-connected. 
Then there are two vertices, say, $v$ and $v'$, such that 
$G-\{v, v'\}$ is disconnected in two components, say, $C_1$ and $C_2$.
Amongst the $4$ edges from $v$ (respectively $v'$), the edges 
$\{e_1,\dots, e_{t}\}$ (respectively $\{e'_1,\dots, e'_{t'}\}$) goes
to $C_1$. Two numbers $t$ and $t'$ can takes values $1$, $2$ or $3$; we
will consider all possible cases.

Assume that $t=1$ and $t'=1$, then the edges $e$ and $e'$ must be
distinct, since otherwise $C_1$ is emptygraph. Moreover, $e$ and $e'$ have
no common vertex since, otherwise, $G$ would not be $2$-connected.
So, $v$ and $v'$ are connected by $e$ and $e'$ to a vertex $w$ and $w'$,
respectively. Since face size is at most $4$, the vertices $v$ and $v'$,
(respectively, $w$ and $w'$) are linked by two edges
(see Figure \ref{fig:TheThreeCases}).

Two points $w$ and $w'$ can either be connected by two edges and
we are done, or disconnect the graph. In the latter case, we can iterate
the construction. Since the graph is finite, the construction eventually
finish and we get a graph $J_{4,t}$. If $t=1$ and
$t'=3$, then by a similar reasoning, one gets again a graph 
$J_{4,2m!!}$.

Assume that $t=2$ and $t'=2$. One has $\{e_1, e_2\}\cap \{e'_1, e'_2\}=\emptyset$, since, otherwise, one can attribute an edge to $C_2$ and get the case
$t=1$ and $t'=1$. So, one has, say, $e_1\cap e'_1=\{w_1\}$ and 
$e_1\cap e'_1=\{w_2\}$ and the following two possibilities 
(see Figure \ref{fig:TheThreeCases}): either $w_1=w_2$ (this
corresponds to $\{e_1, e_2\}$ and $\{e'_1, e'_2\}$ being the edges
of two $2$-gons) and we are done, or $w_1\not= w_2$.

Assume now that $w_1\not= w_2$; two points $w_1$ and $w_2$ can 
either be connected by two edges and we are done, or disconnect the
graph. In the latter case, we can iterate the construction. Since the 
graph is finite, the construction eventually finish. If we do the same
construction on the other side, then we get a similar structure and the 
graph is of the form $I_{4,2m+2}$, $I_{5, 2m+1}$ or $I_{6,2m}$ with 
$m\geq 2$.


Assume now that $t=2$ and $t'=1$. The edges $e_1$, $e_2$ and $e'_1$ are
all distinct, since, otherwise, the vertex $v$ disconnects the graph.
So, $v'$ is connected by $e'_1$ to a vertex $w'$. Now either $v'$ or $w'$
is connected to $v$ since, otherwise, we would have a $5$-gonal face.
If $w'$ is connected to $v$, then the pair $\{w', v\}$ disconnects the graph.
This construction is infinite (see Figure \ref{fig:TheThreeCases}), so,
we get a contradiction.




\begin{figure}
\centering
\mbox{
\subfigure[The case $t=1$, $t'=1$]{\epsfig{figure=Case3-3.eps,width=.30\textwidth}}
\subfigure[The case!! $t=2$, $t'=2$]{\epsfig{figure=Case2-2.eps,width=.20\textwidth}}
\subfigure[The case $t=2$, $t'=1$]{\epsfig{figure=InfiniteStructure.eps,width=.30\textwidth}}
}
\caption{The three cases of Theorem \ref{3-connectedness}}
\label{fig:TheThreeCases}
\end{figure}



\begin{theorem}
(i) If $G$ is an $i$-hedrite with two adjacent $2$-gons, then 
this is $4$-hedrite {\bf 2-1} or!! a $J_{4,2m}$ with $m\geq 2$.

(ii) If $G$ is an $i$-hedrite with two $2$-gons!! sharing a 
vertex, then it is either $4$-hedrite {\bf 4-1}, or!! an $I_{4,2m+2}$,
 or $5$-hedrite {\bf 3-1},!! or an $I_{5,2m+1}$ with $m\geq 2$.

\end{theorem}

\proof The proof for (i), (ii) is similar to the cases $(t,t')$=$(1,1)$, $(2,2)$ of Theorem \ref{3-connectedness}.




\begin{theorem}

(i) $4$-hedrites exist if and only if $n\geq 2$, even.

(ii) $5$-hedrites exist if and only if $n\geq 3$, $n\not= 4$.

(iii) $6$-hedrites exist if and only if $n\geq 4$.

(iv) $7$-hedrites exist if and only if $n\geq 7$.

(v) $8$-hedrites exist if and only if $n\geq 6$, $n\not= 7$.

\end{theorem}
\proof The case (v) is proven in \cite{Gr}, page 282.
!!The case (i) is trivial; take, for example, the serie 
$J_{4,2m}$.!!
The series $I_{6,2m}$, $I_{5,2m+1}$ for any $m \ge 2$, give $6$-hedrites
and $5$-hedrites with even and, respectively, odd number of vertices.

For $5$-, $6$- and $7$-hedrites, we can by $(t-1)$-inflation 
along a!!
central circuit of length $4$ in corresponding $i$-hedrite 
{\bf 6-1},
{\bf 5-1} or {\bf 7-1}!!, get series with $4t+2$, $4t+1$ and 
$4t+3$ vertices. By $(t-2)$-inflation
along such central circuit in $7$-hedrite {\bf 8-1}, we get a serie of
$7$-hedrite with $4t$ vertices.
By $(t-1)$-inflation along central circuit of length $6$ in 
$5$-hedrite {\bf 10-2}, we get serie of $5$-hedrites with $6t+4$ vertices.
By $(t-1)$-inflation along central circuit in $6$-hedrite {\bf 11-4}, we
get serie of $6$-hedrites with $8t+3$ vertices.

!!next 3 lines are not needed, since remaining for 7-hedrites 
case 4t+1 also OK!!
%Now, by inscribing concecutively $3$-gons in the $3$-gon, which is adjacent
%only to $3$-gons, in $7$-hedrites {\bf 7-1} and {\bf 8-1}, we get series
%of $7$-hedrites with $3t+1$ and $3t+2$ vertices.
Inscribing concecutively $4$-gons in the $4$-gon, which is adjacent only to
$3$-gons, in $7$-hedrites!! {\bf 9-1} and!! {\bf 10-2}, we get 
series!! of $7$-hedrites
with $4t+1$ and!! $4t+2$ vertices.

!!But, I think, for 5-hedrites the proof from my E-mail is needed!!
!!For 6-hedrites the case 8t+7 still I not see how prove!!



Our computation presented all $i$-hedrites with at most $14$ vertices. Easy to see NEED TO COMPLETE
Suppose that an $i$-hedrite has a $t$-gonal face $F$ adjacent to $t$ $3$-gons. Then, one can add to every edge $e_i$ of $F$ a vertex $v_i$ and form another $t$-gon $v_1\dots v_t$. The obtained graph is an $i$-hedrite with $n+t$ vertices.




\section{Irreducible $i$-hedrites}



\begin{theorem}\label{irre}
Any irreducible $i$-hedrite has at most $i-2$ central circuits and equality is attained for each $i$, $4\leq i\leq 8$.
\end{theorem}
\proof For $i=8$, the theorem is proved in \cite{DSt}. We will show, using suitable cutting, that this result implies the Theorem for $i<8$.

Let us start with the simplest case of a $7$-hedrite. Consider a simple circuit $S$ in its curvature graph, which contains the vertex corresponding to unique $2$-gon. Remind that the vertices in the curvature graph correspond to $2$- or $3$-gons, while edges correspond to pseudo-roads !!never defined
in oct2!!. So, the simple circuit $S$ corresponds to the circuit of faces of $G$, containing our $2$-gon, some $4$-gons and, possibly, some of six $3$-gons; see the picture below.


\begin{center}
\epsfxsize=40mm
\epsfbox{CC_FowlerGraph.eps}
\end{center}






Suppose that the $7$-hedrite $G$ is irreducible and has $k$ central circuits. By adding the central circuit $C$, which is shown by dotted lines on picture above, we produce an $8$-hedrite (since, the $2$-gon is cut by $C$ in two $3$-gons), which is still irreducible and has $k+1$ central circuits. So, $k+1\leq 6$, by Theorem 3 of \cite{DSt}.


For remaining cases of $i$-hedrites with $i=4,5,6$, the proof is similar. In each case, we consider all possible distribution of $2$-gons by simple circuits in the graph of curvatures and for each such circuit, to add suitable number of new central circuits. 


All possibilities are presented on picture below: two for $i=6$, three for $i=5$ and two for $i=4$. In the last case, there are no $3$-gons and so, simple circuits in the curvature graph contain only even number of $2$-gons by local Euler formula (\ref{Local-Euler-Formula}). Note that the case of
$4$-hedrites is obvious by Theorem 5 of \cite{DSt}.


%\begin{center}
%\epsfxsize=60mm
%\epsfbox{PossibilitiesTwoToFour.eps}
%\end{center}

\begin{figure}
\centering
\mbox{\subfigure[The two cases for $6$-hedrites]{\epsfig{figure=PossibilitiesTwo.eps,width=.25\textwidth}}\quad
\subfigure[The three cases for $5$-hedrites]{\epsfig{figure=PossibilitiesThree.eps,width=.25\textwidth}}\quad
\subfigure[The two cases for $4$-hedrites]{\epsfig{figure=PossibilitiesFour.eps,width=.25\textwidth}}}
\caption{The construction of Theorem \ref{irre} for $6$-, $5$-, $4$-hedrites}
\label{The456hedriteCases}
\end{figure}




\begin{lemma}
Let $G$ be an irreducible $8$-hedrite and $C$ a central circuit, assume that $C$ is adjacent to three $3$-gons on one side. Then one can add another central circuit to $G$ while staying irreducible.
\end{lemma}
\proof From every one of the three $3$-gons, say $T_1$, $T_2$, $T_3$, one can define a pseudo-road from the side of the central $3$-gon that belongs to the central circuit. See picture below.
\begin{center}
\epsfxsize=40mm
\epsfbox{ThrreTrianglesCentral.eps}
\end{center}
Each such pseudo-road define an edge, say $e_1$, $e_2$, $e_3$ in the graph 
of curvatures. We have two possibilities: either a path exist from $T_i$ 
to $T_j$ that use the edges $e_i$, $e_j$ or such a path does not exist. In 
first case one can add a central circuit which cut all $3$-gons $T_1$, 
$T_2$, $T_3$ in the form depicted below:
\begin{center}
\epsfxsize=40mm
\epsfbox{CuttingAllTriangles.eps}
\end{center}

Assume now that there is no path between $T_i$ and $T_j$. Then, since the graph of curvatures is $3$-valent, the graph NEED TO COMPLETE
!!!!!












See below example of an irreducible $7$-hedrite (its CC-vector is $(10^2, 12^2; 20)$ with the maximum number of central circuits.


\begin{center}
\epsfxsize=60mm
\epsfbox{7hedrite32Vertex5CC.eps}
\end{center}

For all other $i$, there is an example of pure irreducible $i$-hedrite with $i-2$ central circuits (see 
Theorem \ref{TheOneWithSimpleCentralCircuit}), which are,!! moreover, pure.















\begin{conjecture}
An irreducible $i$-hedrite is maximal irreducible if and only if it has $i-2$ central circuits.
\end{conjecture}
%\proof NEED TO FIND.



\section{Classification of pure irreducible $i$-hedrites}
The easiest case, $i=4$, of i-hedrites admits following complete
characterization:

\begin{theorem}\label{Theorem-for-4-hedrite}
(i) Any $4$-hedrite can be obtained from some $4$-hedrite with two central
circuits by simultanous $t_1$- and $t_2$-inflation along those circuits; it is
irreducible if and only if $t_1=t_2=1$.

(ii) Any $4$-hedrite with two central circuits is defined by
its number of vertices $n$ and by {\em shift} $j$, $0 \le j \le n/4$,
vertices between the pair of boundary $2$-gons on the horizontal
circuit (see, for example, $4$-hedrite {\bf 8-1}) and the remaining
pair of $2$-gons.
!!While above is true, some shift defines 4-hedrite with more than
2 c.circuits!!

(iii) Any $4$-hedrite is balanced.!!

\end{theorem}
\proof (i) and (ii) are proved in \cite{DSt}, while (iii) balancedness
is obvious for $4$-hedrite with two central circuits and remain true
under $t_1$- and $t_2$-inflation.

The case $j=0$ corresponds to {\bf 2-1} and its one-circuits inflations
$J_{4,n=2m}$. The case $j=1$ correspond to {\bf 4-1} and $I_{4,n=2m+2}$.
Denote by $K_{4,4m}$, for any $m \ge 2$, any $4m$-vertex $4$-hedrite obtained
from {\bf 4-1} by m-inflation of only one its central circuit; so, its 
CC-vector is $(4^m,4m)$, its symmetry is $D_{2d}$ and it is reducible.
Clearly, any $K_{4,n=4m}$ has shift $j=n/4$.




%\begin{theorem}\label{Theorem-for-4-hedrite}
%
%Let $G$ be an irreducible $4$-hedrite, $G$ is defined
%by the shift by $i, i=0,1,2,..., [\frac{m}{2}]$ 
%vertices between the pair of boundary $2$-gons of the
%horizontal circuit and the remaining pair of $2$-gons.
%\end{theorem}










\begin{theorem}\label{TheOneWithSimpleCentralCircuit}

Any pure irreducible $i$-hedrite is, either any $4$-hedrite with two
central circuits, or a $5$-hedrite {\bf 6-2}, or one of $6$-hedrites
{\bf 8-6}, {\bf 14-20}, or one of the following eight $8$-hedrites:
{\bf 6-1}, {\bf 12-4}, {\bf 12-5}, {\bf 14-7}, and (see Figure 
\ref{ThePureIrreducibleOctahedriteWith56CC}) {\bf 20-1}, {\bf 22-1},
{\bf 30-1}, {\bf 32-1}.


\end{theorem}


\proof Let $G$ be a pure irreducible $i$-hedrite having $r$ central circuits. 
If one deletes a central circuit, then, in general, $1$-gon can appear. It 
does not happen for $G$, since it will imply a self-intersection of a central 
circuit. Clearly, the result of deletion of a central circuit from $G$ 
produces an pure irreducible $i$-hedrite with $r-1$ central circuits.

First, if $r=2$,!! then the Theorem 5 from \cite{DSt} gives that such $G$ are 
exactly $4$-hedrites with two central circuits; all of them are classified 
in Theorem \ref{Theorem-for-4-hedrite}.

We prove the Theorem by systematical analysis of all possible ways to add 
to $G$ (for $r=2,3,4,5$) a central circuit, in order to get a pure 
irreducible $i$-hedrite with $r+1$ central circuits. 


(i) Let $r=2$. Then $G$ can be only one of two smallest $4$-hedrites. In 
fact, if $G$ is another $4$-hedrite,!! then because of classification Theorem 
\ref{Theorem-for-4-hedrite},!! it has a form as in Figure \ref{Cutting4hedrite}.

New central circuit should cut both $2$-gons on opposite edges, since otherwise there is a rail-road. But Figure \ref{Cutting4hedrite} shows, on example for $n=6$, that a self-intersection appears if the two central circuits interesect in more than four vertices of intersection.


\begin{figure}
\centering
\epsfxsize=40mm
\epsfbox{SelfIntersectionForSix.eps}
\caption{No pure irreducible $i$-hedrite comes by!! cutting of above one}
\label{Cutting4hedrite}
\end{figure}



So, the only possible $4$-hedrites with two central circuits, which can
be cut in order to produce irreducible pure $i$-hedrite
are $4$-hedrites {\bf 2-1} and {\bf 4-1}. All cases are indicated below. 

\begin{center}
\epsfxsize=120mm
\epsfbox{PurityThreeCentral.eps}
\end{center}

So, all irreducible pure $i$-hedrites with three central circuits are $5$-hedrite {\bf 6-2}, $6$-hedrite {\bf 8-5}, $8$-hedrite {\bf 6-1} (i.e. the projections of links $6^3_1$, $8^3_6$, $6^3_2$).
Now we apply the same procedure to those three $i$-hedrites; see Figure below:

\begin{center}
\epsfxsize=120mm
\epsfbox{PurityFourCentral.eps}
\end{center}

So, all irreducible pure $i$-hedrites with four central circuits are $8$-hedrites {\bf 12-4}, {\bf 12-5}, {\bf 14-7} and $6$-hedrite {\bf 14-20}.

By the same method, one can see that there are exactly two pure irreducible $i$-hedrites with five central circuits and two with six central circuits (see Figure \ref{ThePureIrreducibleOctahedriteWith56CC}).

\begin{figure}
{\small
\setlength{\unitlength}{1cm}
\begin{minipage}[t]{3.5cm}
\begin{picture}(3.5,3.5)
\leavevmode
\epsfxsize=3.5cm
\epsffile{4reg_21_114.ps}
\end{picture}\par
\begin{center}
{{\bf Nr.20-1} \quad $D_{2d}$ \\ $(8^5)$ \\ }
\end{center}
\end{minipage}
% \setlength{\unitlength}{1cm}
\begin{minipage}[t]{3.5cm}
\begin{picture}(3.5,3.5)
\leavevmode
\epsfxsize=2.5cm
\epsffile{PureOctahedrite22.eps}
\end{picture}
\par
\begin{center}
{{\bf Nr.22-1} \quad $D_{2h}$ \\ $(8^3,10^2)$ \\ }
\end{center}
\end{minipage}
\setlength{\unitlength}{1cm}
\begin{minipage}[t]{3.5cm}
\hfil\begin{picture}(2.3,2.3)
\leavevmode
\epsfxsize=2.3cm
\epsffile{oc30-1.ps}
\end{picture}\hfil\par
\begin{center}
{{\bf Nr.30-1} \quad $O$ \\ $(10^6)$ \\}
\end{center}
\end{minipage}
\setlength{\unitlength}{1cm}
\begin{minipage}[t]{3.5cm}
\begin{picture}(3.5,3.5)
\leavevmode
% \epsfxsize=3.5cm
% \epsffile{fig-32.pic}
\mbox{} \hspace{-1cm}
\input fig-32.pic
\end{picture}\par
\begin{center}
{{\bf Nr.32-1} \quad $D_{4h}$ \\ $(10^4,12^2)$ \\}
\end{center}
\end{minipage}
}
\caption{Pure irreducible $i$-hedrites with $5$ or $6$ central circuits}
\label{ThePureIrreducibleOctahedriteWith56CC}
\end{figure}


\begin{corollary}
Any pure $i$-hedrite come from a pure irreducible $i$-hedrite with $j$ central circuits by simultaneous $t_1$-, \dots, $t_j$-inflation along those circuits; it is irreducible if and only if $t_1=\dots=t_j=1$.



\end{corollary}





\section{Symmetry group of $i$-hedrites}
We consider below the maximal symmetry groups of plane graphs; these groups are identified with the corresponding point groups.



%SHTOGRIN METHOD, PLEASE KEEP
%--------------------------------
%Consider a putative $5$-hedrite $G$ with symmetry $C_{3h}$. Then its three $2$-gons form orbit of size three. The only possibility for each of those $2$-gons is to intersect the plane of symmetry. But there are two ways of such intersection: either throught vertices of $2$-gons, or throught its edges. So, there is a belt of $4$-gons on which three $2$-gons are placed on equal distance. In the first case, the $4$-gons and $2$-gons of the belt are incident to its neighbor by opposite vertices; in the second case, they are adjacent by opposite edges.
%
%INSERT DRAWING, PLEASE
%
%One can see that the only way to complete those belts while preserving the $C_3$-symmetry, produces an inflation of $5$-hedrite {\bf 3-1} (in first case), or, in the second case, of the $5$-hedrite {\bf 6-2} (the medial of {\bf 3-1}).
%
%Also (NEED TO COMPLETE), the symmetry $C_3$ implies the symmetry $D_3$, so the symmetry $C_{3v}$ implies the symmetry $D_{3h}$.
%---------------------------




\begin{theorem}
We indicate here the list of symmetry groups of $i$-hedrites 
together with the value of $n$ of their first appearance:
\begin{itemize}
\item[(i)] The only symmetry groups of $4$-hedrites are point subgroups of $D_{4h}$, which contain $D_{2}$ as point subgroup, i.e. $D_{4h}$($n=2$), $D_4$($n=10$), $D_{2h}$($n=4$), $D_{2d}$($n=6$), $D_2$($n=12$).

\item[(ii)] The only symmetry groups of $5$-hedrites are $C_1$($n=10$), $C_2$($n=8$), $C_s$($n=7$), $C_{2v}$($n=5$), $D_3$($n=15$) and $D_{3h}$($n=3$).

\item[(iii)] The only symmetry groups of $6$-hedrites are 
$D_{2d}$($n=4$)!!, $D_{2h}$($n=6$)!! and all their point 
subgroups, i.e. $D_{2}$(n=12)!!, $C_{2h}$($n=10$)!!, 
$C_{2v}$(n=5)!$$!, $C_i$($15 \le n \le 34$)!!, $C_{2}$($n=6$)!!, 
$C_{s}$($n=9$)!!, $C_{1}$($n=9$)!!.

\item[(iv)] The only symmetry groups of $7$-hedrites are point subgroups of $C_{2v}$, i.e. $C_{2v}$($n=7$), $C_{2}$($n=11$), $C_{s}$($n=8$), $C_{1}$($n=11$).

\item[(v)] The only symmetry groups of $8$-hedrites are $C_{1}$($n=16$), $C_s$($n=14$), $C_2$($n=12$), $C_{2v}$($n=11$), $C_i$($22\leq n\leq 46$), $C_{2h}$($22\leq n \leq 26$), $S_4$($22\leq n\leq 60$), $D_2$($n=10$), $D_{2d}$($n=14$), $D_{2h}$($n=22$), $D_3$($n=18$), $D_{3d}$($n=12$), $D_{3h}$($n=9$), $D_4$($n=18$), $D_{4d}$($n=8$), $D_{4h}$($n=10$), $O$($n=30$), $O_h$($n=6$).

\end{itemize}


\end{theorem}
\proof For $4$-hedrites, see \cite{DSt}. Any transformation
stabilizing a $2$-gon, can interchange its two edges and two vertices. So, the stabilizing point subgroup of a $2$-gon can be $C_{2v}$, $C_s$, $C_2$ or $C_1$ only.

A $7$-hedrite has only one $2$-gon which has to be preserved by the symmetry group; so, all possibilities are $C_{2v}$, $C_s$, $C_2$, $C_1$.

Every symmetry of an $i$-hedrite induces a symmetry on its $2$-gons and $3$-gons. Since the stabilizer of a $2$-gon, $3$-gon has maximal size $4$, $6$, this imply that the order of the symmetry group of an $i$-hedrite is bounded above by $4|Sym(8-i)|=4(8-i)!$ and $6|Sym(2i-8)|=6(2i-8)!$.

So, in particular, the order of symmetry group of an $6$-hedrite is at most $8$. If $f$ is an element of order three then it fix each of two $2$-gons. Since, the stabilizer of a $2$-gon does not contain an element of order three, we have that no such $f$ exists. If $f$ is a rotation of order $4$ then $f^2$ is a rotation of order $2$ stabilizing each $2$-gon; so, the axis of $f$ goes throught the two $2$-gons. This is a contradiction. 
By a!! search in the Tables of the groups one can see that the only possibilities are $C_1$, $C_s$, $C_2$, $C_i$, $C_{2v}$, $C_{2h}$, $D_2$, $D_{2h}$, $D_{2d}$. In fact, there exists!! a $6$-hedrite for any of such symmetries in Figure \ref{special-i-hedrites} and subsection!! 
\ref{subsection-8-hedrites} below!!.


For $5$-hedrites, since there are two $3$-gons, the maximal order of the group is $12$. The oddness of the number of $2$-gons excludes central symmetry, axis of order $4$, and groups $D_2$, $D_{2h}$, $D_{2d}$.

If $G$ is a $5$-hedrite with a $3$-fold axis then this axis goes throught the two $3$-gons, say $T_1$, $T_2$. If one consider a belt of $4$-gons around $T_1$, then after a number $p$ of steps, one will encounter a $2$-gon and so, by symmetry three $2$-gons. So, we will have the following possibilities:

\begin{center}
\epsfxsize=60mm
\epsfbox{PossibleNeighbor.eps}
\end{center}

There is only one way to extend this graph to a $5$-hedrite and the obtained extension has symmetry at least $D_3$. So, the group is $D_{3}$, 
$D_{3h}$ or $D_{3d}$.

An $8$-hedrite can a priori have $k$-fold axis of rotation with $k=2, 3, 4$. If $k=3$, then the axis of the rotation goes through two $3$-gons, say $T_1$, $T_8$. If one consider around $3$-gon $T_1$ a belt of $4$-gons, then after a number $p$ of steps, one will encounter a $3$-gon and so, by symmetry three $3$-gons, say $\{T_2, T_3, T_4\}$. Adding, if necessary, belts of $4$-gons, one will encounter the last three $3$-gons, say $\{T_5, T_6, T_7\}$. There is an unique way to complete the graph while preserving the $3$-fold symmetry. The symmetry of the obtained graph is!! $D_3$, $D_{3d}$ or!! $D_{3h}$. 

If $k=4$, then the axis of the rotation goes through a vertex or a $4$-gon. Assume for simplicity, that this axis goes through a vertex, then repeating above reasoning, one obtains two orbits of $3$-gons, say $\{T_1, T_2, T_3, T_4\}$ and $\{T_5, T_6, T_7, T_8\}$ under $4$-fold symmetry and the symmetry group is $D_4$, $D_{4h}$, $D_{4d}$, $O$ or $O_h$.

So, one obtains the above list of $18$ possible point groups. All these groups appear in subsection \ref{subsection-8-hedrites} (groups 
$O_h$, $D_{4d}$, $D_{3h}$, $D_2$, $D_{4h}$, $C_{2v}$, $D_{3d}$, 
$C_{2}$ !!I reordered them in order of appearence!!), in Figure \ref{special-i-hedrites} (groups $C_{i}$, $C_{2h}$, $S_4$, $D_{3}$, $C_1$, $C_s$,
$D_{2d}$,!! $D_4$), Figure \ref{ThePureIrreducibleOctahedriteWith56CC} (groups 
!! deleted D_2d!! $D_{2h}$, $O$); for all (!!except, possibly,!! $C_i$, $C_{2h}$ and $S_4$)!! those examples of $8$-hedrites have smallest number of vertices!!.


\begin{figure}
{\small
\setlength{\unitlength}{1cm}
\begin{minipage}[t]{4cm}
\begin{picture}(5,5)
\leavevmode
\epsfxsize=4cm
\epsffile{6-hedriteCi.eps}
\end{picture}\par
\begin{center}
{$6$-hedrite {\bf 34-1} $C_i$ $(8;26^2)$}
\end{center}
\end{minipage}
\setlength{\unitlength}{1cm}
\begin{minipage}[t]{4cm}
\begin{picture}(5,5)
\leavevmode
\epsfxsize=4cm
\epsffile{5-hedrite15-3sec.eps}
\end{picture}\par
\begin{center}
{$5$-hedrite {\bf 15-1} smallest $D_3$ $(30)$}
\end{center}
\end{minipage}
\setlength{\unitlength}{1cm}
\begin{minipage}[t]{4cm}
\begin{picture}(5,5)
\leavevmode
\epsfxsize=4cm
\epsffile{OctahedriteS4.eps}
\end{picture}\par
\begin{center}
{$8$-hedrite {\bf 60-1} $S_4$ $(16;26^4)$}
\end{center}
\end{minipage}
\setlength{\unitlength}{1cm}
\begin{minipage}[t]{4cm}
\begin{picture}(5,5)
\leavevmode
\epsfxsize=4cm
\epsffile{8-hedriteCi.eps}
\end{picture}\par
\begin{center}
{$8$-hedrite {\bf 46-1} $C_i$ $(10; 82)$}
\end{center}
\end{minipage}
%\setlength{\unitlength}{1cm}
%\begin{minipage}[t]{4cm}
%\begin{picture}(5,5)
%\leavevmode
%\epsfxsize=4cm
%\epsffile{OcahedriteC2h.eps}
%\end{picture}\par
%\begin{center}
%{$8$-hedrite {\bf 48-1} $C_{2h}$ $(10; 18^2, 50)$}
%\end{center}
%\end{minipage}
\setlength{\unitlength}{1cm}
\begin{minipage}[t]{4cm}
\begin{picture}(5,5)
\leavevmode
\epsfxsize=4cm
\epsffile{OcahedriteC2hOther.eps}
\end{picture}\par
\begin{center}
{$8$-hedrite {\bf 26-1} $C_{2h}$ $(8^2; 36)$}
\end{center}
\end{minipage}
\setlength{\unitlength}{1cm}
\begin{minipage}[t]{4cm}
\begin{picture}(5,5)
\leavevmode
\epsfxsize=4cm
\epsffile{ZZprojection7:3.ps}
\end{picture}\par
\begin{center}
{$8$-hedrite {\bf 22-1} smallest $D_3$ $(44)$}
\end{center}
\end{minipage}
\setlength{\unitlength}{1cm}
\begin{minipage}[t]{4cm}
\begin{picture}(5,5)
\leavevmode
\epsfxsize=4cm
\epsffile{OctahedriteSymmetryD4.eps}
\end{picture}\par
\begin{center}
{$8$-hedrite {\bf 18-1!!} smallest $D_4$ $(18^2)$}
\end{center}
\end{minipage}
\setlength{\unitlength}{1cm}
\begin{minipage}[t]{4cm}
\begin{picture}(5,5)
\leavevmode
\epsfxsize=4cm
\epsffile{OctahedriteCssec.eps}
\end{picture}\par
\begin{center}
{$8$-hedrite {\bf 14-2!!} smallest $C_s$ $(6;22)$}
\end{center}
\end{minipage}
\setlength{\unitlength}{1cm}
\begin{minipage}[t]{4cm}
\begin{picture}(5,5)
\leavevmode
\epsfxsize=4cm
\epsffile{OctahedriteC1sec.eps}
\end{picture}\par
\begin{center}
{$8$-hedrite {\bf 16-1!!} smallest $C_1$ $(6, 8; 18)$}
\end{center}
\end{minipage}
\setlength{\unitlength}{1cm}
\begin{minipage}[t]{4cm}
\begin{picture}(5,5)
\leavevmode
\epsfxsize=4cm
\epsffile{First8hedriteD2d.eps}
\end{picture}\par
\begin{center}
{$8$-hedrite {\bf 14-4!!} smallest $D_{2d}$ $(14^2)$}
\end{center}
\end{minipage}
}
\caption{Some $i$-hedrites with special symmetry groups}
\label{special-i-hedrites}
\end{figure}

!!please, put always ``smallest'' AFTER group!!

%The smallest $i$-hedrite and the smallest $8$-hedrite with symmetry $D_3$ are given in Figure \ref{fig:WithSymmetryD3} (both are knots).
%\begin{figure}
%\centering
%\mbox{\subfigure[$15$-vertices $5$-hedrite]{\epsfig{figure=5-hedrite15-3sec.eps,width=.20\textwidth}}\quad
%\subfigure[$18$-vertices $8$-hedrite]{\epsfig{figure=ZZprojection7:3.ps,width=.20\textwidth}}}\caption{Two smallest $i$-hedrites with symmetry $D_3$}
%\label{fig:WithSymmetryD3}
%\end{figure}





\begin{remark}
The simplest case, $i=4$, of $i$-hedrites admits characterization for each of
its five possible groups.
It has symmetry $D_{2h}$  (respectively, $D_{2d}$) if and only if it is an
$t$-inflation, for some $t \ge 1, m \ge 2$, of $J_{4,2m}$ or $I_{4,2m+2}$ with
even $m$ (respectively, of $K_{4,4m}$ or $I_{4,2m+2}$ with odd $m$).
Any $4$-hedrite with group $D_{4h}$ or $D_4$ has $2(k^2+l^2)$ vertices for some
$k \ge l \ge 0$ (the case $D_{4h}$ corresponds to $l=k, 0$); it 
comes!! from the
smallest $4$-hedrite {\bf 2-1} by Goldgerg-Coxeter operation (see \cite{Gold37}, \cite{Cox71} and \cite{DD}).
All other 4-hedrites have symmetry $D_2$; in shift terms, they are exactly those, for which interchange of central circuits changes the value of shift.
\end{remark}


\begin{conjecture}
$C_{2v}$ $7$-hedrite exists and only one if and only if $n \ge 7$, except $8$ and $9$.
!!perhaps, it will be more decent just to say: We expect that...!!
\end{conjecture}




\section{Small $i$-hedrites}

Here and below all links are given 
in Rolfsen's notation (see the table in \cite{Rolf} and also,  
for example, \cite{Kaw}) for links with at most 9 
crossings and knots with 10 crossings, or, otherwise, in
Dowker-Thistlewhaite's numbering (see \cite{T}), if any.
We write $\sim$ if the projection in our Table is different 
from the one given in corresponding cases of one of the above 

We give below all $i$-hedrites with at most $12$ vertices indicating under 
picture of each its symmetry, CC-vector and corresponding alternating link.
If an $i$-hedrite is $2$-connected but not $3$-connected, then we add
a symbol ${\bf *}$ just after the number. If an $i$-hedrite is reducible
(i.e. has a rail-road), then we add mention ``red.''. All $i$-hedrites
with $13$ and $14$ vertices are listed in Table \ref{tab:i-hedrite13_14}. 
!!add, perhaps
Only three reducible $i$-hedrits with $n \ le 14$ have self-intersecting
railroad: $5$-hedrites {\bf 12-3}, {\bf 14-6} and $$-hedrite {\bf 13-11}.!!

On the Figure below, in order to express better the (maximal)
symmetry of an $i$-hedrite, we put:

(i) a double arrow, in order to represent an edge passing at infinity,

(ii) a quadruple arrow, in order to represent a vertex at infinity.





\subsection{$8$-hedrites}\label{subsection-8-hedrites}
{\small
\setlength{\unitlength}{1cm}
\begin{minipage}[t]{3.5cm}
\centering
\epsfxsize=2.5cm
\epsffile{8-hedrite6-1.eps}\par
{{\bf Nr.6-1} \quad $O_h$\\ $6^3_2$ \quad $(4^3)$\\\vspace{3mm} }
\end{minipage}
\setlength{\unitlength}{1cm}
\begin{minipage}[t]{3.5cm}
\centering
\epsfxsize=2.5cm
\epsffile{8-hedrite8-1.eps}\par
{{\bf Nr.8-1} \quad $D_{4d}$\\ $8_{18}$ \quad $(16)$\\\vspace{3mm} }
\end{minipage}
\setlength{\unitlength}{1cm}
\begin{minipage}[t]{3.5cm}
\centering
\epsfxsize=2.5cm
\epsffile{8-hedrite9_1.eps}\par
{{\bf Nr.9-1} \quad $D_{3h}$\\ $9_{40}$ \quad $(18)$\\\vspace{3mm} }
\end{minipage}
\setlength{\unitlength}{1cm}
\begin{minipage}[t]{3.5cm}
\centering
\epsfxsize=2.5cm
\epsffile{8-hedrite10_1sec.eps}\par
{{\bf Nr.10-1} \quad $D_{2}$\\ $10^2_{56}$ \quad $(6;14)$\\\vspace{3mm} }
\end{minipage}
\setlength{\unitlength}{1cm}
\begin{minipage}[t]{3.5cm}
\centering
\epsfxsize=2.5cm
\epsffile{8-hedrite10_2sec.eps}\par
{{\bf Nr.10-2} \quad $D_{4h}$\\ $10^4_{169}$ \quad $(4^2,6^2)$ red.\\\vspace{3mm} }
\end{minipage}
\setlength{\unitlength}{1cm}
\begin{minipage}[t]{3.5cm}
\centering
\epsfxsize=2.5cm
\epsffile{8-hedrite11_1.eps}\par
{{\bf Nr.11-1} \quad $C_{2v}$\\ $11^3_{520}$ \quad $(6^2;10)$\\\vspace{3mm} }
\end{minipage}
\setlength{\unitlength}{1cm}
\begin{minipage}[t]{3.5cm}
\centering
\epsfxsize=2.2cm
\epsffile{8-hedrite12-5cage.ps}\par
{{\bf Nr.12-1} \quad $D_{3d}$\\ $12_{1019}$ \quad $(24)$\\\vspace{3mm} }
\end{minipage}
\setlength{\unitlength}{1cm}
\begin{minipage}[t]{3.5cm}
\centering
\epsfxsize=2.5cm
\epsffile{8-hedrite12_2.eps}\par
{{\bf Nr.12-2} \quad $D_2$\\ $12_{868}$ \quad $(24)$\\\vspace{3mm} }
\end{minipage}
\setlength{\unitlength}{1cm}
\begin{minipage}[t]{3.5cm}
\centering
\epsfxsize=2.5cm
\epsffile{8-hedrite12_3thi.eps}\par
{{\bf Nr.12-3} \quad $C_2$\\ $?????$ \quad $(6;18)$\\\vspace{3mm} }
\end{minipage}
\setlength{\unitlength}{1cm}
\begin{minipage}[t]{3.5cm}
\centering
\epsfxsize=2.5cm
\epsffile{8-hedrite12_4.eps}\par
{{\bf Nr.12-4} \quad $O_h$\\ $?????$ \quad $(6^4)$\\\vspace{3mm} }
\end{minipage}
\setlength{\unitlength}{1cm}
\begin{minipage}[t]{3.5cm}
\centering
\epsfxsize=2.2cm
\epsffile{8-hedrite12-1cage.ps}\par
{{\bf Nr.12-5} \quad $D_{3h}$\\ $????$ \quad $(6^4)$\\\vspace{3mm} }
\end{minipage}
}






\subsection{$7$-hedrites}
{\small
\setlength{\unitlength}{1cm}
\begin{minipage}[t]{3.5cm}
\begin{picture}(3.5,3.5)
\leavevmode
\centering
\epsfxsize=2.5cm
\epsffile{7-hedrite7_1sec.eps}
\end{picture}\par
\begin{center}
{{\bf Nr.7-1} \quad $C_{2v}$\\ $7^2_{6}$ \quad $(4;10)$\\ }
\end{center}
\end{minipage}
\setlength{\unitlength}{1cm}
\begin{minipage}[t]{3.5cm}
\begin{picture}(3.5,3.5)
\leavevmode
\centering
\epsfxsize=2.5cm
\epsffile{7-hedrite8_1.eps}
\end{picture}\par
\begin{center}
{{\bf Nr.8-1} \quad $C_{s}$\\ $8^2_{13}$ \quad $(4;12)$\\ }
\end{center}
\end{minipage}
\setlength{\unitlength}{1cm}
\begin{minipage}[t]{3.5cm}
\begin{picture}(3.5,3.5)
\leavevmode
\centering
\epsfxsize=2.5cm
\epsffile{7-hedrite9_1.eps}
\end{picture}\par
\begin{center}
{{\bf Nr.9-1} \quad $C_{s}$\\ $9_{34}$ \quad $(18)$\\ }
\end{center}
\end{minipage}
\setlength{\unitlength}{1cm}
\begin{minipage}[t]{3.5cm}
\begin{picture}(3.5,3.5)
\leavevmode
\centering
\epsfxsize=2.5cm
\epsffile{7-hedrite10_1sec.eps}
\end{picture}\par
\begin{center}
{{\bf Nr.10-1} \quad $C_{s}$\\ $10_{121}$ \quad $(20)$\\ }
\end{center}
\end{minipage}
\setlength{\unitlength}{1cm}
\begin{minipage}[t]{3.5cm}
\begin{picture}(3.5,3.5)
\leavevmode
\centering
\epsfxsize=2.5cm
\epsffile{7-hedrite10_2sec.eps}
\end{picture}\par
\begin{center}
{{\bf Nr.10-2} \quad $C_{2v}$\\ $10^2_{111}$ \quad $(10^2)$\\ }
\end{center}
\end{minipage}
\setlength{\unitlength}{1cm}
\begin{minipage}[t]{3.5cm}
\begin{picture}(3.5,3.5)
\leavevmode
\centering
\epsfxsize=2.5cm
\epsffile{7-hedrite10_3.eps}
\end{picture}\par
\begin{center}
{{\bf Nr.10-3} \quad $C_{s}$\\ $\sim 10^2_{69}$ \quad $(8,12)$\\ }
\end{center}
\end{minipage}
\setlength{\unitlength}{1cm}
\begin{minipage}[t]{3.5cm}
\begin{picture}(3.5,3.5)
\leavevmode
\centering
\epsfxsize=2.5cm
\epsffile{7-hedrite11_1sec.eps}
\end{picture}\par
\begin{center}
{{\bf Nr.11-1} \quad $C_2$\\ $11_{288}$ \quad $(22)$\\ }
\end{center}
\end{minipage}
\setlength{\unitlength}{1cm}
\begin{minipage}[t]{3.5cm}
\begin{picture}(3.5,3.5)
\leavevmode
\centering
\epsfxsize=2.5cm
\epsffile{7-hedrite11_2.eps}
\end{picture}\par
\begin{center}
{{\bf Nr.11-2} \quad $C_{1}$\\ $11_{301}$ \quad $(22)$\\ }
\end{center}
\end{minipage}
\setlength{\unitlength}{1cm}
\begin{minipage}[t]{3.5cm}
\begin{picture}(3.5,3.5)
\leavevmode
\centering
\epsfxsize=2.5cm
\epsffile{7-hedrite11_3sec.eps}
\end{picture}\par
\begin{center}
{{\bf Nr.11-3} \quad $C_{s}$\\ $11^2_{150}$ \quad $(8,14)$\\ }
\end{center}
\end{minipage}
\setlength{\unitlength}{1cm}
\begin{minipage}[t]{3.5cm}
\begin{picture}(3.5,3.5)
\leavevmode
\centering
\epsfxsize=2.5cm
\epsffile{7-hedrite11_4sec.eps}
\end{picture}\par
\begin{center}
{{\bf Nr.11-4} \quad $C_{2v}$\\ $11^3_{487}$ \quad $(4^2;14)$ red.\\ }
\end{center}
\end{minipage}
\setlength{\unitlength}{1cm}
\begin{minipage}[t]{3.5cm}
\begin{picture}(3.5,3.5)
\leavevmode
\centering
\epsfxsize=2.5cm
\epsffile{7-hedrite12_1.eps}
\end{picture}\par
\begin{center}
{{\bf Nr.12-1} \quad $C_{1}$\\ $\sim 12_{361}$ \quad $(24)$\\ }
\end{center}
\end{minipage}
\setlength{\unitlength}{1cm}
\begin{minipage}[t]{3.5cm}
\begin{picture}(3.5,3.5)
\leavevmode
\centering
\epsfxsize=2.5cm
\epsffile{7-hedrite12_2.eps}
\end{picture}\par
\begin{center}
{{\bf Nr.12-2} \quad $C_{1}$\\ $?????$ \quad $(6;18)$\\ }
\end{center}
\end{minipage}
\setlength{\unitlength}{1cm}
\begin{minipage}[t]{3.5cm}
\begin{picture}(3.5,3.5)
\leavevmode
\epsfxsize=2.5cm
\epsffile{7-hedrite12_3sec.eps}
\end{picture}\par
\begin{center}
{{\bf Nr.12-3} \quad $C_{s}$\\ $????$ \quad $(10,14)$\\ }
\end{center}
\end{minipage}
\setlength{\unitlength}{1cm}
\begin{minipage}[t]{3.5cm}
\begin{picture}(3.5,3.5)
\leavevmode
\epsfxsize=2.5cm
\epsffile{7-hedrite12_4.eps}
\end{picture}\par
\begin{center}
{{\bf Nr.12-4} \quad $C_{2v}$\\ $????$ \quad $(6^2;12)$\\ }
\end{center}
\end{minipage}
\setlength{\unitlength}{1cm}
\begin{minipage}[t]{3.5cm}
\begin{picture}(3.5,3.5)
\leavevmode
\epsfxsize=2.5cm
\epsffile{7-hedrite12_5.eps}
\end{picture}\par
\begin{center}
{{\bf Nr.12-5} \quad $C_{s}$\\ $???$ \quad $(4^2;16)$ red.\\ }
\end{center}
\end{minipage}
}







\subsection{$6$-hedrites}
{\small
\setlength{\unitlength}{1cm}
\begin{minipage}[t]{3.5cm}
\begin{picture}(3.5,3.5)
\leavevmode
\epsfxsize=2.5cm
\epsffile{6-hedrite4_1.eps}
\end{picture}\par
\begin{center}
{{\bf Nr.4-1${}^*$} \quad $D_{2d}$\\ $4_{1}$ \quad $(8)$\\ }
\end{center}
\end{minipage}
\setlength{\unitlength}{1cm}
\begin{minipage}[t]{3.5cm}
\begin{picture}(3.5,3.5)
\leavevmode
\epsfxsize=2.5cm
\epsffile{6-hedrite5_1.eps}
\end{picture}\par
\begin{center}
{{\bf Nr.5-1} \quad $C_{2v}$\\ $5^2_{1}$ \quad $(4;6)$\\ }
\end{center}
\end{minipage}
\setlength{\unitlength}{1cm}
\begin{minipage}[t]{3.5cm}
\begin{picture}(3.5,3.5)
\leavevmode
\epsfxsize=2.5cm
\epsffile{6-hedrite6_1sec.eps}
\end{picture}\par
\begin{center}
{{\bf Nr.6-1} \quad $C_{2}$\\ $6_{3}$ \quad $(12)$\\ }
\end{center}
\end{minipage}
\setlength{\unitlength}{1cm}
\begin{minipage}[t]{3.5cm}
\begin{picture}(3.5,3.5)
\leavevmode
\epsfxsize=2.5cm
\epsffile{6-hedrite6_2.eps}
\end{picture}\par
\begin{center}
{{\bf Nr.6-2${}^*$} \quad $D_{2h}$\\ $\sim 6^2_{3}$ \quad $(6^2)$\\ }
\end{center}
\end{minipage}
\setlength{\unitlength}{1cm}
\begin{minipage}[t]{3.5cm}
\begin{picture}(3.5,3.5)
\leavevmode
\epsfxsize=2.5cm
\epsffile{6-hedrite7_1.eps}
\end{picture}\par
\begin{center}
{{\bf Nr.7-1} \quad $C_{2}$\\ $\sim 7_{7}$ \quad $(14)$\\ }
\end{center}
\end{minipage}
\setlength{\unitlength}{1cm}
\begin{minipage}[t]{3.5cm}
\begin{picture}(3.5,3.5)
\leavevmode
\epsfxsize=2.5cm
\epsffile{6-hedrite8_1sec.eps}
\end{picture}\par
\begin{center}
{{\bf Nr.8-1${}^*$} \quad $D_{2h}$\\ $\sim 8_{12}$ \quad $(16)$\\ }
\end{center}
\end{minipage}
\setlength{\unitlength}{1cm}
\begin{minipage}[t]{3.5cm}
\begin{picture}(3.5,3.5)
\leavevmode
\epsfxsize=2.5cm
\epsffile{6-hedrite8_2sec.eps}
\end{picture}\par
\begin{center}
{{\bf Nr.8-2} \quad $C_{2}$\\ $8_{17}$ \quad $(16)$\\ }
\end{center}
\end{minipage}
\setlength{\unitlength}{1cm}
\begin{minipage}[t]{3.5cm}
\begin{picture}(3.5,3.5)
\leavevmode
\epsfxsize=2.5cm
\epsffile{6-hedrite8_3.eps}
\end{picture}\par
\begin{center}
{{\bf Nr.8-3} \quad $D_{2d}$\\ $8^2_{14}$ \quad $(4;12)$\\ }
\end{center}
\end{minipage}
\setlength{\unitlength}{1cm}
\begin{minipage}[t]{3.5cm}
\begin{picture}(3.5,3.5)
\leavevmode
\epsfxsize=2.5cm
\epsffile{6-hedrite8_4.eps}
\end{picture}\par
\begin{center}
{{\bf Nr.8-4} \quad $C_{2}$\\ $\sim 8^2_{8}$ \quad $(6;10)$\\ }
\end{center}
\end{minipage}
\setlength{\unitlength}{1cm}
\begin{minipage}[t]{3.5cm}
\begin{picture}(3.5,3.5)
\leavevmode
\epsfxsize=2.5cm
\epsffile{6-hedrite8_5.eps}
\end{picture}\par
\begin{center}
{{\bf Nr.8-5} \quad $D_{2h}$\\ $8^3_{6}$ \quad $(4,6^2)$\\ }
\end{center}
\end{minipage}
\setlength{\unitlength}{1cm}
\begin{minipage}[t]{3.5cm}
\begin{picture}(3.5,3.5)
\leavevmode
\epsfxsize=2.5cm
\epsffile{6-hedrite9_1.eps}
\end{picture}\par
\begin{center}
{{\bf Nr.9-1} \quad $C_{2}$\\ $\sim 9_{31}$ \quad $(18)$\\ }
\end{center}
\end{minipage}
\setlength{\unitlength}{1cm}
\begin{minipage}[t]{3.5cm}
\begin{picture}(3.5,3.5)
\leavevmode
\epsfxsize=2.5cm
\epsffile{6-hedrite9_2.eps}
\end{picture}\par
\begin{center}
{{\bf Nr.9-2} \quad $C_{1}$\\ $9_{33}$ \quad $(18)$\\ }
\end{center}
\end{minipage}
\setlength{\unitlength}{1cm}
\begin{minipage}[t]{3.5cm}
\begin{picture}(3.5,3.5)
\leavevmode
\epsfxsize=2.5cm
\epsffile{6-hedrite9_3sec.eps}
\end{picture}\par
\begin{center}
{{\bf Nr.9-3} \quad $C_{s}$\\ $9^2_{38}$ \quad $(4;14)$\\ }
\end{center}
\end{minipage}
\setlength{\unitlength}{1cm}
\begin{minipage}[t]{3.5cm}
\begin{picture}(3.5,3.5)
\leavevmode
\epsfxsize=2.5cm
\epsffile{6-hedrite9_4sec.eps}
\end{picture}\par
\begin{center}
{{\bf Nr.9-4} \quad $C_{2v}$\\ $9^3_{12}$ \quad $(4^2;10)$ red.\\ }
\end{center}
\end{minipage}
\setlength{\unitlength}{1cm}
\begin{minipage}[t]{3.5cm}
\begin{picture}(3.5,3.5)
\leavevmode
\epsfxsize=2.5cm
\epsffile{6-hedrite9_5.eps}
\end{picture}\par
\begin{center}
{{\bf Nr.9-5} \quad $C_{s}$\\ $9^3_{11}$ \quad $(4,6;8)$\\ }
\end{center}
\end{minipage}
%\setlength{\unitlength}{1cm}
%\begin{minipage}[t]{3.5cm}
%\begin{picture}(3.5,3.5)
%\leavevmode
%\epsfxsize=2.5cm
%\epsffile{6-hedrite10_1.eps}
%\end{picture}\par
%\begin{center}
%{{\bf Nr.10-1} \quad $C_{2}$\\ $10_{115}$ \quad $(20)$\\ }
%\end{center}
%\end{minipage}
\setlength{\unitlength}{1cm}
\begin{minipage}[t]{3.5cm}
\begin{picture}(3.5,3.5)
\leavevmode
\epsfxsize=2.5cm
\epsffile{6-hedrite10_3.eps}
\end{picture}\par
\begin{center}
{{\bf Nr.10-1} \quad $C_{2v}$\\ $10_{120}$ \quad $(20)$\\ }
\end{center}
\end{minipage}
\setlength{\unitlength}{1cm}
\begin{minipage}[t]{3.5cm}
\begin{picture}(3.5,3.5)
\leavevmode
\epsfxsize=2.5cm
\epsffile{6-hedrite10_4sec.eps}
\end{picture}\par
\begin{center}
{{\bf Nr.10-2} \quad $C_{2}$\\ $\sim 10_{88}$ \quad $(20)$\\ }
\end{center}
\end{minipage}
\setlength{\unitlength}{1cm}
\begin{minipage}[t]{3.5cm}
\begin{picture}(3.5,3.5)
\leavevmode
\epsfxsize=2.5cm
\epsffile{6-hedrite10_5sec.eps}
\end{picture}\par
\begin{center}
{{\bf Nr.10-3} \quad $C_{2}$\\ $\sim 10_{45}$ \quad $(20)$\\ }
\end{center}
\end{minipage}
\setlength{\unitlength}{1cm}
\begin{minipage}[t]{3.5cm}
\begin{picture}(3.5,3.5)
\leavevmode
\epsfxsize=2.5cm
\epsffile{6-hedrite10_6sec.eps}
\end{picture}\par
\begin{center}
{{\bf Nr.10-4} \quad $C_{2}$\\ $10_{115}$ \quad $(20)$\\ }
\end{center}
\end{minipage}
\setlength{\unitlength}{1cm}
\begin{minipage}[t]{3.5cm}
\begin{picture}(3.5,3.5)
\leavevmode
\epsfxsize=2.5cm
\epsffile{6-hedrite10_7.eps}
\end{picture}\par
\begin{center}
{{\bf Nr.10-5${}^*$} \quad $D_{2h}$\\ $\sim 10^2_{87}$ \quad $(10^2)$\\ }
\end{center}
\end{minipage}
\setlength{\unitlength}{1cm}
\begin{minipage}[t]{3.5cm}
\begin{picture}(3.5,3.5)
\leavevmode
\epsfxsize=2.5cm
\epsffile{6-hedrite10_9sec.eps}
\end{picture}\par
\begin{center}
{{\bf Nr.10-6} \quad $C_{2}$\\ $10^2_{86}$ \quad $(6;14)$\\ }
\end{center}
\end{minipage}
\setlength{\unitlength}{1cm}
\begin{minipage}[t]{3.5cm}
\begin{picture}(3.5,3.5)
\leavevmode
\epsfxsize=2.5cm
\epsffile{6-hedrite10_8sec.eps}
\end{picture}\par
\begin{center}
{{\bf Nr.10-7} \quad $C_{2}$\\ $10^2_{43}$ \quad $(4;16)$\\ }
\end{center}
\end{minipage}
\setlength{\unitlength}{1cm}
\begin{minipage}[t]{3.5cm}
\begin{picture}(3.5,3.5)
\leavevmode
\epsfxsize=2.5cm
\epsffile{6-hedrite10_10sec.eps}
\end{picture}\par
\begin{center}
{{\bf Nr.10-8} \quad $C_{2h}$\\ $\sim 10^3_{136}$ \quad $(4;8^2)$\\ }
\end{center}
\end{minipage}
\setlength{\unitlength}{1cm}
\begin{minipage}[t]{3.5cm}
\begin{picture}(3.5,3.5)
\leavevmode
\epsfxsize=2.5cm
\epsffile{6-hedrite10_11sec.eps}
\end{picture}\par
\begin{center}
{{\bf Nr.10-9} \quad $C_{2v}$\\ $10^3_{136}$ \quad $(4,6;10)$\\ }
\end{center}
\end{minipage}
\setlength{\unitlength}{1cm}
\begin{minipage}[t]{3.5cm}
\begin{picture}(3.5,3.5)
\leavevmode
\epsfxsize=2.5cm
\epsffile{6-hedrite11_1.eps}
\end{picture}\par
\begin{center}
{{\bf Nr.11-1} \quad $C_{2v}$\\ $11_{332}$ \quad $(22)$\\ }
\end{center}
\end{minipage}
\setlength{\unitlength}{1cm}
\begin{minipage}[t]{3.5cm}
\begin{picture}(3.5,3.5)
\leavevmode
\epsfxsize=2.5cm
\epsffile{6-hedrite11_2.eps}
\end{picture}\par
\begin{center}
{{\bf Nr.11-2} \quad $C_{2v}$\\ $11_{297}$ \quad $(22)$\\ }
\end{center}
\end{minipage}
\setlength{\unitlength}{1cm}
\begin{minipage}[t]{3.5cm}
\begin{picture}(3.5,3.5)
\leavevmode
\epsfxsize=2.5cm
\epsffile{6-hedrite11_3.eps}
\end{picture}\par
\begin{center}
{{\bf Nr.11-3} \quad $C_{1}$\\ $\sim 11_{125}$ \quad $(22)$\\ }
\end{center}
\end{minipage}
\setlength{\unitlength}{1cm}
\begin{minipage}[t]{3.5cm}
\begin{picture}(3.5,3.5)
\leavevmode
\epsfxsize=2.5cm
\epsffile{6-hedrite11_6.eps}
\end{picture}\par
\begin{center}
{{\bf Nr.11-4} \quad $C_{2}$\\ $11^2_{317}$ \quad $(8;14)$\\ }
\end{center}
\end{minipage}
\setlength{\unitlength}{1cm}
\begin{minipage}[t]{3.5cm}
\begin{picture}(3.5,3.5)
\leavevmode
\epsfxsize=2.5cm
\epsffile{6-hedrite11_5.eps}
\end{picture}\par
\begin{center}
{{\bf Nr.11-5} \quad $C_{2}$\\ $11^2_{?????}$ \quad $(8;14)$\\ }
\end{center}
\end{minipage}
\setlength{\unitlength}{1cm}
\begin{minipage}[t]{3.5cm}
\begin{picture}(3.5,3.5)
\leavevmode
\epsfxsize=2.5cm
\epsffile{6-hedrite11_4.eps}
\end{picture}\par
\begin{center}
{{\bf Nr.11-6} \quad $C_{s}$\\ $11^2_{????}$ \quad $(8,14)$\\ }
\end{center}
\end{minipage}
\setlength{\unitlength}{1cm}
\begin{minipage}[t]{3.5cm}
\begin{picture}(3.5,3.5)
\leavevmode
\epsfxsize=2.5cm
\epsffile{6-hedrite11_7.eps}
\end{picture}\par
\begin{center}
{{\bf Nr.11-7} \quad $C_{s}$\\ $11^2_{351}$ \quad $(10,12)$\\ }
\end{center}
\end{minipage}
\setlength{\unitlength}{1cm}
\begin{minipage}[t]{3.5cm}
\begin{picture}(3.5,3.5)
\leavevmode
\epsfxsize=2.5cm
\epsffile{6-hedrite12_2sec.eps}
\end{picture}\par
\begin{center}
{{\bf Nr.12-1${}^*$} \quad $D_{2d}$\\ $\sim 12_{477}$ \quad $(24)$\\ }
\end{center}
\end{minipage}
\setlength{\unitlength}{1cm}
\begin{minipage}[t]{3.5cm}
\begin{picture}(3.5,3.5)
\leavevmode
\epsfxsize=2.5cm
\epsffile{6-hedrite12_4sec.eps}
\end{picture}\par
\begin{center}
{{\bf Nr.12-2} \quad $D_{2}$\\ $12_{1152}$ \quad $(24)$\\ }
\end{center}
\end{minipage}
\setlength{\unitlength}{1cm}
\begin{minipage}[t]{3.5cm}
\begin{picture}(3.5,3.5)
\leavevmode
\epsfxsize=2.5cm
\epsffile{6-hedrite12_1sec.eps}
\end{picture}\par
\begin{center}
{{\bf Nr.12-3} \quad $C_{2}$\\ $\sim 12_{499}$ \quad $(24)$\\ }
\end{center}
\end{minipage}
\setlength{\unitlength}{1cm}
\begin{minipage}[t]{3.5cm}
\begin{picture}(3.5,3.5)
\leavevmode
\epsfxsize=2.5cm
\epsffile{6-hedrite12_3sec.eps}
\end{picture}\par
\begin{center}
{{\bf Nr.12-4} \quad $C_{2}$\\ $\sim 12_{458}$ \quad $(24)$\\ }
\end{center}
\end{minipage}
\setlength{\unitlength}{1cm}
\begin{minipage}[t]{3.5cm}
\begin{picture}(3.5,3.5)
\leavevmode
\epsfxsize=2.5cm
\epsffile{6-hedrite12_5sec.eps}
\end{picture}\par
\begin{center}
{{\bf Nr.12-5} \quad $C_{2}$\\ $12_{1102}$ \quad $(24)$\\ }
\end{center}
\end{minipage}
\setlength{\unitlength}{1cm}
\begin{minipage}[t]{3.5cm}
\begin{picture}(3.5,3.5)
\leavevmode
\epsfxsize=2.5cm
\epsffile{6-hedrite12_7sec.eps}
\end{picture}\par
\begin{center}
{{\bf Nr.12-6} \quad $C_{2}$\\ $12_{1167}$ \quad $(24)$\\ }
\end{center}
\end{minipage}
\setlength{\unitlength}{1cm}
\begin{minipage}[t]{3.5cm}
\begin{picture}(3.5,3.5)
\leavevmode
\epsfxsize=2.5cm
\epsffile{6-hedrite12_6.eps}
\end{picture}\par
\begin{center}
{{\bf Nr.12-7} \quad $C_{1}$\\ $\sim 12_{626}$ \quad $(24)$\\ }
\end{center}
\end{minipage}
\setlength{\unitlength}{1cm}
\begin{minipage}[t]{3.5cm}
\begin{picture}(3.5,3.5)
\leavevmode
\epsfxsize=2.5cm
\epsffile{6-hedrite12_8.eps}
\end{picture}\par
\begin{center}
{{\bf Nr.12-8} \quad $C_{1}$\\ $????$ \quad $(6;18)$\\ }
\end{center}
\end{minipage}
\setlength{\unitlength}{1cm}
\begin{minipage}[t]{3.5cm}
\begin{picture}(3.5,3.5)
\leavevmode
\epsfxsize=2.5cm
\epsffile{6-hedrite12_9sec.eps}
\end{picture}\par
\begin{center}
{{\bf Nr.12-9} \quad $C_{2}$\\ $????$ \quad $(10,14)$\\ }
\end{center}
\end{minipage}
\setlength{\unitlength}{1cm}
\begin{minipage}[t]{3.5cm}
\begin{picture}(3.5,3.5)
\leavevmode
\epsfxsize=2.5cm
\epsffile{6-hedrite12_10sec.eps}
\end{picture}\par
\begin{center}
{{\bf Nr.12-10} \quad $C_{s}$\\ $?????$ \quad $(8,16)$\\ }
\end{center}
\end{minipage}
\setlength{\unitlength}{1cm}
\begin{minipage}[t]{3.5cm}
\begin{picture}(3.5,3.5)
\leavevmode
\epsfxsize=2.5cm
\epsffile{6-hedrite12_11sec.eps}
\end{picture}\par
\begin{center}
{{\bf Nr.12-11} \quad $D_{2d}$\\ $?????$ \quad $(4^2;16)$ red.\\ }
\end{center}
\end{minipage}
\setlength{\unitlength}{1cm}
\begin{minipage}[t]{3.5cm}
\begin{picture}(3.5,3.5)
\leavevmode
\epsfxsize=2.5cm
\epsffile{6-hedrite12_12.eps}
\end{picture}\par
\begin{center}
{{\bf Nr.12-12} \quad $C_{2v}$\\ $????$ \quad $(8;8^2)$\\ }
\end{center}
\end{minipage}
\setlength{\unitlength}{1cm}
\begin{minipage}[t]{3.5cm}
\begin{picture}(3.5,3.5)
\leavevmode
\epsfxsize=2.5cm
\epsffile{6-hedrite12_13.eps}
\end{picture}\par
\begin{center}
{{\bf Nr.12-13} \quad $C_{s}$\\ $?????$ \quad $(6;8,10)$\\ }
\end{center}
\end{minipage}
\setlength{\unitlength}{1cm}
\begin{minipage}[t]{3.5cm}
\begin{picture}(3.5,3.5)
\leavevmode
\epsfxsize=2.5cm
\epsffile{6-hedrite12_14.eps}
\end{picture}\par
\begin{center}
{{\bf Nr.12-14} \quad $D_{2h}$\\ $????$ \quad $(4^2,8^2)$ red.\\ }
\end{center}
\end{minipage}
}












\subsection{$5$-hedrites}
{\small
\setlength{\unitlength}{1cm}
\begin{minipage}[t]{3.5cm}
\begin{picture}(3.5,3.5)
\leavevmode
\epsfxsize=2.5cm
\epsffile{5-hedrite3_1.eps}
\end{picture}\par
\begin{center}
{{\bf Nr.3-1} \quad $D_{3h}$\\ $3_{1}$ \quad $(6)$\\ }
\end{center}
\end{minipage}
\setlength{\unitlength}{1cm}
\begin{minipage}[t]{3.5cm}
\begin{picture}(3.5,3.5)
\leavevmode
\epsfxsize=2.5cm
\epsffile{5-hedrite5_1.eps}
\end{picture}\par
\begin{center}
{{\bf Nr.5-1${}^*$} \quad $C_{2v}$\\ $5_{2}$ \quad $(10)$\\ }
\end{center}
\end{minipage}
\setlength{\unitlength}{1cm}
\begin{minipage}[t]{3.5cm}
\begin{picture}(3.5,3.5)
\leavevmode
\epsfxsize=2.5cm
\epsffile{5-hedrite6_1.eps}
\end{picture}\par
\begin{center}
{{\bf Nr.6-1} \quad $C_{2v}$\\ $6^2_{3}$ \quad $(4;8)$\\ }
\end{center}
\end{minipage}
\setlength{\unitlength}{1cm}
\begin{minipage}[t]{3.5cm}
\begin{picture}(3.5,3.5)
\leavevmode
\epsfxsize=2.5cm
\epsffile{5-hedrite6_2.eps}
\end{picture}\par
\begin{center}
{{\bf Nr.6-2} \quad $D_{3h}$\\ $6^3_{1}$ \quad $(4^3)$\\ }
\end{center}
\end{minipage}
\setlength{\unitlength}{1cm}
\begin{minipage}[t]{3.5cm}
\begin{picture}(3.5,3.5)
\leavevmode
\epsfxsize=2.5cm
\epsffile{5-hedrite7_1sec.eps}
\end{picture}\par
\begin{center}
{{\bf Nr.7-1${}^*$} \quad $C_{2v}$\\ $\sim 7_{5}$ \quad $(14)$\\ }
\end{center}
\end{minipage}
\setlength{\unitlength}{1cm}
\begin{minipage}[t]{3.5cm}
\begin{picture}(3.5,3.5)
\leavevmode
\epsfxsize=2.5cm
\epsffile{5-hedrite7_2.eps}
\end{picture}\par
\begin{center}
{{\bf Nr.7-2} \quad $C_{s}$\\ $7^2_{5}$ \quad $(4;10)$\\ }
\end{center}
\end{minipage}
\setlength{\unitlength}{1cm}
\begin{minipage}[t]{3.5cm}
\begin{picture}(3.5,3.5)
\leavevmode
\epsfxsize=2.5cm
\epsffile{5-hedrite7_3sec.eps}
\end{picture}\par
\begin{center}
{{\bf Nr.7-3} \quad $C_{2v}$\\ $7^3_1$ \quad $(4^2;6)$\\ }
\end{center}
\end{minipage}
\setlength{\unitlength}{1cm}
\begin{minipage}[t]{3.5cm}
\begin{picture}(3.5,3.5)
\leavevmode
\epsfxsize=2.5cm
\epsffile{5-hedrite8_1sec.eps}
\end{picture}\par
\begin{center}
{{\bf Nr.8-1} \quad $C_{2}$\\ $\sim 8_{15}$ \quad $(16)$\\ }
\end{center}
\end{minipage}
\setlength{\unitlength}{1cm}
\begin{minipage}[t]{3.5cm}
\begin{picture}(3.5,3.5)
\leavevmode
\epsfxsize=2.5cm
\epsffile{5-hedrite9_1sec.eps}
\end{picture}\par
\begin{center}
{{\bf Nr.9-1} \quad $C_{2}$\\ $9_{38}$ \quad $(18)$\\ }
\end{center}
\end{minipage}
\setlength{\unitlength}{1cm}
\begin{minipage}[t]{3.5cm}
\begin{picture}(3.5,3.5)
\leavevmode
\epsfxsize=2.5cm
\epsffile{5-hedrite9_2sec.eps}
\end{picture}\par
\begin{center}
{{\bf Nr.9-2${}^*$} \quad $C_{2v}$\\ $\sim 9_{18}$ \quad $(18)$\\ }
\end{center}
\end{minipage}
\setlength{\unitlength}{1cm}
\begin{minipage}[t]{3.5cm}
\begin{picture}(3.5,3.5)
\leavevmode
\epsfxsize=2.5cm
\epsffile{5-hedrite10_3.eps}
\end{picture}\par
\begin{center}
{{\bf Nr.10-1} \quad $C_{2v}$\\ $10^3_{155}$ \quad $(4^2;12)$ red.\\ }
\end{center}
\end{minipage}
\setlength{\unitlength}{1cm}
\begin{minipage}[t]{3.5cm}
\begin{picture}(3.5,3.5)
\leavevmode
\epsfxsize=2.5cm
\epsffile{5-hedrite10_1.eps}
\end{picture}\par
\begin{center}
{{\bf Nr.10-2} \quad $C_{1}$\\ $10^2_{85}$ \quad $(6;14)$\\ }
\end{center}
\end{minipage}
\setlength{\unitlength}{1cm}
\begin{minipage}[t]{3.5cm}
\begin{picture}(3.5,3.5)
\leavevmode
\epsfxsize=2.5cm
\epsffile{5-hedrite10_2sec.eps}
\end{picture}\par
\begin{center}
{{\bf Nr.10-3} \quad $C_{2v}$\\ $10^4_{173}$ \quad $(4^2,6^2)$ red.\\ }
\end{center}
\end{minipage}
\setlength{\unitlength}{1cm}
\begin{minipage}[t]{3.5cm}
\begin{picture}(3.5,3.5)
\leavevmode
\epsfxsize=2.5cm
\epsffile{5-hedrite11_2sec.eps}
\end{picture}\par
\begin{center}
{{\bf Nr.11-1${}^*$} \quad $C_{2v}$\\ $\sim 11_{236}$ \quad $(22)$\\ }
\end{center}
\end{minipage}
\setlength{\unitlength}{1cm}
\begin{minipage}[t]{3.5cm}
\begin{picture}(3.5,3.5)
\leavevmode
\epsfxsize=2.5cm
\epsffile{5-hedrite11_1sec.eps}
\end{picture}\par
\begin{center}
{{\bf Nr.11-2} \quad $C_{2}$\\ $\sim 11_{124}$ \quad $(22)$\\ }
\end{center}
\end{minipage}
\setlength{\unitlength}{1cm}
\begin{minipage}[t]{3.5cm}
\begin{picture}(3.5,3.5)
\leavevmode
\epsfxsize=2.5cm
\epsffile{5-hedrite11_3.eps}
\end{picture}\par
\begin{center}
{{\bf Nr.11-3} \quad $C_{1}$\\ $11^2_{226}$ \quad $(6;16)$\\ }
\end{center}
\end{minipage}
\setlength{\unitlength}{1cm}
\begin{minipage}[t]{3.5cm}
\begin{picture}(3.5,3.5)
\leavevmode
\epsfxsize=2.5cm
\epsffile{5-hedrite11_4.eps}
\end{picture}\par
\begin{center}
{{\bf Nr.11-4} \quad $C_{s}$\\ $11^3_{500}$ \quad $(4^2;14)$ red.\\ }
\end{center}
\end{minipage}
\setlength{\unitlength}{1cm}
\begin{minipage}[t]{3.5cm}
\begin{picture}(3.5,3.5)
\leavevmode
\epsfxsize=2.5cm
\epsffile{5-hedrite11_5.eps}
\end{picture}\par
\begin{center}
{{\bf Nr.11-5} \quad $C_{s}$\\ $11^4_{547}$ \quad $(4^2,6;8)$ red.\\ }
\end{center}
\end{minipage}
\setlength{\unitlength}{1cm}
\begin{minipage}[t]{3.5cm}
\begin{picture}(3.5,3.5)
\leavevmode
\epsfxsize=2.5cm
\epsffile{5-hedrite12_1.eps}
\end{picture}\par
\begin{center}
{{\bf Nr.12-1} \quad $C_{1}$\\  $\sim 12_{431}$ \quad $(24)$\\ }
\end{center}
\end{minipage}
\setlength{\unitlength}{1cm}
\begin{minipage}[t]{3.5cm}
\begin{picture}(3.5,3.5)
\leavevmode
\epsfxsize=2.5cm
\epsffile{5-hedrite12_2.eps}
\end{picture}\par
\begin{center}
{{\bf Nr.12-2} \quad $C_{2v}$\\ $????$ \quad $(12^2)$\\ }
\end{center}
\end{minipage}
\setlength{\unitlength}{1cm}
\begin{minipage}[t]{3.5cm}
\begin{picture}(3.5,3.5)
\leavevmode
\epsfxsize=2.5cm
\epsffile{5-hedrite12D3hsec.eps}
\end{picture}\par
\begin{center}
{{\bf Nr.12-3} \quad $D_{3h}$\\ $????$ \quad $(12^2)$ red.\\ }
\end{center}
\end{minipage}
}







\subsection{$4$-hedrites}
{\small
\setlength{\unitlength}{1cm}
\begin{minipage}[t]{3.5cm}
\begin{picture}(3.5,3.5)
\leavevmode
\epsfxsize=2.5cm
\epsffile{4-hedrite2_1.eps}
\end{picture}\par
\begin{center}
{{\bf Nr.2-1} \quad $D_{4h}$\\ $2^2_1$ \quad $(2^2)$\\ }
\end{center}
\end{minipage}
\setlength{\unitlength}{1cm}
\begin{minipage}[t]{3.5cm}
\begin{picture}(3.5,3.5)
\leavevmode
\epsfxsize=2.5cm
\epsffile{4-hedrite4_1.eps}
\end{picture}\par
\begin{center}
{{\bf Nr.4-1} \quad $D_{4h}$\\ $4^2_1$ \quad $(4^2)$\\ }
\end{center}
\end{minipage}
\setlength{\unitlength}{1cm}
\begin{minipage}[t]{3.5cm}
\begin{picture}(3.5,3.5)
\leavevmode
\epsfxsize=2.5cm
\epsffile{4-hedrite4-2.eps}
\end{picture}\par
\begin{center}
{{\bf Nr.4-2${}^*$} \quad $D_{2h}$\\ $2\times 2^2_1$ \quad $(2^2,4)$ red.\\}
\end{center}
\end{minipage}
\setlength{\unitlength}{1cm}
\begin{minipage}[t]{3.5cm}
\begin{picture}(3.5,3.5)
\leavevmode
\epsfxsize=2.5cm
\epsffile{4-hedrite6_1sec.eps}
\end{picture}\par
\begin{center}
{{\bf Nr.6-1${}^*$} \quad $D_{2d}$\\ $6^2_2$ \quad $(6^2)$\\ }
\end{center}
\end{minipage}
\setlength{\unitlength}{1cm}
\begin{minipage}[t]{3.5cm}
\begin{picture}(3.5,3.5)
\leavevmode
\epsfxsize=2.5cm
\epsffile{4-hedrite6-2.eps}
\end{picture}\par
\begin{center}
{{\bf Nr.6-2${}^*$} \quad $D_{2h}$\\ $3\times 2^2_1$ \quad $(2^3,6)$ red.\\}
\end{center}
\end{minipage}
\setlength{\unitlength}{1cm}
\begin{minipage}[t]{3.5cm}
\begin{picture}(3.5,3.5)
\leavevmode
\epsfxsize=2.5cm
\epsffile{4-hedrite8_1.eps}
\end{picture}\par
\begin{center}
{{\bf Nr.8-1${}^*$} \quad $D_{2h}$\\ $\sim 8^2_4$ \quad $(8^2)$\\ }
\end{center}
\end{minipage}
\setlength{\unitlength}{1cm}
\begin{minipage}[t]{3.5cm}
\begin{picture}(3.5,3.5)
\leavevmode
\epsfxsize=2.5cm
\epsffile{4-hedrite8_2.eps}
\end{picture}\par
\begin{center}
{{\bf Nr.8-2} \quad $D_{2d}$\\ $8^3_4$ \quad $(4^2,8)$ red.\\ }
\end{center}
\end{minipage}
\setlength{\unitlength}{1cm}
\begin{minipage}[t]{3.5cm}
\begin{picture}(3.5,3.5)
\leavevmode
\epsfxsize=2.5cm
\epsffile{4-hedrite8_3.eps}
\end{picture}\par
\begin{center}
{{\bf Nr.8-3} \quad $D_{4h}$\\ $8^4_{1}$ \quad $(4^4)$ red.\\ }
\end{center}
\end{minipage}
\setlength{\unitlength}{1cm}
\begin{minipage}[t]{3.5cm}
\begin{picture}(3.5,3.5)
\leavevmode
\epsfxsize=2.5cm
\epsffile{4-hedrite8-4.eps}
\end{picture}\par
\begin{center}
{{\bf Nr.8-4${}^*$} \quad $D_{2h}$\\ $4\times 2^2_1$ \quad $(2^4,8)$ red.\\}
\end{center}
\end{minipage}
\setlength{\unitlength}{1cm}
\begin{minipage}[t]{3.5cm}
\begin{picture}(3.5,3.5)
\leavevmode
\epsfxsize=2.5cm
\epsffile{4-hedrite10_1sec.eps}
\end{picture}\par
\begin{center}
{{\bf Nr.10-1} \quad $D_4$\\ $10^2_{121}$ \quad $(10^2)$\\ }
\end{center}
\end{minipage}
\setlength{\unitlength}{1cm}
\begin{minipage}[t]{3.5cm}
\begin{picture}(3.5,3.5)
\leavevmode
\epsfxsize=2.5cm
\epsffile{4-hedrite10_2sec.eps}
\end{picture}\par
\begin{center}
{{\bf Nr.10-2${}^*$} \quad $D_{2d}$\\ $\sim 10^2_{120}$ \quad $(10^2)$\\ }
\end{center}
\end{minipage}
\setlength{\unitlength}{1cm}
\begin{minipage}[t]{3.5cm}
\begin{picture}(3.5,3.5)
\leavevmode
\epsfxsize=2.5cm
\epsffile{4-hedrite10-3.eps}
\end{picture}\par
\begin{center}
{{\bf Nr.10-3${}^*$} \quad $D_{2h}$\\ $5\times 2^2_1$ \quad $(2^5,10)$ red.\\}
\end{center}
\end{minipage}
\setlength{\unitlength}{1cm}
\begin{minipage}[t]{3.5cm}
\begin{picture}(3.5,3.5)
\leavevmode
\epsfxsize=2.5cm
\epsffile{4-hedrite12_1.eps}
\end{picture}\par
\begin{center}
{{\bf Nr.12-1${}^*$} \quad $D_{2h}$\\ $?????$ \quad $(12^2)$\\ }
\end{center}
\end{minipage}
\setlength{\unitlength}{1cm}
\begin{minipage}[t]{3.5cm}
\begin{picture}(3.5,3.5)
\leavevmode
\epsfxsize=2.5cm
\epsffile{4-hedrite12_2.eps}
\end{picture}\par
\begin{center}
{{\bf Nr.12-2} \quad $D_2$\\ $????$ \quad $(6^2,12)$ red.\\ }
\end{center}
\end{minipage}
\setlength{\unitlength}{1cm}
\begin{minipage}[t]{3.5cm}
\begin{picture}(3.5,3.5)
\leavevmode
\epsfxsize=2.5cm
\epsffile{4-hedrite12_3.eps}
\end{picture}\par
\begin{center}
{{\bf Nr.12-3} \quad $D_{2d}$\\ $????$ \quad $(4^3,12)$ red.\\ }
\end{center}
\end{minipage}
\setlength{\unitlength}{1cm}
\begin{minipage}[t]{3.5cm}
\begin{picture}(3.5,3.5)
\leavevmode
\epsfxsize=2.5cm
\epsffile{4-hedrite12_4.eps}
\end{picture}\par
\begin{center}
{{\bf Nr.12-4} \quad $D_{2h}$\\ $????$ \quad $(4^3,6^2)$ red.\\ }
\end{center}
\end{minipage}
\setlength{\unitlength}{1cm}
\begin{minipage}[t]{3.5cm}
\begin{picture}(3.5,3.5)
\leavevmode
\epsfxsize=2.5cm
\epsffile{4-hedrite12-5.eps}
\end{picture}\par
\begin{center}
{{\bf Nr.12-5${}^*$} \quad $D_{2h}$\\ $6\times 2^2_1$ \quad $(2^6,12)$ red.\\}
\end{center}
\end{minipage}
}




\begin{table}
\begin{center}
{\small
\begin{minipage}{7cm}
\begin{tabular}{||l|l|l|l||}
\hline
Nr.	&Grp 	&CC-vector	&link\\\hline
\multicolumn{4}{||c||}{$5$-hedrites}\\\hline
{\bf 13-1${}^*$}&$C_{2v}$	&$26$		&$13_{3097}$\\
{\bf 13-2}	&$C_1$	&$26$		&$13_{4054}$\\
{\bf 13-3}	&$C_2$	&$6^2; 14$	&????\\
{\bf 13-4}	&$C_s$	&$6; 8, 12$	&????\\\hline
{\bf 14-1}	&$C_s$	&$28$		&$13_{16368}$\\
{\bf 14-2}	&$C_1$	&$6; 22$		&????\\
{\bf 14-3}	&$C_1$	&$8; 20$		&????\\
{\bf 14-4}	&$C_{2v}$	&$8^2; 12$	&????\\
{\bf 14-5}	&$C_{2v}$	&$4^3; 16$ red.	&????\\
{\bf 14-6}	&$C_{2v}$	&$6^2; 8^2$ red.!!	&????\\
{\bf 14-7}	&$C_{2v}$	&$4^3, 8^2$ red.	&????\\\hline
\hline
\multicolumn{4}{||c||}{$7$-hedrites}\\\hline
{\bf 13-1}	&$C_s$	&$26$		&$13_{3861}$\\
{\bf 13-2}	&$C_1$	&$26$		&$13_{3769}$\\
{\bf 13-3}	&$C_1$	&$6; 20$		&????\\
{\bf 13-4}	&$C_1$	&$10, 16$		&????\\
{\bf 13-5}	&$C_s$	&$10, 16$		&????\\
{\bf 13-6}	&$C_{2v}$	&$6^2; 14$	&????\\
{\bf 13-7}	&$C_{s}$	&$6^2; 14$	&????\\\hline
{\bf 14-1}	&$C_1$	&$28$		&$14_{13725}$\\
{\bf 14-2}	&$C_1$	&$28$		&$14_{10841}$\\
{\bf 14-3}	&$C_1$	&$28$		&$14_{5714}$\\
{\bf 14-4}	&$C_1$	&$28$		&$14_{14207}$\\
{\bf 14-5}	&$C_1$	&$6; 22$		&????\\
{\bf 14-6}	&$C_s$	&$10, 18$		&????\\
{\bf 14-7}	&$C_2$	&$14^2$		&????\\
{\bf 14-8}	&$C_s$	&$6^2; 16$	&????\\
{\bf 14-9}	&$C_{2v}$	&$6^2; 16$	&????\\\hline
\hline
\multicolumn{4}{||c||}{$8$-hedrites}\\\hline
{\bf 13-1}	&$C_2$	&$26$		&$13_{3478}$\\
{\bf 13-2}	&$C_{2v}$	&$6^2; 14$	&????\\\hline
{\bf 14-1}	&$C_2$	&$28$		&$14_{17895}$\\
{\bf 14-2}	&$C_s$	&$6; 22$		&????\\
{\bf 14-3}	&$D_2$	&$6; 22$		&????\\
{\bf 14-4}	&$D_{2d}$	&$14^2$		&????\\
{\bf 14-5}	&$C_2$	&$6^2; 16$	&????\\
{\bf 14-6}	&$D_2$	&$8; 10^2$	&????\\
{\bf 14-7}	&$D_{4h}$	&$6^2, 8^2$	&????\\
{\bf 14-8}	&$D_{4h}$	&$4^3, 8^2$ red.	&????\\\hline
\end{tabular}
\end{minipage}
\begin{minipage}[t]{7cm}
\begin{tabular}{||l|l|l|l||}
\hline
Nr.	&Grp 	&CC-vector	&link\\\hline
\multicolumn{4}{||c||}{$4$-hedrites}\\\hline
{\bf 14-1${}^*$}&$D_{2d}$	&$14^2$		&????\\\hline
{\bf 14-2}	&$D_2$	&$14^2$		&????\\	
{\bf 14-3${}^*$}&$D_{2h}$       &$2^7, 14$ red.  &????\\
\hline
\multicolumn{4}{||c||}{$6$-hedrites}\\\hline
{\bf 13-1}	&$C_2$	&$26$		&$13_{1739}$\\
{\bf 13-2} 	&$C_2$	&$26$		&$13_{3586}$\\
{\bf 13-3} 	&$C_2$	&$26$		&$13_{1345}$\\
{\bf 13-4} 	&$C_s$	&$26$		&$13_{3811}$\\
{\bf 13-5} 	&$C_1$	&$26$		&$13_{1485}$\\
{\bf 13-6}	&$C_1$	&$26$		&$13_{3957}$\\
{\bf 13-7}	&$C_1$	&$26$		&$13_{2957}$\\
{\bf 13-8}	&$C_1$	&$8; 18$		&????\\
{\bf 13-9}	&$C_s$	&$12, 14$		&????\\
{\bf 13-10}	&$C_s$	&$4^2; 18$ red.	&????\\
{\bf 13-11}	&$C_{2v}$	&$8^2, 10$ red.!!	&????\\
{\bf 13-12}	&$C_{2v}$	&$8^2; 10$	&????\\
{\bf 13-13}	&$C_{2v}$	&$4^3; 14$ red.	&????\\
{\bf 13-14}	&$C_s$	&$4^2,8;10$ red.	&????\\\hline
{\bf 14-1} 	&$C_{2h}$	&$28$		&$14_{17173}$\\
{\bf 14-2}	&$C_{2}$	&$28$		&$14_{17079}$\\
{\bf 14-3}	&$C_2$	&$28$		&$14_{8767}$\\
{\bf 14-4} 	&$C_2$	&$28$		&$14_{17734}$\\
{\bf 14-5} 	&$C_2$	&$28$		&$14_{17148}$\\
{\bf 14-6} 	&$C_1$	&$28$		&$14_{17309}$\\
{\bf 14-7} 	&$C_1$	&$28$		&$14_{5570}$\\
{\bf 14-8}	&$C_{2}$	&$6; 22$		&????\\
{\bf 14-9} 	&$C_2$	&$6; 22$		&????\\
{\bf 14-10}	&$C_2$	&$10; 18$		&????\\
{\bf 14-11}	&$C_2$	&$10; 18$		&????\\
{\bf 14-12}	&$C_s$	&$10, 18$		&????\\
{\bf 14-13${}^{*}$}	&$D_{2h}$	&$14^2$		&????\\
{\bf 14-14}	&$C_{2v}$	&$14^2$		&????\\
{\bf 14-15}	&$C_2$	&$12, 16$		&????\\
{\bf 14-16}	&$C_{s}$	&$12, 16$		&????\\
{\bf 14-17}	&$C_2$	&$4^2; 20$ red.	&????\\
{\bf 14-18}	&$C_2$	&$8; 10^2$	&????\\
{\bf 14-19}	&$C_2$	&$6^2; 16$	&????\\
{\bf 14-20}	&$D_{2h}$	&$6^2, 8^2$	&????\\
{\bf 14-21}	&$C_{2v}$	&$6^3, 10$ red.	&????\\
{\bf 14-22}	&$C_{2h}$	&$4^2; 10^2$ red.	&????\\
{\bf 14-23}	&$C_{2v}$	&$4^2, 8; 12$ red.	&????\\\hline
\end{tabular}
\end{minipage}
}
\end{center}
\caption{All $i$-hedrites with $13$ and $14$ vertices}
\label{tab:i-hedrite13_14}
\end{table}











\begin{thebibliography}{99}


%\bibitem[Alex50]{A}
%A.D.Alexandrov, {\em Vypuklie Mnogogranniki},
%GITL, Moscow, 1950. Translated in German as {\em Convexe Polyheder},
%\item Akademie-Verlag, Berlin, 1958.

\bibitem[Cox71]{Cox71}
H.S.M.Coxeter, {\em Virus macromolecules and geodesic domes}, in {\em A spectrum of mathematics}; ed. by J.C.Butcher, Oxford University Press/Auckland University Press: Oxford, U.K./Auckland New-Zealand, (1971) 98--107.


\bibitem[DDF02]{DDF}
M.Deza, M.Dutour and P.W.Fowler,
{\em Zigzags, Rail-roads and Knots in Fullerenes},
submitted.

!!add here, perhaps, as [DD02] zig2!!

\bibitem[DeGr99]{DG2}
M.Deza and V.P.Grishukhin,
{\em $l_1$-embeddable polyhedra},
in: Algebras and Combinatorics, Int. Congress CAC '97 Hong Kong,
ed. by K.P. Shum et al., Springer-Verlag (1999), pp. 189--210.


\bibitem[DHL02]{DHL}
M.Deza, T.Huang and K-W.Lih,
{\em Central Circuit Coverings of Octahedrites and Medial Polyhedra},
Journal of Math. Research \& Exposition {\bf 22-1} (2002) 49--66.


\bibitem[DeSt02]{DSt}
M.Deza and M.Shtogrin,
{\em Octahedrites}, 
Symmetry, Special Issue ``Polyhedra and Science and Art'', 2002.


\bibitem[GaKe94]{GK}
M.L.Gargano and J.W.Kennedy,
{\em Gaussian graphs and digraphs}, Congressus Numerantium {\bf 101}
(1994) 161--170.


\bibitem[GoRo01]{God}
C.Godsil and G.Royle, {\em Algebraic Graph Theory}, Graduate Texts in 
Mathematics {\bf 207}, Springer-Verlag, Berlin - New York, 2001.


\bibitem[Gold37]{Gold37}
M.Goldberg, {\em A class of multisymmetric polyhedra}, Tohoku Math.
Journal, {\bf 43} (1937) 104--108.


\bibitem[Gr\"{u}n67]{Gr}
B.Gr\"{u}nbaum, {\em Convex polytopes}, Interscience, New York, 1967.


\bibitem[Gr\"{u}n72]{Gr2}
B.Gr\"{u}nbaum, {\em Arrangements and Spreads}, Regional Conference Series in
Mathematics {\bf 10}, American Mathematical Society, 1972.


\bibitem[Gr\"{u}nMo63]{GrMo}
B.Gr\"{u}nbaum and T.S.Motzkin, {\em The number of hexagons and the simplicity
of geodesics on certain polyhedra}, Canadian Journal of Mathematics {\bf 15} (1963) 744--751.


\bibitem[Harb97]{Ha}
H.Harborth, {\em Eulerian straight ahead cycles in drawings of complete
bipartite graphs}, Bericht 97/23, Institute f\"{u}r Mathematik, Tech. 
Universit\"{a}t
Braunschweg, 1997.


\bibitem[Heid98]{He}
O.Heidemeier, {\em Die Erzeugung von 4-regul\"{a}ren, planaren,
simplen, zusammenh\"{a}ngenden Graphen mit vorgegebenen Fl\"{a}chentypen},
Diplomarbeit, Universit\"{a}t Bielefeld, Fakult\"{a}t f\"{u}r Wirtschaft und
Mathematik, 1998. 

\bibitem[Jeo95]{Je}
D.Jeong, {\em Realizations with a cut-through Eulerian circuit},
Discrete Mathematics {\bf 137} (1995) 265--275.


%\bibitem[Kau87]{Kau}
%L.H.Kauffman, {\em On knots}, Princeton University Press, Princeton, New
%Jersey, 1987.

\bibitem[Kaw96]{Kaw}
A.Kawauchi, {\em A survey of knot theory}, Birkh\"{a}user, 1996.


\bibitem[Kir85]{Kirk}
T. Kirkman, {\em The enumeration, description, and construction of knots with fewer than $10$ crossings}, Trans. Roy. Soc. Edin. {\bf 32} (1885), 281--309.


\bibitem[Kot69]{Ko}
A.Kotzig, {\em Eulerian lines in finite 4-valent graphs and their 
transformations}, in: Theory of Graphs, Proceedings of a colloquium, 
Tihany 1966, ed. by P.Erdos and G.Katona, Academic Press, 
New York (1969), pp. 219--230.


\bibitem[PTZ96]{PTZ}
T.Pisanski, T.Tucker and A.Zitnik, {\em Eulerian Embedding of Graphs},
University of Ljubljana, IMMF Preprint Series {\bf 34}
(1996) 531.


\bibitem[Rol76]{Rolf}
D.Rolfsen, {\em Knots and Links}, Mathematics Lecture Series 7, Publish or
Perish, Berkeley, 1976;
second corrected printing: Publish or Perish, Houston, 1990.


\bibitem[Sha75]{Sh}
H.Shank, {\em The Theory of Left-Right Paths}, in: Combinatorial 
Mathematics III,
Proceedings of 3rd Australian Conference, St Lucia 1974, Lecture Notes in
Mathematics {\bf 452}, Springer-Verlag, Berlin - New York (1975),  pp. 42--54.


\bibitem[Thi]{T}
M.Thistlewaite, \url{http://www.math.utk.edu/~morwen}.


\bibitem[Dut]{Dut}
M.Dutour, {\em PlanGraph, a gap package for Planar Graph}, in preparation


\bibitem[DD]{DD}
M.Dutour and!! M.Deza, {\em Goldberg-Coxeter Construction for 
Convex!! Polyhedra!!}, in preparation.
!!perhaps, move this and Dutour up, i.e. just after [DeSt02]!!
!!OK about this order of authorship!!
!!perhaps?? write 03, i.e. 2003 instead of in preparation!!

\bibitem[Liu]{Liu98!!}
Liu Yanpei, {\em Embedding in Graphs}, Kluwer, Dodrecht, 1998.
!!put it after Kotzig!!

%\bibitem[Weis99]{Weis}
%E.W.Weisstein, {\em CRC Concise Encyclopedia of Mathematics},
%Chapman and Hall/CRC, Boca Raton, 1999. 


 
\end{thebibliography}


\end{document}
